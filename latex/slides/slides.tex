\documentclass[11pt, aspectratio=1610, xcolor={dvipsnames}]{beamer}

% font packages
\usepackage[utf8]{inputenc}
\usepackage[T1]{fontenc}
\usepackage{lmodern}

% misc packages
\usepackage{amsmath}
\usepackage{booktabs}
\usepackage{mathtools}
\usepackage{multirow}
\usepackage[round]{natbib}
\usepackage{transparent}

% theme and colours
\usetheme{Montpellier}
\setbeamercolor*{structure}{bg=PineGreen!10,fg=PineGreen}
\setbeamercolor*{footline}{bg=PineGreen!10,fg=PineGreen}

% line spread
\linespread{1.5}

% paths
\graphicspath{{../thesis/} {../../stata/figures/pdf_tex/}}

\makeatletter
\def\input@path{{../../stata/figures/pdf_tex/}}
\makeatother

\pdfsuppresswarningpagegroup=1

% set apa style
\bibliographystyle{apalike}


\title[Putting the GDR's Legacy Effect under the Microscope]{\texorpdfstring{Putting the GDR's Legacy Effect \linebreak under the Microscope:}{Putting the GDR's Legacy Effect under the Microscope:}}
\subtitle{Eastern Female STEM Professionals in Reunified Germany.}

%\author{\href{mailto:johannes.schuett@fu-berlin.de}{Johannes Schütt}}
\author{\texorpdfstring{\href{mailto:johannes.schuett@fu-berlin.de}{Johannes Schütt}}{Johannes Schütt}}

\institute{\texorpdfstring{Free University of Berlin \linebreak M.Sc. Public Economics \linebreak\linebreak Supervisor: Prof. Natalia Danzer, Ph.D.}{}}

\date{4 July 2024}

% logo on cover page
\titlegraphic{
	\vspace{5mm}
	\includegraphics[width=3cm]{fu_logo.pdf}
}

% define custom footline
\newcommand{\Ffootline}{%
	\hfill
	\insertframenumber/\inserttotalframenumber % The center
	\hfill
}

% set footline
\setbeamertemplate{navigation symbols}{}
\setbeamertemplate{footline}{%
	\ifnum\thepage=1
	\begin{beamercolorbox}[wd=\paperwidth,ht=0ex,dp=1ex]{footlinecolor}
	\end{beamercolorbox}
	\else
	\begin{beamercolorbox}[wd=\paperwidth,ht=0ex,dp=1ex]{footlinecolor}%
		\tiny\hspace*{4mm} \Ffootline \hspace*{4mm}
	\end{beamercolorbox}
	\fi
}

% bottom right transparent logo
\logo{
	{\transparent{0.6}
		\includegraphics[width = 1cm]{fu_seal.pdf}
		\hspace{1mm}
		\vspace{1mm}
	}
}

% sectiontitleframes
\AtBeginSection[]{
	\begin{frame}
		\vfill
		\centering
		\begin{beamercolorbox}[sep=8pt,center,shadow=false,rounded=false]{title}
			\usebeamerfont{title}\insertsectionhead\par%
		\end{beamercolorbox}
		\vfill
	\end{frame}
}

% special fonts
\newcommand{\highlight}[1]{\textbf{\textcolor{PineGreen}{#1}}}
\newcommand{\todo}[1]{\textbf{\textcolor{red}{#1}}}

% frame continuation for references
\makeatletter
\setbeamertemplate{frametitle continuation}{\gdef\beamer@frametitle{}}
\makeatother

% table def
\def\sym#1{\ifmmode^{#1}\else\(^{#1}\)\fi}

% change alignment for single cell inside tabular
\newcommand*{\myalign}[2]{\multicolumn{1}{#1}{#2}}


\begin{document}
	
	\begin{frame}[plain]
		\maketitle
	\end{frame}
	
	\begin{frame}
		\frametitle{}
		{\linespread{1}
		\tableofcontents
		}
	\end{frame}
	
	\section{Motivation}
	\begin{frame}
		\frametitle{}
		
		\begin{itemize}
			\item How do \highlight{social norms} influence our occupational decisions?
			\item Do social norms  \highlight{persist} when \highlight{institutions that uphold them vanish}?
			\item Consider these questions with regard to \highlight{Eastern females} who work in \highlight{traditionally male-dominated work sectors} after socialist \highlight{GDR collapses}.
		\end{itemize}
		
	\end{frame}
	
	\subsection{Previous Literature}
	\begin{frame}
		\frametitle{Previous Literature}
		
		\begin{itemize}
			\item Econometric literature describes \highlight{persistence of East German social norms and culture} after Reunification.
			\item \cite{Lippmann2018} use PISA data from \highlight{2003} to show that \highlight{performance gap of girls in maths} in former GDR is being reduced.
			\item \cite{Jessen2023} find that an above-median \highlight{inflow of East German migrants} into West Germany led to \highlight{increase in labour supply} among employed \highlight{West German females}.
		\end{itemize}
		
	\end{frame}
	
	\begin{frame}
		\frametitle{}
		
		\begin{itemize}
			\item \cite{FriedmanSokuler2020} investigate \highlight{vertical persistence} (across generations) and \highlight{horizontal diffusion} (through peers) of socialist norms.
			\item Jewish immigrants from former \highlight{Soviet Union to Israel} in \highlight{early 1990s}.
			\item Females from FSU tend to \highlight{avoid work in educational or social sector} and instead opt for \highlight{STEM} fields.
		\end{itemize}
		
	\end{frame}
	
	\begin{frame}
		\frametitle{}
		
		\textcolor{PineGreen}{\underline{Definition.}} STEM, field and curriculum centred on education in disciplines of \textbf{S}cience, \textbf{T}echnology, \textbf{E}ngineering, and \textbf{M}athematics \textcolor{darkgray}{\citep{Hallinen2024}}.
		
	\end{frame}
	
	\begin{frame}
		\frametitle{}
		
		\begin{itemize}
			\item Even females who \highlight{immigrated as infants} are more likely to pursue STEM subjects in education than native Israeli females \highlight{(vertical persistence)}.
			\item \highlight{Native Israeli females} who attend school with high concentration of FSU immigrants shift their educational choices towards STEM \highlight{(horizontal diffusion)}.
		\end{itemize}
		
	\end{frame}
	
	\subsection{Developing a Research Question}
	\begin{frame}
		\frametitle{Developing a Research Question}
		
		\textcolor{PineGreen}{\underline{Initial Idea.}} Do Eastern Female STEM professionals have an impact on Western Females after the Reunification?
		
		\vspace{1cm}
		
		General trend in Germany regarding STEM professions using SOEP…
		
	\end{frame}
	
	\begin{frame}
		\frametitle{}
		
		\begin{itemize}
			\item \highlight{Share of STEM professionals} within demographic groups in Germany over time.
			\item Group by: \highlight{Eastern origin} (location in 1989) and \highlight{gender}.
			\item Inspect survey years 1990 (year of Reunification) -- 2017.
			\item All individuals in \highlight{working age} (17--65).
			\item Both employed and unemployed.
			\item Apply SOEP's individual weights.
		\end{itemize}
		
	\end{frame}
	
	\begin{frame}
		\frametitle{}
		
		{\linespread{1}\small
			\begin{table}[h]
				\centering
				\caption{Examples of STEM Professions (sorted by Frequency)}
				\label{tab:stem_examples}
				
				\begin{tabular}{ll}
					\toprule
					ISCO-88 Code & Professional title\\
					\midrule
					{[2149]} & Architects, engineers and related scientists\\
					{[3152]} & Health, environmental and quality inspectors\\
					{[3119]} & Materials and engineering specialists\\
					{[2142]} & Civil engineers\\
					{[3121]} & Data processing assistants\\
					{[3111]} & Chemo- and physicotechnician\\
					{[2145]} & Mechanical engineers\\
					{[3114]} & Electronics and telecommunications technology\\
					{[3120]} & Data processing specialists\\
					{[2144]} & Electronics and telecommunications engineers\\
					… &\\
					\bottomrule
				\end{tabular}
			\end{table}
		}
		
	\end{frame}
	
	\begin{frame}
		\frametitle{}
		
		\begin{figure}[h]
			\centering
			\caption{Time Trend (by Region and Gender), 1984--2017}
			\label{fig:trend}
			\resizebox{75mm}{!}{\input{trend.pdf_tex}}
		\end{figure}
		
		{\scriptsize
			\textcolor{darkgray}{Source: \cite{SOEP2023}. Own calculations.}
		}
		
		\hyperlink{coefficients}{\beamerbutton{Coefficient Graph}}
	
	\end{frame}
	
	\begin{frame}
		\frametitle{}
		
		\hypertarget{graph}{}
		
		\begin{figure}[h]
			\centering
			\caption{Time Trend (by Region and Gender), 1984--2017}
			\label{fig:trend_highlight}
			\resizebox{75mm}{!}{\input{trend_highlight.pdf_tex}}
		\end{figure}
		
		{\scriptsize
			\textcolor{darkgray}{Source: \cite{SOEP2023}. Own calculations.}
		}
		
		\hyperlink{coefficients}{\beamerbutton{Coefficient Graph}}
		
	\end{frame}
	
	\begin{frame}
		\frametitle{}
		
		\begin{figure}[h]
			\centering
			\caption{Time Trend (by Region, Females only), 1990--1999}
			\label{fig:trend_zoomed}
			\resizebox{75mm}{!}{\input{trend_zoomed.pdf_tex}}
		\end{figure}
		
		{\scriptsize
			\textcolor{darkgray}{Source: \cite{SOEP2023}. Own calculations.}
		}
		
	\end{frame}
	
	\begin{frame}
		
		\textcolor{PineGreen}{\underline{Research Question.}} What drives the downward trend of East German females in STEM professions in the first decade after Reunification?
		
	\end{frame}
	
	\subsection{Historical Background}
	\begin{frame}
		\frametitle{Historical Background}
		
		\begin{itemize}
			\item 92\% of all \highlight{female apprentices} in GDR were concentrated in thirteen \highlight{occupational groups}, compared to five in FRG \textcolor{darkgray}{\citep{Menschik1974}}.
			\item \highlight{Horizontal segregation} in labour sector in GDR was not nearly as high as in societies of Western bloc \textcolor{darkgray}{\citep{Lane1983}}.
		\end{itemize}
		
	\end{frame}
	
	\begin{frame}
		\frametitle{}
		
		\begin{itemize}
			\item \highlight{17\% of female apprentices} who successfully passed examination in 1980 were in \highlight{STEM field} \textcolor{darkgray}{\citep[p. 293]{DDRJahrbuch1981}}.
			\item Unlike in FRG, in GDR there was the \highlight{same school curriculum for boys and girls} \textcolor{darkgray}{\citep{FuchsSchuendeln2016, Lippmann2018}}.
		\end{itemize}
		
	\end{frame}
	
	\iffalse
	\begin{frame}
		\frametitle{}
		
		\begin{itemize}
			\item In \highlight{1949}, females made up \highlight{40\% of total labour force} both in GDR and in FRG.
			\item Unlike West German females, \highlight{East German females remained in labour market} after economic upturn in 1960s \textcolor{darkgray}{\citep{Menschik1974}}.
		\end{itemize}
		
		%\todo{Notizen einarbeiten!}
		
	\end{frame}
	\fi
	
	\begin{frame}
		\frametitle{}
		
		\begin{itemize}
			\item \highlight{\emph{Chemiedreieck}}: Halle (Saale), Merseburg, and Bitterfeld.
			\item Strong \highlight{lignite reserves}.
			\item 1893: \highlight{industrial chemical plants} processed waste products of coal processing into valuable chemicals.
			\item GDR period: \highlight{crude oil} is supplied from Soviet Union to \emph{Chemiedreieck} for \highlight{refining} via \emph{Druzhba} pipeline.
		\end{itemize}
		
	\end{frame}
	
		\begin{frame}
		\frametitle{}
		
		\begin{itemize}
			\item \highlight{High density of STEM professionals} in agglomeration.
			\item \highlight{Largest industrial area} in GDR.
			\item Extremely \highlight{high environmental pollution} in the entire region.
			\item After reunification: administration through \emph{Treuhandanstalt} and \highlight{partial dissolution}.
		\end{itemize}
		
	\end{frame}
	
	\section{Empirical Approach}
	\begin{frame}
		\frametitle{}
		
		\begin{itemize}
			\item Compare \highlight{East German females} with  \highlight{East German males}.
			\begin{itemize}
				\item East German males and females had \highlight{same probability}\linebreak of working as a \highlight{STEM} professional in \highlight{1990} (what about before?).
				\item Examine in what way they each were affected by Reunification.
			\end{itemize}
			\item Regressor of interest: \highlight{years after Reunification $\times$ being female}.
			\item Use suitable controls to \highlight{explain what drives downward slope}.
		\end{itemize}
		
		
	\end{frame}
	
	\subsection{Econometric Specification}
	\begin{frame}
		\frametitle{Econometric Specification}
		
			{\linespread{1}\normalsize
				\begin{align}
					&\begin{aligned}
						\label{eq:spec}
						STEM_{itr} = & \alpha + \sum_{t=1}^{9}\beta_{t} SurveyYear_{t} + \sum_{t=1}^{9}\beta_{t + 9} (SurveyYear_{t} \times Female_{i}) +\\
						& \gamma Female_{i} + X_{itr} \delta + \epsilon_{itr}
					\end{aligned}
				\end{align}
			}
		
		\vspace{1cm}
		
		with $SurveyYear_{t}$ being a binary variable and $t = 0$ in 1990.
		
	\end{frame}
	
	\subsection{Data}
	\begin{frame}
		\frametitle{Data}
		
		\begin{itemize}
			\item Use \highlight{SOEP} data; survey years \highlight{1990--1999}.
			\item 19,438 observations (\highlight{18,962} using person weights) of \highlight{4,910 unique individuals}.
			\item \highlight{8\%} are employed in \highlight{STEM} sector $\longrightarrow$ Cloglog! % those are all unweighted percentages!
			\item 47\% are female.
			\item 94\% are from sample C ``1990 Initial Sample (East)''.
			\item \highlight{All} of them are of \highlight{East German origin} (all lived in GDR in 1989).
		\end{itemize}
		
	\end{frame}
	
	\section{Preliminary Results}
	\begin{frame}
		\frametitle{}
		{\linespread{1}\tiny
			\begin{table}[h]
				\centering
				\caption{Descriptive Statistics for East Germans in Survey Year 1990}
				\label{tab:descriptives}
				\begin{tabular}{l*{3}{c}}
					\toprule
					& (Females) & (Males) & (Males $-$ Females) \\
					\midrule
					STEM Profession     &        0.10&             0.11&          0.00      \\
					&     (0.31)&           (0.31)&           (0.01)\\
					\addlinespace
					Age                 &      38.05&          38.73&        0.68 \\
					&     (11.35)&        (11.96)         &      (0.40)\\
					\addlinespace
					Spouse/Life Partner &        0.80&            0.81&        0.01       \\
					&      (0.40)&          (0.39)&         (0.01)\\
					\addlinespace
					Household Size      &        3.22&         3.29&           0.07        \\
					&          (1.08)&       (1.10)        &      (0.04)\\
					\addlinespace
					Monthly Household Income (Net)&     1,005.93&       1,041.08&      35.15\sym{**} \\
					&    (339.99)&      (346.39)&         (11.89)\\
					\addlinespace
					Residence in West Germany&        0.00&        0.00&         0.00         \\
					&         (0.03)&       (0.02)&          (0.00)\\
					\addlinespace
					Residence in Chemiedreieck&        0.17&        0.17&         0.00     \\
					&          (0.38)&       (0.38)&            (0.01)\\
					\midrule
					Observations        &        1,595&             1,739&            3,334               \\
					\bottomrule
				\end{tabular}
			\end{table}
		}
		
		{\scriptsize
			\textcolor{darkgray}{Source: \cite{SOEP2023}. Own calculations.}
		}
		
	\end{frame}
		
	\subsection{Main Findings}
	\begin{frame}
		\frametitle{Main Findings}
		
		\begin{itemize}
			\item Use \eqref{eq:spec} to \highlight{model downward trend of Eastern females in STEM}.
			\item Add \highlight{controls} in \highlight{stepwise} manner.
			\item Run person-weighted Cloglog regressions.
			\item Cluster standard errors at household level.
		\end{itemize}
		
	\end{frame}
	
	\begin{frame}
		\frametitle{}
		
		{\linespread{1}\tiny
			\begin{table}[h]
				\centering
				\caption{Margins (dy/dx) of Cloglog Regressions with y = being a STEM Professional}
				\label{tab:margins1}
				\begin{tabular}{p{25em}*{4}{c}}
					\toprule
					&\multicolumn{1}{c}{(1)}         &\multicolumn{1}{c}{(2)}         &\multicolumn{1}{c}{(3)}         &\multicolumn{1}{c}{(4)}         \\
					\midrule
					Residence in West Germany&                     &      0.0510\sym{**} &      0.0451\sym{**} &      0.0417\sym{**} \\
					&                     &    (0.0222)         &    (0.0201)         &    (0.0201)         \\
					\addlinespace
					Residence in West Germany $\times$ Female&                     &     -0.0870\sym{**} &     -0.0751\sym{**} &     -0.0794\sym{**} \\
					&                     &    (0.0349)         &    (0.0331)         &    (0.0333)         \\
					\midrule
					Observations        &       18,962         &       18,962         &       18,962         &       18,962         \\
					\bottomrule
				\end{tabular}
			\end{table}
		}
		
		{\scriptsize
			\textcolor{darkgray}{Source: \cite{SOEP2023}. Own calculations.}
		}
		
	\end{frame}
	
	\begin{frame}
		\frametitle{}
		
		{\linespread{1}\tiny
			\begin{table}[h]
				\centering
				\caption{Margins (dy/dx) of Cloglog Regressions with y = being a STEM Professional \emph{(cont.)}}
				\label{tab:margins2}
				\begin{tabular}{p{25em}*{4}{c}}
					\toprule
					&\multicolumn{1}{c}{(1)}         &\multicolumn{1}{c}{(2)}         &\multicolumn{1}{c}{(3)}         &\multicolumn{1}{c}{(4)}         \\
					\midrule
					Spouse/Life Partner &                     &                     &      0.0128         &      0.0126         \\
					&                     &                     &    (0.0207)         &    (0.0208)         \\
					\addlinespace
					Spouse/Life Partner $\times$ Female&                     &                     &     -0.0178         &     -0.0170         \\
					&                     &                     &    (0.0254)         &    (0.0255)         \\
					\addlinespace
					Household Size      &                     &                     &     -0.0028         &     -0.0024         \\
					&                     &                     &    (0.0052)         &    (0.0051)         \\
					\addlinespace
					Household Size $\times$ Female&                     &                     &      0.0115\sym{*}  &      0.0111         \\
					&                     &                     &    (0.0069)         &    (0.0068)         \\
					\addlinespace
					Monthly Household Income in Thousands (Net)&                     &                     &      0.0000\sym{***}&      0.0000\sym{***}\\
					&                     &                     &    (0.0000)         &    (0.0000)         \\
					\addlinespace
					Monthly Household Income in Thousands (Net) $\times$ Female&                     &                     &     -0.0000         &     -0.0000         \\
					&                     &                     &    (0.0000)         &    (0.0000)         \\
					\midrule
					Observations        &       18,962         &       18,962         &       18,962         &       18,962         \\
					\bottomrule
				\end{tabular}
			\end{table}
		}
		
		{\scriptsize
			\textcolor{darkgray}{Source: \cite{SOEP2023}. Own calculations.}
		}
		
	\end{frame}
	
	\begin{frame}
		\frametitle{}
		
		{\linespread{1}\tiny
			\begin{table}[h]
				\centering
				\caption{Margins (dy/dx) of Cloglog Regressions with y = being a STEM Professional \emph{(cont.)}}
				\label{tab:margins3}
				\begin{tabular}{p{25em}*{4}{c}}
					\toprule
					&\multicolumn{1}{c}{(1)}         &\multicolumn{1}{c}{(2)}         &\multicolumn{1}{c}{(3)}         &\multicolumn{1}{c}{(4)}         \\
					\midrule
					Residence in Chemiedreieck&                     &                     &                     &      0.0091         \\
					&                     &                     &                     &    (0.0156)         \\
					\addlinespace
					Residence in Chemiedreieck $\times$ Female&                     &                     &                     &     -0.0246         \\
					&                     &                     &                     &    (0.0222)         \\
					\midrule
					Observations        &       18,962         &       18,962         &       18,962         &       18,962         \\
					\bottomrule
				\end{tabular}
			\end{table}
		}
		
		{\scriptsize
			\textcolor{darkgray}{Source: \cite{SOEP2023}. Own calculations.}
		}
		
	\end{frame}
	
		\begin{frame}
		\frametitle{}
		
		\hypertarget{coefficients}{}
		
		\begin{figure}[h]
			\centering
			\caption{Coefficients of $SurveyYear_{t} \times Female_{i}$, 1990--1999}
			\label{fig:coefficient_trend}
			\resizebox{75mm}{!}{\input{coefficient_trend.pdf_tex}}
		\end{figure}
		
		{\scriptsize
			\textcolor{darkgray}{Source: \cite{SOEP2023}. Own calculations.}
		}
		
		\hyperlink{graph}{\beamerbutton{General Time Trend}}
		
	\end{frame}
	
	\subsection{Change of Occupation}
	\begin{frame}
		\frametitle{Change of Occupation}
		
		\begin{itemize}
			\item Effect driven by \highlight{panel attrition}?
			\item New panel \highlight{entrants}?
			\item What do Eastern females do \highlight{after quitting STEM}?
		\end{itemize}
		
	\end{frame}

	\begin{frame}
		\frametitle{}
		
		\begin{figure}[h]
			\centering
			\caption{Net Switches among Eastern Females, 1990--1999}
			\label{fig:eastern_female_tracking}
			\resizebox{75mm}{!}{\input{eastern_female_tracking.pdf_tex}}
		\end{figure}
		
		{\scriptsize
			\textcolor{darkgray}{Source: \cite{SOEP2023}. Own calculations.}
		}
		
	\end{frame}
		
	\section{Conclusion}	
	\begin{frame}
		\frametitle{}
		
		\begin{itemize}
			\item Both controlling for \highlight{residence in West} and \highlight{residence in \emph{Chemiedreieck}} significantly reduces size of coefficient of \highlight{regressor of interest} when interacting them with \highlight{being female}.
			\item In both cases, effect is positive for \highlight{males}, while it is negative for \highlight{females}.
			\item No significant explanatory power through \highlight{household controls}.
			\begin{itemize}
				\item[$\rightarrow$] Theory of \highlight{vertical cultural persistence} might hold within households!
			\end{itemize}
		\end{itemize}
		
	\end{frame}
	
	\begin{frame}
		\frametitle{}
		
		\begin{itemize}
			\item \highlight{Deindustrialisation of East Germany} especially \highlight{hurt East German females in scientific and technological occupations}.
			\item \highlight{Restructuring of \emph{Chemiedreieck}} is an example for that.
			\item They do not switch to unemployment, but instead scatter into \highlight{other sectors}.
			\item \highlight{Infrastructural shock} with \highlight{long-term consequences}.
		\end{itemize}
		
	\end{frame}
	
	\section*{References}
	\begin{frame}[allowframebreaks]
		\frametitle{References}
		
		{\scriptsize
		\bibliography{../references.bib}
		}
		
	\end{frame}
	
	\appendix
	
	\section{Appendix}
	\begin{frame}
		\frametitle{}
		{\linespread{1}\tiny
			\begin{table}[h]
				\centering
				\caption{Descriptive Statistics for West Germans in Survey Year 1990}
				\label{tab:descriptiveswest}
				\begin{tabular}{l*{3}{c}}
					\toprule
					& (Females) & (Males) & (Males $-$ Females) \\
					\midrule
					STEM Profession     &        0.04&       0.10&         0.06\sym{***}\\
					&       (0.20)&       (0.30)&           (0.02)\\
					\addlinespace
					Age                 &       30.16&           30.29&          0.13         \\
					&        (9.91)&         (10.27)&             (0.61)\\
					\addlinespace
					Spouse/Life Partner &        0.63&            0.56&         -0.07\sym{*}  \\
					&         (0.48)&         (0.50)&        (0.03)\\
					\addlinespace
					Household Size      &        3.23&           3.30&         0.07        \\
					&       (1.52)&        (1.65)       &      (0.10)\\
					\addlinespace
					Monthly Household Income (Net)&     2,154.94&      1,983.31&       -171.64     \\
					&     (2447.84)&      (891.03)&    (114.27)\\
					\addlinespace
					Residence in West Germany&        0.96&         0.97&            0.01      \\
					&       (0.18)&        (0.17)&         (0.01)\\
					\addlinespace
					Residence in Chemiedreieck&        0.00&            0.00&            0.00       \\
					&           (0.00)&          (0.00)&         (0.00)\\
					\midrule
					Observations        &        1,595&             1,739&            3,334               \\
					\bottomrule
				\end{tabular}
			\end{table}
		}
		
		{\scriptsize
			\textcolor{darkgray}{Source: \cite{SOEP2023}. Own calculations.}
		}
		
	\end{frame}
	
	\begin{frame}
		\frametitle{}
		
		\begin{figure}[h]
			\centering
			\caption{Cohorts (Start of Occupation by Gender), 1945--1990}
			\label{fig:validity}
			\resizebox{75mm}{!}{\input{validity.pdf_tex}}
		\end{figure}
		
		{\scriptsize
			\textcolor{darkgray}{Source: \cite{Mayer1995}. Own calculations.}
		}
		
	\end{frame}
	
	\begin{frame}
		\frametitle{}
		
		{\linespread{1}\tiny
			\begin{table}[h]
				\centering
				\caption{Share of Females in the Skilled Labour Force by Economic Sector in 1971 (in \%)}
				\label{tab:gdr_yearbook}
				
				\begin{tabular}{llrrrrrrr}
					\toprule
					Labour sector & Selected Segment & \multicolumn{5}{c}{Below the age of} & & \multirow[c]{2}{*}{Total} \\
					& & \myalign{c}{25} & \myalign{c}{30} & \myalign{c}{40} & \myalign{c}{50} & \myalign{c}{60} & \myalign{c}{60\texttt{+}} & \\
					\midrule
					Technical sciences & & 32.6 & 16.4 & 8.6 & 4.3 & 3.0 & 1.1 & 9.3 \\
					& Mechanical engineering & 24.2 & 10.1 & 4.1 & 1.1 & 0.5 & 0.2 & 4.8 \\
					& Textile technology (mechanical) & 84.0 & 67.8 & 41.3 & 23.3 & 16.8 & 5.2 & 36.7 \\
					& Chemical engineering & 60.3 & 50.3 & 39.1 & 38.0 & 29.2 & 17.2 & 42.1 \\
					& Automation engineering & 18.3 & 6.1 & 3.8 & 0.6 & 1.4 & 0.0 & 5.8 \\
					& Electrical engineering & 13.6 & 4.1 & 2.7 & 1.2 & 0.3 & 0.1 & 2.7 \\
					& Energy technology & 16.3 & 8.9 & 4.6 & 2.9 & 2.2 & 0.9 & 6.1 \\
					& Construction industry technology & 34.3 & 17.6 & 6.3 & 1.6 & 0.5 & 0.2 & 6.6 \\
					& Mining engineering &  15.8 & 3.5 & 1.4 & 0.5 & 0.3 & 0.0 & 1.5 \\
					Economic sciences & & 72.4 & 53.2 & 37.5 & 18.8 & 15.0 & 8.9 & 30.4 \\
					Medicine, Agricultural sciences & & 53.4 & 40.3 & 29.8 & 14.7 & 10.2 & 4.1 & 25.3 \\
					& Medical and Pharmacy technology & 96.7 & 89.9 & 84.2 &  64.8 & 36.6 & 20.1 & 76.8 \\
					& Agricultural sciences & 46.0 & 29.3 & 21.6 & 10.3 & 6.7 & 1.3 & 18.3 \\
					Cultural, Educational, and Sports sciences & & 95.4 & 86.8 & 80.0 & 72.1 & 64.9 & 43.3 & 80.0 \\
					Literature, Journalism & & 71.4 & 46.8 & 35.6 & 34.6 & 32.7 & 17.3 & 34.9 \\
					\bottomrule
				\end{tabular}
			\end{table}
		}
		
		{\scriptsize
			\textcolor{darkgray}{Source: \cite[p. 442]{DDRJahrbuch1973}. Own calculations.}
		}
		
		
	\end{frame}
	
\end{document}
