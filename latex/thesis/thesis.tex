\documentclass[a4paper, oneside, hyperfootnotes = false]{article}

\usepackage[utf8]{inputenc}
\usepackage[T1]{fontenc}
\usepackage[UKenglish]{babel}
\usepackage{amsmath}
\usepackage{anyfontsize}
\usepackage{bookmark}
\usepackage{booktabs}
\usepackage{caption}
\usepackage{dsfont}
\usepackage{fancyvrb}
\usepackage[left = 3cm, right = 2.5cm, top = 2.5cm, bottom = 2.5cm]{geometry}
\usepackage{graphicx}
\usepackage{hyperref}
\usepackage{multirow}
\usepackage[round]{natbib}
\usepackage{pdflscape}
\usepackage{svg}
\usepackage{tocloft}

\bibliographystyle{apalike}

% suppress page group warning
\pdfsuppresswarningpagegroup=1

% line spread
\linespread{1.5}

% remove ugly boxes around hyperlinks
\hypersetup{
	pdfborder = {0 0 0}
}

% vertical space between rule and main text
\addtolength{\skip\footins}{8pt}
% vertical space between footnotes
\addtolength{\footnotesep}{5pt}

% path to figures
\graphicspath{{../../stata/figures/pdf_tex/}}

\makeatletter
\def\input@path{{../../stata/figures/pdf_tex/}}
\makeatother

% for tables
\def\sym#1{\ifmmode^{#1}\else\(^{#1}\)\fi}

% title page ------
\title{Putting the GDR's Legacy Effect under the Microscope: \linebreak Eastern Female STEM Professionals in Reunified Germany.}

\author{\href{mailto:johannes.schuett@fu-berlin.de}{Schütt, Johannes} \\
5574549}

\date{\today{}}
% title page end ---

\begin{document}

{\fontsize{12pt}{18pt}\selectfont

\maketitle

\thispagestyle{empty}
\pagenumbering{Alph}

\vspace{1.5cm}

\noindent\begin{tabular}{l}
	A Master Thesis presented to \\
	The Faculty of Economics at the Freie Universität Berlin \\
    [\normalbaselineskip]
    In Partial Fulfillment of Requirements for \\
    The Degree of Master of Science \\
    [\normalbaselineskip]
    Module Examination according to 0353cE1.1P at \\
    The Chair of Empirical Economics and Gender under \\
    The Supervision of Prof. Natalia Danzer, Ph.D. \\
    [\normalbaselineskip]
    Summer Term 2024
\end{tabular}

\vspace{3cm}

\begin{center}
    \includegraphics[width=0.4\textwidth, angle=0]{fu_logo.pdf}
\end{center}

\newpage

\pagestyle{plain}
\pagenumbering{roman}

{\fontsize{12pt}{15pt}\selectfont

\tableofcontents
\newpage

\phantomsection
\addcontentsline{toc}{section}{List of Figures}
\listoffigures

\vspace{2cm}

\phantomsection
\addcontentsline{toc}{section}{List of Tables}
\listoftables

%\newpage

\vspace{2cm}

\phantomsection
\addcontentsline{toc}{section}{List of Abbreviations}
\section*{List of Abbreviations}
\noindent\begin{tabular}{@{}ll}
	cdf & cumulative distribution function \\
	DFD & \emph{Demokratischer Frauenbund Deutschlands} \\
    DIW & German Institute for Economic Research \\
    FLFP & Female Labour Force Participation \\
    FRG & Federal Republic of Germany \\
    GDR & German Democratic Republic \\
    SOEP & German Socio-Economic Panel \\
    ISCO & International Standard Classification of Occupations \\
    pdf & probability density function \\
    PISA & Programme for International Student Assessment \\
    pp & percentage points \\
    SED & \emph{Sozialistische Einheitspartei Deutschlands} \\
    STEM & Science, Technology, Engineering, and Mathematics
\end{tabular}

\newpage

\begin{center}
{\large\bfseries Abstract}

\vspace{5mm}

\parbox{400pt}{
    \emph{\noindent Lorem ipsum dolor sit amet, consetetur sadipscing elitr, sed diam nonumy eirmod tempor invidunt ut labore et dolore magna aliquyam erat, sed diam voluptua. At vero eos et accusam et justo duo dolores et ea rebum. Stet clita kasd gubergren, no sea takimata sanctus est Lorem ipsum dolor sit amet. Lorem ipsum dolor sit amet, consetetur sadipscing elitr, sed diam nonumy eirmod tempor invidunt ut labore et dolore magna aliquyam erat, sed diam voluptua. At vero eos et accusam et justo duo dolores et ea rebum. Stet clita kasd gubergren, no sea takimata sanctus est Lorem ipsum dolor sit amet.
    }
}

\end{center}

}% end \fontsize{12pt}{15pt}\selectfont

\vspace{5mm}

\pagenumbering{arabic}
\renewcommand\thesection{\arabic{section}}

\section{Introduction}
\label{intro}

\cite{Jessen2023}
\cite{FriedmanSokuler2020}

\section{Historical Background}
\label{background}

% hour zero
During the Cold War, Germany was divided into two nation states: the socialist German Democratic Republic, which together with the other Warsaw Pact states was allied with the Soviet Union, and the Federal Republic of Germany, which together with the other NATO states was allied with the United States of America.
Since their foundation in 1949, the two countries had grown considerably apart in political and economic terms:
While the GDR operated according to a command economy based on the principles of Marxism-Leninism, the FRG employed a social market economy.
In the first year of the existence of these two states, female labour force participation (FLFP for short) was 40\% \citep[Chapter~2]{Menschik1974} for both.
However, this high share of females in the labour force can be explained by the large number of males of working age who were absent due to the war and its aftermath, and even when the husband returned to his family, one income was not enough to support them.
Following the \emph{Wirtschaftswunder} of the 1950s, FLFP declined in the FRG, while females in the GDR never returned in significant numbers to the conservative role of a housewife \citep{Ostner1991, Rosenfeld2004}.

% ulbricht's new economic system
In 1963, Walter Ulbricht, the First Secretary of the Socialist Unity Party (SED for short), sought to reform the struggling East German economic regime, which was essentially the Stalinist model that he instated in the first place. He attempted to introduce limited free-market elements, such as a more flexible price system and a certain level of autonomy granted to factory managers \citep[Chapter~4]{Grieder1999}.
This policy shift included the ideological goal of technocratising the general labour force:
At the third party conference in 1956, Ulbricht posited that workers must ``[...] acquire the most progressive insights offered by technology and science in order to increase their labour productivity [...]'' \citep{Ulbricht1956, Sanderson1981}.

% technocratisation
The GDR -- as a ``Worker and Peasant State'' -- had a labour force that largely consisted of blue-collar workers \citep{DDRJahrbuch1957}.
Therefore, every labourer should understand the purpose of every single cog in the machine they operate.
Beyond that, the SED stimulated the scientific and technological sectors in hopes of innovation that could directly translate into higher production power \citep{Hoegselius2009}.
The overarching ambition was to nurture the most pioneering scientific community in the world, which stood in direct service of socialism.
This ambition led to significant movement in these work sectors in the following decade.
Figure \ref{fig:mayer} uses data from the East German Life History Study to approximate how the share of STEM professionals developed over the duration of the GDR's existence.
For the purpose of the East German Life History Study, 2,331 East German individuals were interviewed retrospectively between September 1991 and October 1992 about their lives in the GDR.
The data cover four birth cohorts (1929--1931, 1939--1941, 1951--1953, and 1959--1961) and provide information on the family of origin, housing and household history, qualifications and educational phases, occupational histories, and further areas of the personal life.
The figure displays individuals who took up a STEM occupation as a share of all individuals who took up a new occupation in the respective decade.
As anticipated, the proportion of individuals newly employed in STEM occupations was relatively low prior to the 1960s, as the majority of workers took up occupations that served the purpose of rebuilding the critical infrastructure that had been destroyed during the war.
However, possibly as a result of the ideological shift regarding the sciences in the 1960s, there has been an increase in the share of new male STEM workers, while females stay at a lower share.
Over the last two decades of the existence of the GDR, the male and female shares have converged at approximately 10\%.

% dfd
A number of factors may be identified as contributing to the convergence observed in these shares.
First, the socialist ideology included the concept of removing traditional gender associations from specific occupations.
The law on maternal and child protection and women's rights of 1950 states that ``[...] women should be given more opportunities to work in industry, [...] and public estates, in all organs of state administration, [...] and other institutions of the German Democratic Republic.
The work of women in production should not be limited to the traditional female professions, but should extend to all branches of production, especially the electrical industry, optics, mechanical engineering, precision mechanics, the wood and furniture industry, the shoe industry, and the construction and graphic arts trades.'' \citep[§19.1]{GBl1950}.
This law created the necessity for the state to implement measures to actually facilitate the inclusion of females in technical professions.
It was recognised that there was a disparity in the starting points of females and males in this regard, given that historically, these roles have been predominantly occupied by males.
Second, the school curriculum for basic education was identical for both female and male students, placing a significant emphasis on natural sciences and mathematics \citep{FuchsSchuendeln2016, Campa2019, Davoli2021}.
This was not the case in the FRG, where school curricula for females and males differed until the 1970s.
Third, the state took steps to encourage females to pursue careers in science by establishing the ``Democratic Women's Association of Germany'' (DFD for short).
This association was founded during the short period of the Soviet occupation zone and constituted one of the biggest mass organisations within the GDR.
The organisation's stated objective was to secure equal rights for females in education and employment.
The SED viewed the DFD as an instrument for mobilising females into the labour force, which was in dire need of additional workers throughout the entire existence of the GDR.
In 1962, the Ministerial Council approved a programme designed to encourage females to work in technical occupations, acknowleding the underrepresentation of females in this sector \citep{GBl1962}.
In the area of school education, the SED set itself the goal of improving the quality of science lessons at the extended secondary schools and increasing the proportion of girls in the science stream, thereby fuelling their interest in science.
Also, a special focus on science and technology in grades 5 to 8 of the ten-grade polytechnic secondary school is intended for the same motivation.
When promoting university studies in STEM fields, females in particular should be targeted.
For certain degree programmes, a quota for females must be met for new enrolments.
Evening courses are to be offered for working females and mothers who are unable to study full-time.
In this context, special ``women's classes'' are to be set up.
Furthermore, the legislation sets out a list of occupations that are required to meet a specified quota for female representation.
This list includes, for instance, electromechanics, telecommunication mechanics, skilled chemical workers, and more.

% further steps
The SED implemented additional measures to facilitate female participation in the workforce, including the provision of childcare facilities, the introduction of a monthly ``household day'' to acknowledge the ``double burden'' of working females, and the legalisation of abortion \citep{Budde1999}.
The issue of female participation in the workforce has always been a practical and ideological concern for the party.
Consequently, the working mother has been a consistent focus of state propaganda:
Female tractor drivers and engineers were depicted in East German propaganda films and novels, discussing their enthusiasm for advancing socialism.

Despite all the lip service paid by the SED, hardly any females reached leading positions in the factories and organisations in which they worked \citep{Ross2017, Frauenreport1990}.

\section{Data and Empirical Approach}
\label{dataemp}

\subsection{SOEP}
\label{gsoep}

% gsoep
In order to examine the developments that have occurred in the STEM sector following the Reunification, this paper employs data from the German Socio-Economic Panel (SOEP for short), a nationally representative panel study initiated in 1984 and maintained by the German Institute for Economic Research (DIW for short) \citep{Siegers2022}.
It frequently serves as a basis for research on the Reunification and East-West differences in Germany across different domains \citep{Petrunyk2016, Bird1994, Hadjar2010}.

% sample for descriptive analysis
In order to provide a descriptive analysis in the first part of this paper, SOEP data from the survey years 1990--1999 is utilised.
This sample comprises individuals of working age (16--65 years old) with information on their employment status, ISCO-88 occupation code, and location in 1989 (the year preceding Reunification) available.
All individuals were either born in the GDR or the FRG.
The International Standard Classification of Occupations (ISCO for short) is an internationally used classification standard for jobs \citep{Elias1997}.
It is developed and maintained by the International Labour Organisation (ILB for short).
The ISCO is designed to provide a structured framework for organising jobs into clearly defined groups based on the tasks and duties involved.
The location in 1989 (either East or West Germany) is used to identify former GDR citizens in the sample.

% sample for epidemiological approach
The estimation sample used for the epidemiological analysis of the transmission of cultural values is derived from the previous sample as follows:
Every individual from the sample used in the descriptive analysis who has one or more children who themselves participate in the SOEP is linked to them.
The estimation sample comprises all adult children who have both parents sucessfully linked to them.
They were born after 1983, i.e. they were 6 years or younger at the time of Reunification, and have valid information on their field of tertiary education.

\subsection{Fernández' Epidemiological Approach}
\label{epid}

In order to analyse the transmission of the cultural emphasis on STEM in the GDR, this paper employs a version of the epidemiological approach formulated by \cite{Fernandez2011}.
Fernández argues that culture can have a significant influence on various economic outcomes, such as savings rates, fertility rates, and FLFP.
She defines the epidemiological approach as an attempt to separate culture from environment and, more precisely, as a method to study different immigrant groups within the same host country. % Special cases Germany where GDR and FRG cititzens are compared
Thus, the outcomes of individuals whose cultures differ but who live in the same economic and institutional setting can be studied.

Fernández brings forward a medical example from the field of epidemiology (hence the name) to illustrate the idea that drives the approach:
Suppose the incidence of heart disease differs significantly between two countries (the source and the host country).
If the incidence of heart disease in immigrants converges to that of natives of the host country, the difference between the two countries is unlikely to be driven by genetics but rather has environmental causes.
Crucially, Fernández points out that failure of convergence does not imply the opposite.

Translating this example into the economic sphere, the objective is to isolate the variation in outcomes that is caused by culture versus the variation that is caused by economic and institutional factors.

Fernández provides a minimal example of how the approach can be executed with a regression model.
For this purpose, suppose that there is data on individuals who live in one given country but whose parents were born in some other country $c$.
Suppose then that we want to estimate the probability that individual $i$ takes some action $y_{ic}$ modelled as

\vspace{-8mm}

\begin{equation}
	\label{eq:fernandezexercise}
	y_{ic} = \beta_{0} + \beta_{1}Y_{c} + \gamma{}X'_{i} + \epsilon_{i}\textnormal{,}
\end{equation}

\noindent where $Y_{c}$ is a proxy for culture in country of ancestry $c$ and $X'_{i}$ is the transpose of a matrix of individual characteristics.
The regressor of interest is the latter, which aims to capture variation in $y_{ic}$ caused by cultural influence.
Culture can be proxied by employing a country-of-ancestry dummy for $Y_{c}$.
However, this approach may give rise to several issues:
First, a dummy captures a broad, undifferentiated effect and does not vary for specific cultural traits that might not be homogenous within a country of ancestry.
Second, the economic and cultural influences of the country of ancestry are collapsed into one dummy.
Therefore, additionally controlling for the macroeconomic properties of the country of ancestry might help to obtain a more nuanced picture instead of an average effect.
Third, the country-of-ancestry dummy does not vary over time.
While a country might have significantly changed in several dimensions since the departure of the parents of the individual, this change cannot be captured in a time-constant variable.

% application on former gdr citizens in reunified germany (simple model)
Before specifying a regression model for the purpose of this paper, it is first  beneficial to discuss the epidemiological approach in further detail, following the description of Fernández but modifying her theoretical model of a female's extensive work decision to fit the subject of this paper.
Suppose the decision of a female living in $k \in \{\textnormal{GDR, FRG}\}$ whether to study a STEM subject or a non-STEM subject at university can be thought of as a maximisation problem of the two-period model

\vspace{-8mm}

\begin{equation}
	\label{eq:utility}
	U = u(c_{0,k}) - \mathds{1}v_{i} + \beta{}u(c_{1,k})\textnormal{,}
\end{equation}

\noindent where $u$ is a strictly increasing and concave function, $\mathds{1}$ is an indicator function which equals one if the female $i$ is enrolled in a STEM subject and zero if she is enrolled in a non-STEM subject, $v_{i}$ is the disutility of studying a STEM subject, $\beta \in (0,1)$ is an exogenously given discount factor of future consumption that does not vary across individuals or countries,

\vspace{-8mm}

\begin{equation*}
	\label{eq:c0k}
	c_{0,k} = w_{0,k}
\end{equation*}

\noindent is consumption while studying at the university, and 

\vspace{-8mm}

\begin{equation*}
	\label{eq:c1k}
	c_{1,k} = w_{1,k} + \mathds{1}(w_{1,k,\textnormal{STEM}} - w_{1,k})
\end{equation*}

\noindent is consumption after receiving the university degree and working.
For simplicity, the incomes $w_{0,k}$ (income as a student in period 0), $w_{1,k}$ (working in non-STEM in period 1), and $w_{1,k,\textnormal{STEM}}$ (working in STEM in period 1) are taken as exogenous and do not vary across individuals.
Further, assume that $w_{0,k} < w_{1,k} < w_{1,k,\textnormal{STEM}}$ holds and all females who study a STEM subject in period 0 will work in STEM in period 1, while all females who study a non-STEM subject in period 0 will work in non-STEM in period 1.

The disutility of studying a STEM subject $v_{i}$ varies across females and is assumed to be a random draw from a country-specific distribution with mean $m_{k}$ and variance $\sigma^{2}$.
$\sigma^{2}$ is assumed to be constant across countries.
The cumulative distribution function (cdf for short) is denoted by $G_{k}(m_{k}, \sigma)$.
Therefore, cultural differences between the GDR and the FRG with respect to females studying STEM subjects are modelled as an exogenous difference in the mean of the distribution that characterises the disutility of studying a STEM subject.
Fernández points out that there are two sources of heterogeneity in the model, namely a within-country heterogeneity given by the fact that females from the same country receive different draws from the same distribution and a cross-country heterogeneity that is reflected in the mean of the distribution from which the females draw their disutility levels.

Given wages in country $k$ and the distribution of disutility $G_{k}$, one can solve for the share of enrolled females who study a STEM subject in that country, $S_{k}$. It is given by the cdf evaluated at $v^{*}_{k}$, i. e.

\vspace{-8mm}

\begin{equation}
	\label{eq:share}
	S_{k} = G_{k}(v^{*}_{k})\textnormal{,}
\end{equation}

\noindent where

\vspace{-8mm}

\begin{equation*}
	\label{eq:indiff}
	v^{*}_{k} \equiv v^{*}(w_{1,k}, w_{1,k,\textnormal{STEM}}) = u(w_{1,k,\textnormal{STEM}}) - u(w_{1,k})
\end{equation*}

\noindent is the level of disutility from studying a STEM subject in country $k$ which makes a female indifferent between studying a STEM subject or studying a non-STEM subject.

Suppose now that the disutility of studying a STEM subject can be characterised by the standard normal cdf $\Phi{}(x)$ evaluated at $x$, then

\vspace{-8mm}

\begin{equation}
	\label{eq:sharecdf}
	S_{k} = \Phi\left(\frac{v^{*}_{k} - m_{k}}{\sigma}\right)\textnormal{.}
\end{equation}

\noindent (\ref{eq:sharecdf}) depends on $v^{*}_{k}$, which itself only depends only on exogenously given wages in country $k$, therefore making it the environmental factor in the model, but also on $m_{k}$, which reflects the cultural stance of country $k$ towards females studying STEM subjects.

Making females studying a STEM subject less culturally favourable, i. e. marginally increasing $m_{k}$, leads ceteris paribus to a decrease of the share of enrolled females who study a STEM subject in that country:

\vspace{-8mm}

\begin{equation*}
	\label{eq:derivmk}
	\frac{\delta{}S_{k}}{\delta{}m_{k}} = -\phi\left(\frac{v^{*}_{k} - m_{k}}{\sigma}\right)\frac{1}{\sigma} < 0\textnormal{,}
\end{equation*}

\noindent where $\phi(x)$ is the probability density function (pdf for short) of the standard normal distribution. Similarly, marginally increasing non-STEM wages $w_{1,k}$ leads ceteris paribus to a decrease as well, i. e.

\vspace{-8mm}

\begin{equation*}
\label{eq:derivw1k}
\frac{\delta{}S_{k}}{\delta{}w_{1,k}} = -u'(w_{1,k})\phi\left(\frac{v^{*}_{k} - m_{k}}{\sigma}\right)\frac{1}{\sigma} < 0\textnormal{,}
\end{equation*}

\noindent whereas marginally increasing STEM-wages $w_{1,k,\textnormal{STEM}}$ leads ceteris paribus to an increase, i. e.

\vspace{-8mm}

\begin{equation*}
	\label{eq:derivw1kstem}
	\frac{\delta{}S_{k}}{\delta{}w_{1,k,\textnormal{STEM}}} = u'(w_{1,k,\textnormal{STEM}})\phi\left(\frac{v^{*}_{k} - m_{k}}{\sigma}\right)\frac{1}{\sigma} > 0\textnormal{.}
\end{equation*}

Finally, consider the situation of females in reunified Germany $j$ with parental roots $k \in \{\textnormal{GDR, FRG}\}$ who are confronted with the same decision regarding whether to study a STEM subject or not. Suppose these females are identical except for their cultural beliefs and culture is transmitted perfectly, i. e. they inherit the same $v_{i}$ that their mothers drew in the Cold War era.
Assume again that $G$ is a normal distribution.
Thus, while $v^{*}$ will be identical for all females in reunified Germany, since this threshold parameter only depends on wages in $j$, the individuals inherit their draws from their mothers who drew from two different distributions, namely $\phi(m_{\textnormal{GDR}},\sigma^{2})$ and $\phi(m_{\textnormal{FRG}},\sigma^{2})$.
The females with ancestry $k$ who study a STEM subject as a share of all enrolled females with ancestry $k$ is given by

\vspace{-8mm}

\begin{equation}
	\label{eq:sharecdfancestry}
	S_{k,j} = \Phi\left(\frac{v^{*}_{j} - m_{k}}{\sigma}\right)\textnormal{.}
\end{equation}

Fernández points out that culture not mattering would require that $m_{GDR} = m_{FRG}$, i. e. there was the same cultural appreciation for females studying STEM subjects both in the GDR and the FRG.

\subsection{Epidemiological Approach in Economic Literature}
\label{epidliterature}

\subsection{Econometric Specification}
\label{specification}

This paper applies the epidemiological approach to measure effects of cultural heritage on two dependent binary variables.
The first dependent variable, University$_{its}$, indicates whether the individual $i$ in federal state $s$ at time $t$ has obtained a university degree as opposed to a vocational degree.
All individuals in the estimation sample must either have obtained a university degree or a vocational degree.
The second dependent variable, STEM$_{its}$, indicates whether the individual has obtained a STEM university degree rather than a non-STEM university degree.
The logit regression equation is given by

\vspace{-8mm}

\begin{equation}
	\begin{split}
		\label{eq:specsplit}
		\textnormal{Education}_{its} &={} \beta_{0} + \beta_{1}\textnormal{MotherEverSTEM}_{i} + \beta_{2}\textnormal{FatherEverSTEM}_{i} \\
		& + \beta_{3}\textnormal{MotherEast}_{i} + \beta_{4}\textnormal{FatherEast}_{i} \\
		& + \beta_{5}(\textnormal{MotherEverSTEM}_{i} \times \textnormal{MotherEast}_{i}) \\
		& + \beta_{5}(\textnormal{FatherEverSTEM}_{i} \times \textnormal{FatherEast}_{i}) \\
		& + \gamma{}X'_{its} + \epsilon_{its}\textnormal{,}
	\end{split}
\end{equation}

\noindent where MotherEverSTEM$_{i}$ and FatherEverSTEM$_{i}$ indicate whether the parents were ever employed in the STEM sector while particpating in the SOEP. % clear weakness: no info on pre-reunification work
MotherEast$_{i}$ and FatherEast$_{i}$ indicate whether the parents are of East German origin.
This is defined as having resided in East Germany as opposed to West Germany in 1989.
$X'_{its}$ is the transpose of a control matrix containing the following individual and federal state level controls:
age, age squared, and indirect migration background as individuals controls and unemployment rate, gross domestic product, population density, commuting balance (commuters in - commuters out), and first semester university students.
$t$ equals the latest survey year where the individual has nonmissing values in all individual level variables that are part of the specification.
Therefore, every individual appears only once in the sample which distincts this approach from a pooled panel approach.
The federal state level variables are fixed to the point in time where the individual was 17 years old.
This decision was made in order to control for the economic environment in which the individual was situated at the time of making the decision regarding the two dependent variables.
The application of the epidemiological approach is twofold with two different dependent variables in this paper because it aims to capture both the extensive and the intensive effect of the GDR's culture on educational decisions.
Note that for all individuals with nonmissing values in STEM$_{its}$, University$_{its} = 1$, that is, the second part of the application is done on a subsample of the first part's sample with University$_{its} = 1$ for all $i$.
In order to investigate the heterogeneity of the effects regarding sex, the regressions are run separately for females and males.

\section{Results}
\label{results}

\subsection{Descriptive Analysis}
\label{descriptives}

The first empirical subsection of this paper aims to provide a descriptive account of the trends in the proportion of professionals working in STEM fields within specific demographic groups in Germany during the initial decade following the Reunification.
Person-level SOEP data from the survey years 1990--1999 is utilised.
The sample drawn from the SOEP contains 424,829 observations from 56,780 distinct individuals.
Restricting the sample to individuals of working age (17--65) who are either full- or part-time employed yields a total of 159,221 observations from 37,225 unique individuals.
Figure \ref{fig:timetrend} shows the development of the share of STEM professionals between 1984--2017.
Four demographic groups that result from combining the two binary characteristics ``Male/Female'' and ``East/West'' in every possible way are formed and their trends depicted.
In order to account for oversampling, the person weights provided by the SOEP are applied when collapsing the data into the respective groups.
Every marker in the figure contains the information of 1,264 individuals on average.
While data points already exist for West Germans from 1984 onwards, East Germans only enter the panel from 1990 onwards.

Looking at 1990 reveals that both females and males of East German origin have approximately the same number of STEM professionals relative to all full- or part-time employed individuals of the same sex and origin.
Within the first decade following the Reunification, the trend of the East German females converges into the trend of the West German females, which oscillates around 5\%.
This convergence is seen to be robust throughout the entirety of the time span observed.
The West German males show a share of approximately 14\% in 1990.
This share increases in a roughly linear fashion over the entire time span observed.
The East German males appear to be following this trend, albeit with a constant gap of around 3\% below their West German peers.
This finding appears to be somewhat at odds with the findings of \cite{Jessen2023} and \cite{FriedmanSokuler2020}, who both find that immigrants from socialist countries influence their peers native to the host country in terms of female labour supply and educational choices.
However, this figure suggests that East Germans follow the trends of their West German counterparts with regard to STEM occupations.
Nevertheless, this finding is purely descriptive and lacks a causal framework that could validate such a suggestion.
The convergence of trends may be attributable to external factors that are not readily discernible from the internally driven decisions of individuals.

Figure \ref{fig:timetrendzoom} focuses on the trends of the females within the first decade following the Reunification.
From 1992 onwards, the 95\% confidence intervals begin to overlap between the two trends, indicating a quick convergence after a mere three years following the fall of the Berlin Wall.
The confidence intervals remain overlapping after 1992.

For a better understanding of the individuals in question, tables \ref{tab:descr_summary_east} and \ref{tab:descr_summary_west} provide summary statistics for East and West Germans in the survey year 1990.
The subsample sizes for the four groups are 1,490 East German females and 1,645 males and 343 West German females and 377 males.

One possible reason for the low subsample size for the West Germans can be that approximately 93\% of these individuals stem from the SOEP's inital sample ``A'' which was drawn in 1984.
By 1990, panel attrition shrank the number of successful person interviews in the samples ``A'' and ``B'' down to approximately 78\% of the number in 1984 \citep{Siegers2022}.
In the year 1990, 9,519 sucessful person interviews were conducted for these samples.
However, a large portion of the interviewed individuals are either not of working age, are unemployed, or lack occupational information.
Applying the aforementioned filters results in a sample comprising 720 West German individuals.

In contrast, the then freshly drawn sample ``C'' for East Germany, based on the GDR-Master-Sample designed by Infratest in cooperation with the Department for Social Research of the Radio of GDR, comprises 4,453 successful person interviews in 1990 \citep{Infratest2011, Siegers2022}.
The sample used in the descriptive analysis of this paper comprises 3,135 individuals in the survey year 1990.
Columns (1) and (2) in both tables show the mean values of the selected characteristics for females and males.
Column (3) presents the differences in means, i. e. (1) - (2).
While both East German females and males show statistically identical shares of STEM professionals of 11\%, the difference in shares between West German females and males is -8 percentage points (pp for short).

The proportion of individuals in STEM occupations among both East German females and males is statistically identical, with a figure of 11\%.
In contrast, there is a notable discrepancy between the sexes in West Germany, with a difference of -8 percentage points (pp for short).
The East German individuals are approximately 8 years older than their West German peers and therefore also have a higher share of individuals with spouses or life partners.
The proportion of West German females in a relationship is, on average, significantly higher than that of West German males (68\% versus 62\%), despite the former being, on average, slightly younger than the latter.
There is no significant difference in the proportion of East German individuals in a relationship between the sexes.

Figures \ref{fig:survivalfemale} and \ref{fig:survivalmale} show the progression of Eastern females and males who reported being occupied in the STEM sector in 1990, in terms of their occupational status over the subsequent decade.
In 1990, 163 female and 184 male STEM professionals are observed in the East.
The two groups are monitored over the subsequent nine survey years, with occupational data presented as either employment in STEM, employment in non-STEM, lack of regular employment, or missing information.

With regard to the female cohort, the share of STEM professionals decreases to approximately 47\% in 1991, while 26\% of the 163 females work in non-STEM, and 21\% are not engaged in regular employment.
The share of STEM professionals in the male cohort decreases to 50\%, while 30\% of the 184 males work in non-STEM, and only 8\% have no regular employment.
By 1998, the share of STEM professionals within the female cohort had decreased to 19\%, while within the male cohort it had decreased to 23\%.
Note that for the years 1994, 1996, and 1999, the share of item- and (partial) unit-nonresponse is particularly high due to a large share of missing information in the ISCO-88 variable for these survey years.
With the exception of these years, the share of individuals employed in non-STEM is consistently higher than the share of individuals without regular employment for both the female and male cohort.
This appears to be reasonable, given that the majority of STEM roles require a tertiary qualification in the relevant field.
Therefore, the individuals in the observed cohorts are highly educated and, as a result, more in demand in the labour market, which makes them less likely to become unemployed.
In fact, the females have an average of 13 years of education and the males have an average of 14 years of education.
Subtracting the ten years spent in secondary school, this leaves them with 3--4 years spent either in an engineering school, a technical college, or a university.

Figure \ref{fig:netswitches} explores the occupational developments for Eastern females in a different manner.
First, the figure considers all Eastern females, whether they are employed in STEM or non-STEM roles, or are not in regular employment at all.
Second, the figure shows the netted out movements from one occupational status to another.
Thus, for instance, a female who is employed in a STEM role and transitions to a non-STEM position in the subsequent year is classified as having made a switch from STEM to non-STEM.
All switches between two survey years are summed up to a total of switches from one status to another.
Switches in the opposite direction are accounted for by substrating these negative switches from the total of the positive switches.
This yields the net switches between two survey years.

The following three pathways are considered: employment (full- or part-time) to no regular employment, STEM to non-STEM, and STEM to no regular employment.
The markers along $y = 0$ represent the sample size for each survey year.
Given that the markers remain approximately constant in size over the observed period, this indicates that the attrition of panel members is offset by the inflow of new participants to the panel over time.

From 1990 to 1991, there is the highest net movement along all three pathways.
The net movement for the pathways ``STEM to non-STEM'' and ``STEM to no regular employment'' decreases in the subsequent years until it equals approximately zero between 1993 and 1994.
The net movement from employment to no regular employment between 1994 and 1995 is negative, indicating that a greater number of Eastern females transitioned from no regular employment to employment than the reverse.

In conclusion, the rate of relative female departure from the STEM sector was the highest in the initial year following the Reunification, after which the rate declined steadily before reaching zero.
The net movement from STEM to no regular employment is consistently below the net movement from STEM to non-STEM, which lends further support to the findings derived from figure \ref{fig:survivalfemale}, namely, that unemployment was less of an issue with these highly trained individuals.

\subsection{Legacy Effect on Educational Choices}
\label{educational}

\section{Discussion}
\label{discussion}

\section{Robustness: Different Model Specifications}
\label{robustness}

\section{Extension: Comparing the Blocs}
\label{Extension}

\section{Conclusion}
\label{conclusion}

% either or
%\newpage
%\vspace{4cm}

\phantomsection
\addcontentsline{toc}{section}{References}

\makeatletter % prevent newpage within bib-item
\interlinepenalty=10000

\bibliography{../references}
\label{references}

\makeatother

\vspace{-.3cm}

\clearpage

}

{\fontsize{11pt}{16.5pt}\selectfont

\phantomsection
\addcontentsline{toc}{section}{Appendix}
\section*{Appendix}
\label{appendix}

\phantomsection
\addcontentsline{toc}{subsection}{Figures}
\subsection*{Figures}
\label{figures}

\begin{figure}[ht]
    \centering
    \caption{Cohorts (Start of Occupation by Gender), 1945--1990}
    \label{fig:mayer}
    \fontsize{9pt}{11pt}\selectfont
	\def\svgwidth{\textwidth}
	\input{validity.pdf_tex}
    \vspace{2mm}
    \parbox{10cm}{
    \linespread{1}\footnotesize Note: The data come from \cite{Mayer1995}. The mean number of individuals represented by each marker is 891, with a standard deviation of 340.}
\end{figure}

\begin{figure}[ht]
	\centering
	\caption{Time Trend (by Region and Gender), 1984--2017}
	\label{fig:timetrend}
	\fontsize{9pt}{11pt}\selectfont
	\def\svgwidth{\textwidth}
	\input{trend.pdf_tex}
	\vspace{2mm}
	\parbox{10cm}{
	\linespread{1}\footnotesize Note: The data come from \cite{SOEP2023}. The mean number of individuals represented by each marker is 1,264, with a standard deviation of 1,116. Person weights provided by the SOEP are applied when collapsing the data into the respective groups. The vertical dashed line marks the year of the Reunification.}
\end{figure}

\begin{figure}[ht]
	\centering
	\caption{Time Trend (by Region, Females only), 1990--1999}
	\label{fig:timetrendzoom}
	\fontsize{9pt}{11pt}\selectfont
	\def\svgwidth{\textwidth}
	\input{trend_zoomed.pdf_tex}
	\vspace{2mm}
	\parbox{10cm}{
	\linespread{1}\footnotesize Note: The data come from \cite{SOEP2023}. The mean number of individuals represented by each marker is 812, with a standard deviation of 535. Person weights provided by the SOEP are applied when collapsing the data into the respective groups. The whiskers show the 95\% confidence intervals.}
\end{figure}

\begin{figure}[ht]
	\centering
	\caption{Development of 1990's Eastern Female STEM Professionals, 1990--1999}
	\label{fig:survivalfemale}
	\fontsize{9pt}{11pt}\selectfont
	\def\svgwidth{\textwidth}
	\input{survival_female.pdf_tex}
	\vspace{2mm}
	\parbox{10cm}{
	\linespread{1}\footnotesize Note: The data come from \cite{SOEP2023}. In 1990, 163 female STEM professionals are observed in the East. The figure shows how this cohort develops in the subsequent decade with regard to occupational status. The share of item- and (partial) unit-nonresponse is particularly high for the years 1994, 1996, and 1999 due to a large share of missing information in the ISCO-88 variable for these survey years.}
\end{figure}

\begin{figure}[ht]
	\centering
	\caption{Development of 1990's Eastern Male STEM Professionals, 1990--1999}
	\label{fig:survivalmale}
	\fontsize{9pt}{11pt}\selectfont
	\def\svgwidth{\textwidth}
	\input{survival_male.pdf_tex}
	\vspace{2mm}
	\parbox{10cm}{
	\linespread{1}\footnotesize Note: The data come from \cite{SOEP2023}. In 1990, 184 male STEM professionals are observed in the East. The figure shows how this cohort develops in the subsequent decade with regard to occupational status. The share of item- and (partial) unit-nonresponse is particularly high for the years 1994, 1996, and 1999 due to a large share of missing information in the ISCO-88 variable for these survey years.}
\end{figure}

\begin{figure}[ht]
	\centering
	\caption{Net Switches among Eastern Females, 1990--1999}
	\label{fig:netswitches}
	\fontsize{9pt}{11pt}\selectfont
	\def\svgwidth{\textwidth}
	\input{net_switches.pdf_tex}
	\vspace{2mm}
	\parbox{10cm}{
		\linespread{1}\footnotesize Note: The data come from \cite{SOEP2023}. All Eastern females in the survey years 1990--1999, whether they are employed in STEM or non-STEM roles, or are not in regular employment at all are considered in this figure. The figure shows the netted out movements from one occupational status to another. The markers along $y = 0$ represent the sample size for each survey year.}
\end{figure}

\begin{figure}[ht]
	\centering
	\caption{Information Criteria from Different Models}
	\label{fig:information}
	\fontsize{9pt}{11pt}\selectfont
	\def\svgwidth{\textwidth}
	\input{information.pdf_tex}
	\vspace{2mm}
	\parbox{10cm}{
		\linespread{1}\footnotesize Note: The data come from \cite{SOEP2023}.}
\end{figure}

\clearpage

\phantomsection
\addcontentsline{toc}{subsection}{Tables}
\subsection*{Tables}
\label{tables}

\begin{table}[ht]
	\caption{Summary Statistics for East Germans in Survey Year 1990}
	\label{tab:descr_summary_east}
	\begin{center}
		\begin{tabular}{l*{3}{c}}
			\toprule
			& (1) & (2) & (1) - (2) \\
			\midrule
			STEM Profession & 0.11  & 0.11  &   0.00     \\
			&   (0.01)  & (0.01) & (0.01) \\
			\addlinespace
			Age         &   38.63   &  39.32  &  -0.69\sym{*}     \\
			&     (0.28) &        (0.28)         &      (0.40) \\
			\addlinespace
			Spouse/Life Partner &  0.82      &  0.84  &    -0.02    \\
			&      (0.01)&          (0.01)&         (0.01) \\
			\addlinespace
			Household Size      &  3.21   &  3.28   &   -0.07\sym{*}       \\
			&          (0.03)&       (0.03)        &      (0.04)\\
			\addlinespace
			Residence in West Germany&  0.00   &   0.00  &  0.00         \\
			&         (0.00) &       (0.00)&  (0.00)\\
			\midrule
			Observations        &  1,490    &    1,645     &      3,135             \\
			\bottomrule
		\end{tabular}
		
		\vspace{2mm}
		
		\parbox{10cm}{
			\linespread{1}\footnotesize Note: \sym{*} \(p<0.10\), \sym{**} \(p<0.05\), \sym{***} \(p<0.01\). The data come from the \cite{SOEP2023}. The sample is restricted to individuals who were born in Germany. Standard errors are given in parentheses. Individuals are considered to be East German if their household head lived in the GDR prior to Reunification in 1990. Columns (1) and (2) show the mean values of the selected characteristics for females and males. Column (3) presents the differences in means obtained by employing two-sample Student's t-tests.}
		
	\end{center}
\end{table}

\begin{table}[ht]
    \caption{Summary Statistics for West Germans in Survey Year 1990}
    \label{tab:descr_summary_west}
    \begin{center}
        \begin{tabular}{l*{3}{c}}
        	\toprule
        	& (1) & (2) & (1) - (2) \\
        	\midrule
        	STEM Profession & 0.06  & 0.14 &  -0.08\sym{***}      \\
        	&   (0.01)  & (0.02) & (0.02) \\
        	\addlinespace
        	Age         & 30.81     &  31.60   &    -0.79     \\
        	&     (0.48) &        (0.49)         &      (0.69) \\
        	\addlinespace
        	Spouse/Life Partner & 0.68       &  0.62   &     0.06\sym{*}      \\
        	&      (0.03)&          (0.03)&         (0.04) \\
        	\addlinespace
        	Household Size      &  2.93   &  2.89   &    0.04       \\
        	&          (0.07)&       (0.07)        &      (0.10)\\
        	\addlinespace
        	Residence in West Germany& 0.96    &  0.97   &    -0.02        \\
        	&         (0.01) &       (0.01)&  (0.01)\\
        	\midrule
        	Observations        &        343 &     377     &       720                \\
        	\bottomrule
        \end{tabular}
        
        \vspace{2mm}
        
        \parbox{10cm}{
        \linespread{1}\footnotesize Note: \sym{*} \(p<0.10\), \sym{**} \(p<0.05\), \sym{***} \(p<0.01\). The data come from the \cite{SOEP2023}. The sample is restricted to individuals who were born in Germany. Standard errors are given in parentheses. Individuals are considered to be West German if their household head lived in the FRG prior to Reunification in 1990. Columns (1) and (2) show the mean values of the selected characteristics for females and males. Column (3) presents the differences in means obtained by employing two-sample Student's t-tests.}
        
    \end{center}
\end{table}

\begin{table}[ht]
	\caption[Summary Statistics for Adult Children]{Summary Statistics for Adult Children of Individuals from Descriptive Analysis' Sample}
	\label{tab:descr_summary_epid}
	\begin{center}
		\begin{tabular}{l*{3}{c}}
			\toprule
			& (1) & (2) & (1) - (2) \\
			\midrule
			STEM Tertiary Education &  0.26 & 0.46 &   -0.21\sym{***}    \\
			&   (0.01)  & (0.01) & (0.02) \\
			\addlinespace
			Age         &   27.96   &  28.26   &   -0.30\sym{**}     \\
			&     (0.09) &        (0.10)         &      (0.13) \\
			\addlinespace
			Spouse/Life Partner &   0.52     &   0.33  &    0.19\sym{***}     \\
			&      (0.01)&          (0.01)&         (0.02) \\
			\addlinespace
			Household Size      &  2.30   &  2.25   &   0.05        \\
			&          (0.03)&       (0.04)   &   (0.05) \\
			\addlinespace
			Residence in West Germany  &  0.76   &  0.71   &     0.04\sym{**}     \\
			&         (0.01) &       (0.01)&  (0.02)\\
			\addlinespace
			Mother: East German Origin &  0.25   & 0.27    &    -0.01      \\
			&         (0.01) &       (0.01)&  (0.01)\\
			\addlinespace
			Father: East German Origin &  0.24   & 0.25    &   -0.01       \\
			&         (0.01) &       (0.01)&  (0.02)\\
			\addlinespace
			Mother: Ever STEM Profession &  0.12   & 0.08    &  0.04\sym{***}        \\
			&         (0.01) &       (0.01) &  (0.01) \\
			\addlinespace
			Father: Ever STEM Profession &  0.31   &  0.37   &   -0.05\sym{***}       \\
			&         (0.01) &       (0.01)&  (0.02)\\
			\addlinespace
			Indirect Migration Background &  0.02   &   0.01  &   0.01\sym{***}       \\
			&         (0.00) &       (0.00)&  (0.01) \\
			\midrule
			Observations        &  1,324     &  1,198     &        2,522            \\
			\bottomrule
		\end{tabular}
		
		\vspace{2mm}
		
		\parbox{10cm}{
			\linespread{1}\footnotesize Note: \sym{*} \(p<0.10\), \sym{**} \(p<0.05\), \sym{***} \(p<0.01\). The data come from the \cite{SOEP2023}. The sample is restricted to individuals who were born in Germany. Standard errors are given in parentheses. Columns (1) and (2) show the mean values of the selected characteristics for females and males. Column (3) presents the differences in means obtained by employing two-sample Student's t-tests.}
		
	\end{center}
\end{table}

\begin{table}[ht]
	\caption[STEM subject at University (Baseline--6)]{Prob. of being enrolled in a STEM subject at University (Baseline--6)}
	\label{tab:baseline--6}
	\begin{center}
		\begin{tabular}{l*{3}{c}}
			\toprule
			&\multicolumn{1}{c}{(1)}         &\multicolumn{1}{c}{(2)}         &\multicolumn{1}{c}{(3)}         \\
			\midrule
			Female              &     -0.1601\sym{***}&     -0.1568\sym{***}&     -0.1554\sym{***}\\
			&    (0.0474)         &    (0.0473)         &    (0.0473)         \\
			\addlinespace
			Mother: Eastern Origin&      0.0820         &      0.0954         &      0.0829         \\
			&    (0.1190)         &    (0.1184)         &    (0.1088)         \\
			\addlinespace
			Female $\times$ Mother: Eastern Origin&     -0.1172         &     -0.1185         &     -0.1220         \\
			&    (0.1010)         &    (0.1010)         &    (0.0987)         \\
			\addlinespace
			Father: Eastern Origin&     -0.0306         &     -0.0476         &      0.0042         \\
			&    (0.1163)         &    (0.1161)         &    (0.1174)         \\
			\addlinespace
			Mother: Ever STEM Profession&     -0.0496         &     -0.0458         &     -0.0555         \\
			&    (0.0754)         &    (0.0751)         &    (0.0745)         \\
			\addlinespace
			Father: Ever STEM Profession&      0.1292\sym{***}&      0.1298\sym{***}&      0.1291\sym{***}\\
			&    (0.0441)         &    (0.0443)         &    (0.0440)         \\
			\midrule
			Rank                &      6         &      9         &     13         \\
			Observations                   &   2,522         &   2,522         &   2,522         \\
			\bottomrule
		\end{tabular}
		
		\vspace{2mm}
		
		\parbox{10cm}{
			\linespread{1}\footnotesize Note: \sym{*} \(p<0.10\), \sym{**} \(p<0.05\), \sym{***} \(p<0.01\). Margins (dy/dx) of a Logistic Regression Model. The data come from the \cite{SOEP2023}. The sample is restricted to individuals who were born in Germany. Standard errors clustered at the household level are given in parentheses.}
		
	\end{center}
\end{table}

\begin{table}[ht]
	\caption[STEM subject at University (Baseline--8)]{Prob. of being enrolled in a STEM subject at University (Baseline--8)}
	\label{tab:baseline--8}
	\begin{center}
		\begin{tabular}{l*{3}{c}}
			\toprule
			&\multicolumn{1}{c}{(1)}         &\multicolumn{1}{c}{(2)}         &\multicolumn{1}{c}{(3)}         \\
			\midrule
			Female              &     -0.1735\sym{***}&     -0.1703\sym{***}&     -0.1690\sym{***}\\
			&    (0.0499)         &    (0.0497)         &    (0.0495)         \\
			\addlinespace
			Mother: Eastern Origin&      0.0774         &      0.0897         &      0.0784         \\
			&    (0.1192)         &    (0.1188)         &    (0.1092)         \\
			\addlinespace
			Female $\times$ Mother: Eastern Origin&     -0.1010         &     -0.1011         &     -0.1081         \\
			&    (0.1073)         &    (0.1073)         &    (0.1050)         \\
			\addlinespace
			Father: Eastern Origin&     -0.0199         &     -0.0360         &      0.0133         \\
			&    (0.1164)         &    (0.1164)         &    (0.1174)         \\
			\addlinespace
			Mother: Ever STEM Profession&     -0.1142         &     -0.1095         &     -0.1248         \\
			&    (0.1023)         &    (0.1020)         &    (0.0963)         \\
			\addlinespace
			Father: Ever STEM Profession&      0.1314\sym{***}&      0.1324\sym{***}&      0.1315\sym{***}\\
			&    (0.0438)         &    (0.0440)         &    (0.0438)         \\
			\addlinespace
			Female $\times$ Mother: Ever STEM Profession&      0.1595         &      0.1595         &      0.1642         \\
			&    (0.1520)         &    (0.1513)         &    (0.1494)         \\
			\addlinespace
			Female $\times$ Mother: Eastern Or. and Ever STEM Prof. &     -0.1693         &     -0.1759         &     -0.1580         \\
			&    (0.2006)         &    (0.1997)         &    (0.2008)         \\
			\midrule
			Rank                &      8         &     11         &     15         \\
			Observations                   &   2,522         &   2,522         &   2,522         \\
			\bottomrule
		\end{tabular}
		
		\vspace{2mm}
		
		\parbox{10cm}{
			\linespread{1}\footnotesize Note: \sym{*} \(p<0.10\), \sym{**} \(p<0.05\), \sym{***} \(p<0.01\). Margins (dy/dx) of a Logistic Regression Model. The data come from the \cite{SOEP2023}. The sample is restricted to individuals who were born in Germany. Standard errors clustered at the household level are given in parentheses.}
		
	\end{center}
\end{table}

\begin{table}[ht]
	\caption[STEM subject at University (Baseline--4)]{Prob. of being enrolled in a STEM subject at University (Baseline--4)}
	\label{tab:baseline--4}
	\begin{center}
		\begin{tabular}{l*{3}{c}}
			\toprule
			&\multicolumn{1}{c}{(1)}         &\multicolumn{1}{c}{(2)}         &\multicolumn{1}{c}{(3)}         \\
			\midrule
			Female              &     -0.1713\sym{***}&     -0.1678\sym{***}&     -0.1663\sym{***}\\
			&    (0.0479)         &    (0.0478)         &    (0.0476)         \\
			\addlinespace
			Mother: Eastern Origin&      0.0998         &      0.1148         &      0.1043         \\
			&    (0.1327)         &    (0.1308)         &    (0.1187)         \\
			\addlinespace
			Female $\times$ Mother: Eastern Origin&     -0.1094         &     -0.1110         &     -0.1140         \\
			&    (0.1036)         &    (0.1035)         &    (0.1012)         \\
			\addlinespace
			Father: Eastern Origin&     -0.0430         &     -0.0609         &     -0.0140         \\
			&    (0.1288)         &    (0.1276)         &    (0.1252)         \\
			\midrule
			Rank                &      4         &     7         &     11         \\
			Observations                   &   2,522         &   2,522         &   2,522         \\
			\bottomrule
		\end{tabular}
		
		\vspace{2mm}
		
		\parbox{10cm}{
			\linespread{1}\footnotesize Note: \sym{*} \(p<0.10\), \sym{**} \(p<0.05\), \sym{***} \(p<0.01\). Margins (dy/dx) of a Logistic Regression Model. The data come from the \cite{SOEP2023}. The sample is restricted to individuals who were born in Germany. Standard errors clustered at the household level are given in parentheses.}
		
	\end{center}
\end{table}

\begin{table}[ht]
	\caption[STEM subject at University (Baseline--6 + Add. Controls)]{Prob. of being enrolled in a STEM subject at University (Baseline--6 $+$ Add. Controls)}
	\label{tab:baseline--6_controls}
	\begin{center}
		\begin{tabular}{l*{3}{c}}
			\toprule
			&\multicolumn{1}{c}{(1)}         &\multicolumn{1}{c}{(2)}         &\multicolumn{1}{c}{(3)}         \\
			\midrule
			Female              &     -0.1753\sym{***}&     -0.1771\sym{***}&     -0.1826\sym{***}\\
			&    (0.0522)         &    (0.0515)         &    (0.0507)         \\
			\addlinespace
			Mother: Eastern Origin&      0.1224         &      0.1134         &      0.1007         \\
			&    (0.1302)         &    (0.1306)         &    (0.1190)         \\
			\addlinespace
			Female $\times$ Mother: Eastern Origin&     -0.1260         &     -0.1077         &     -0.1101         \\
			&    (0.1072)         &    (0.1048)         &    (0.1014)         \\
			\addlinespace
			Father: Eastern Origin&     -0.0439         &     -0.0726         &     -0.0345         \\
			&    (0.1256)         &    (0.1262)         &    (0.1249)         \\
			\addlinespace
			Mother: Ever STEM Profession&     -0.0778         &     -0.0559         &     -0.0622         \\
			&    (0.0871)         &    (0.0854)         &    (0.0842)         \\
			\addlinespace
			Father: Ever STEM Profession&      0.1098\sym{**} &      0.1153\sym{**} &      0.1154\sym{**} \\
			&    (0.0479)         &    (0.0470)         &    (0.0462)         \\
			\midrule
			Rank                &      6         &     11         &     15         \\
			Observations                   &   1,878         &   1,878         &   1,878         \\
			\bottomrule
		\end{tabular}
		
		\vspace{2mm}
		
		\parbox{10cm}{
			\linespread{1}\footnotesize Note: \sym{*} \(p<0.10\), \sym{**} \(p<0.05\), \sym{***} \(p<0.01\). Margins (dy/dx) of a Logistic Regression Model. The data come from the \cite{SOEP2023}. The sample is restricted to individuals who were born in Germany. Standard errors clustered at the household level are given in parentheses.}
		
	\end{center}
\end{table}

\begin{table}[ht]
	\caption[STEM subject at University (East/West Bloc, Baseline--6)]{Prob. of being enrolled in a STEM subject at University (East/West Bloc, Baseline--6)}
	\label{tab:extension}
	\begin{center}
		\begin{tabular}{l*{3}{c}}
			\toprule
			&\multicolumn{1}{c}{(1)}         &\multicolumn{1}{c}{(2)}         &\multicolumn{1}{c}{(3)}         \\
			\midrule
			Female              &     -0.1675\sym{***}&     -0.1673\sym{***}&     -0.1652\sym{***}\\
			&    (0.0468)         &    (0.0469)         &    (0.0467)         \\
			\addlinespace
			Mother: Former Warsaw Pact State&      0.0818         &      0.0871         &      0.0933         \\
			&    (0.1033)         &    (0.1022)         &    (0.0941)         \\
			\addlinespace
			Female $\times$ Mother: Former Warsaw Pact State&     -0.0767         &     -0.0757         &     -0.0844         \\
			&    (0.0948)         &    (0.0949)         &    (0.0929)         \\
			\addlinespace
			Father: Former Warsaw Pact State&     -0.0377         &     -0.0462         &     -0.0085         \\
			&    (0.0966)         &    (0.0960)         &    (0.0933)         \\
			\addlinespace
			Mother: Ever STEM Profession&     -0.0520         &     -0.0481         &     -0.0582         \\
			&    (0.0730)         &    (0.0730)         &    (0.0722)         \\
			\addlinespace
			Father: Ever STEM Profession&      0.1446\sym{***}&      0.1464\sym{***}&      0.1458\sym{***}\\
			&    (0.0425)         &    (0.0428)         &    (0.0425)         \\
			\midrule
			Rank                &      6         &      9         &     13         \\
			Observations                  &   2,624         &   2,624         &   2,624         \\
			\bottomrule
		\end{tabular}
		
		\vspace{2mm}
		
		\parbox{10cm}{
			\linespread{1}\footnotesize Note: \sym{*} \(p<0.10\), \sym{**} \(p<0.05\), \sym{***} \(p<0.01\). Margins (dy/dx) of a Logistic Regression Model. The data come from the \cite{SOEP2023}. The sample is restricted to individuals who were born in Germany. Standard errors clustered at the household level are given in parentheses.}
		
	\end{center}
\end{table}

\clearpage

}
{\fontsize{11pt}{11pt}\selectfont

\phantomsection
\addcontentsline{toc}{section}{Statutory Declaration}
\section*{Statutory Declaration}
\label{declarations}

I hereby declare that I have written this thesis independently and without the use of sources and aids other than those specified. I have not used the services of any agency providing specimen, model, or ghostwritten work in the preparation of this submitted work. This also includes the use of AI-generated texts or services such as ChatGPT. Sentences or parts of sentences quoted literally are marked as such; other references with regard to the statement and scope are indicated by full details of the publications concerned. The thesis in the same or similar form has not been submitted to any examination body and has not been published. This thesis was not yet, even in part, used in another examination or as a course performance.

\vspace{1.5cm}

\includegraphics[height = 25mm]{unterschrift_schuett.pdf}

\noindent Potsdam, \today{}

} % close \fontsize{11}{16.5}

\end{document}
