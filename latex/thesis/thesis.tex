\documentclass[a4paper, oneside, hyperfootnotes = false]{article}

\usepackage[utf8]{inputenc}
\usepackage[T1]{fontenc}
\usepackage[UKenglish]{babel}
\usepackage{amsmath}
\usepackage{anyfontsize}
\usepackage{bookmark}
\usepackage{booktabs}
\usepackage{caption}
\usepackage{dsfont}
\usepackage{fancyvrb}
\usepackage[left = 3cm, right = 2.5cm, top = 2.5cm, bottom = 2.5cm]{geometry}
\usepackage{graphicx}
\usepackage{hyperref}
\usepackage{multirow}
\usepackage[round]{natbib}
\usepackage{pdflscape}
\usepackage{svg}
\usepackage{tocloft}

\bibliographystyle{apalike}

% suppress page group warning
\pdfsuppresswarningpagegroup=1

% line spread
\linespread{1.5}

% remove ugly boxes around hyperlinks
\hypersetup{
	pdfborder = {0 0 0}
}

% vertical space between rule and main text
\addtolength{\skip\footins}{8pt}
% vertical space between footnotes
\addtolength{\footnotesep}{5pt}

% path to figures
\graphicspath{{../../stata/figures/pdf_tex/}}

\makeatletter
\def\input@path{{../../stata/figures/pdf_tex/}}
\makeatother

% for tables
\def\sym#1{\ifmmode^{#1}\else\(^{#1}\)\fi}


% title page ------
\title{Putting the GDR's Legacy Effect under the Microscope: \linebreak Eastern Female STEM Professionals in Reunified Germany.}

\author{\href{mailto:johannes.schuett@fu-berlin.de}{Schütt, Johannes} \\
5574549}

\date{\today{}}
% title page end ---

\begin{document}

{\fontsize{12pt}{18pt}\selectfont

\maketitle

\thispagestyle{empty}
\pagenumbering{Alph}

\vspace{1.5cm}

\noindent\begin{tabular}{l}
	A Master Thesis presented to \\
	The Faculty of Economics at the Freie Universität Berlin \\
    [\normalbaselineskip]
    In Partial Fulfillment of Requirements for \\
    The Degree of Master of Science in Public Economics\\
    [\normalbaselineskip]
    Module Examination according to 0353cE1.1P at \\
    The Chair of Empirical Economics and Gender under \\
    The Supervision of Prof. Natalia Danzer, Ph.D. \\
    [\normalbaselineskip]
    Summer Term 2024
\end{tabular}

\vspace{3cm}

\begin{center}
    \includegraphics[width=0.4\textwidth, angle=0]{fu_logo.pdf}
\end{center}

\newpage

\pagestyle{plain}
\pagenumbering{roman}

{\fontsize{12pt}{15pt}\selectfont

\tableofcontents
\newpage

\phantomsection
\addcontentsline{toc}{section}{List of Figures}
\listoffigures

\vspace{2cm}

\phantomsection
\addcontentsline{toc}{section}{List of Tables}
\listoftables

\newpage

\phantomsection
\addcontentsline{toc}{section}{List of Abbreviations}
\section*{List of Abbreviations}
\noindent\begin{tabular}{@{}ll}
	cdf & cumulative distribution function \\
	c. p. & ceteris paribus \\
	DFD & \emph{Demokratischer Frauenbund Deutschlands} \\
    DIW & German Institute for Economic Research \\
    FLFP & Female Labour Force Participation \\
    FRG & Federal Republic of Germany \\
    GDR & German Democratic Republic \\
    SOEP & German Socio-Economic Panel \\
    ISCO & International Standard Classification of Occupations \\
    pdf & probability density function \\
    PISA & Programme for International Student Assessment \\
    pp & percentage points \\
    SED & \emph{Sozialistische Einheitspartei Deutschlands} \\
    STEM & Science, Technology, Engineering, and Mathematics
\end{tabular}

\newpage

\pagenumbering{arabic}
\renewcommand\thesection{\arabic{section}}
\begin{center}
{\large\bfseries Abstract}
\phantomsection
\addcontentsline{toc}{section}{Abstract}

\vspace{5mm}

\parbox{400pt}{
    \emph{\noindent This paper investigates the persistence of social norms regarding occupational choices. To this end, the German Reunification is considered as a unique situation where the two German populations collide in one state after 40 years of separation. First, the question on how Eastern female STEM professionals developed in the sector of reunified Germany is analysed descriptively. Second, an epidemiological approach is employed to investigate whether individuals who were socialised in the GDR transmitted the socialist emphasis on both gender equality and science and technology  to their offspring after the socialist apparatus collapsed and thus no longer actively communicated these norms. This paper finds that females from the East swiftly left the STEM sector following Reunification. Whether this was of their own volition or due to external pressure cannot be shown. Further, the paper finds no evidence that mothers from the former GDR who also have a STEM background exert a positive influence on their daughters to follow a STEM career path. However, it does find that fathers from the East who also have a STEM background have a positive influence on their daughters in this respect.
    }
}

\end{center}

}% end \fontsize{12pt}{15pt}\selectfont

\newpage

\section{Introduction}
\label{intro}

Occupational choices are found to be influenced by a mixture of intrinsic and extrinsic factors \citep{Brennan2017, Morales2024}.
These may be expectations an individual holds for themselves, but also what the individual estimates to be regarded as the optimal choice by the wider social system the individual finds themselves in.
The wider social system includes family and peers, and it also includes institutions.
This paper examines what happens when state institutions that are highly ideologically active and interested in shaping the normative values of their citizens disappear, and whether these induced values persist in individuals.

To this end, the paper considers the case of Reunified Germany and its ``new'' citizens from the former German Democratic Republic (GDR for short). The GDR constituted a highly socialist propaganda driven state.
The focus of this paper is on the state's emphasis on both technological progress in the service of socialism and gender equality, although the latter was never fully achieved throughout the existence of the state.
Nevertheless, these two beliefs led to an equal representation of females and males relative to total labour forces of the two sexes in the technical and scientific sector in the GDR by the 1970s.

As a consequence of the Reunification, East Germans were thrown into the system of the former political enemy without any safeguards.
In Germany today, females are concentrated in a narrower range of occupations than males, often in lower-paid sectors, which contributes to the gender pay gap \citep{Kleinert2023, OECD2022}.

There exists a body of literature on the occupational and educational choices of Warsaw Pact emigrants:
\cite{Jessen2023} find that an above-median inflow of East German migrants to West German regions led to an increase in labour supply of employed West German females in these regions.
Therefore, they find that not only the East Germans preserve their values of female employment, but also transmit them to their Western counterparts.
Similarly, \cite{FriedmanSokuler2020} find that female jewish immigrants from the former Soviet Union to Israel in the early 1990s tend to avoid work in the educational and social sector and instead opt for the so-called STEM field.
STEM is a field and curriculum centred on the topics in the disciplines of Science, Technology, Engineering, and Maths \citep{Hallinen2024}.
The term is of U.S. origin and, although it is a fairly broad umbrella term, it refers mainly to highly academic professions.
\cite{FriedmanSokuler2020} report that even females who immigrated as infants are more likely to pursue STEM subjects in education than native Israeli females of the same age.
Further, they find that native Israeli females who attend schools with a high concentration of FSU immigrants shift their educational choices towards STEM.
Thus, the socialist emphasis on STEM is not only vertically preserved, i. e. between generations of members of the cultural circle, but also horizontally dispersed to non-members.
The concept of vertical preservation and horizontal diffusion of cultural norms is coined by \cite{Boyd1985} as a Darwinian theory.

This paper investigates the question of the preservation of the socialist emphasis on STEM in relation to gender equality in a twofold manner:
First, the question on how Eastern female STEM professionals developed in the sector of reunified Germany is analysed descriptively.
Second, an epidemiological approach is employed to investigate whether individuals who were socialised in the GDR passed on the social norms in question to their offspring after the socialist apparatus collapsed and thus no longer actively communicated these norms.
More generally, this paper seeks to show the effect on cultural transmission when the institutions that disseminate the cultural norms in question disappear.

This paper finds no evidence that mothers from the former GDR who also have a STEM background exert a positive influence on their daughters to follow a STEM career path.
However, it does find that fathers from the East who also have a STEM background have a positive influence on their daughters in this respect.
The results do not change significantly when the definition of STEM is adjusted to better fit the socialist understanding of such a term, i.e. to include blue-collar technical occupations.

The remainder of this paper is organised as follows: Section \ref{background} gives a brief historical overview of the two German nation states and the socialist party's approach to gender equality in the GDR. Section \ref{dataemp} covers the data employed in the paper and the empirical approach. Section \ref{results} presents the main findings, which are then discussed in section \ref{discussion}. Section \ref{robustness} presents a sensitivity analysis using an alternative specification of the regression model. Section \ref{extension} presents an extension where the definition of STEM is adapted to the socialist understanding of such a term. Section \ref{conclusion} concludes.

\section{Historical Background}
\label{background}

% hour zero
During the Cold War, Germany was divided into two nation states: the socialist German Democratic Republic, which together with the other Warsaw Pact states was allied with the Soviet Union, and the Federal Republic of Germany, which together with the other NATO states was allied with the United States of America.
Since their foundation in 1949, the two countries had grown considerably apart in political and economic terms:
While the GDR operated according to a command economy based on the principles of Marxism-Leninism, the FRG employed a social market economy.
In the first year of the existence of these two states, female labour force participation (FLFP for short) was 40\% \citep[Chapter~2]{Menschik1974} for both.
However, this high share of females in the labour force can be explained by the large number of males of working age who were absent due to the war and its aftermath, and even when the husband returned to his family, one income was not enough to support them.
Following the \emph{Wirtschaftswunder} of the 1950s, FLFP declined in the FRG, while females in the GDR never returned in significant numbers to the conservative role of a housewife \citep{Ostner1991, Rosenfeld2004}.

% ulbricht's new economic system
In 1963, Walter Ulbricht, the First Secretary of the Socialist Unity Party (SED for short), sought to reform the struggling East German economic regime, which was essentially the Stalinist model that he instated in the first place. He attempted to introduce limited free-market elements, such as a more flexible price system and a certain level of autonomy granted to factory managers \citep[Chapter~4]{Grieder1999}.
This policy shift included the ideological goal of technocratising the general labour force:
At the third party conference in 1956, Ulbricht posited that workers must ``[...] acquire the most progressive insights offered by technology and science in order to increase their labour productivity [...]'' \citep{Ulbricht1956, Sanderson1981}.

% technocratisation
The GDR -- as a ``Worker and Peasant State'' -- had a labour force that largely consisted of blue-collar workers \citep{DDRJahrbuch1957}.
Therefore, every labourer should understand the purpose of every single cog in the machine they operate.
Beyond that, the SED stimulated the scientific and technological sectors in hopes of innovation that could directly translate into higher production power \citep{Hoegselius2009}.
The overarching ambition was to nurture the most pioneering scientific community in the world, which stood in direct service of socialism.
This ambition led to significant movement in these work sectors in the following decade.
Figure \ref{fig:mayer} uses data from the East German Life History Study to approximate how the share of STEM professionals developed over the duration of the GDR's existence.
For the purpose of the East German Life History Study, 2,331 East German individuals were interviewed retrospectively between September 1991 and October 1992 about their lives in the GDR.
The data cover four birth cohorts (1929--1931, 1939--1941, 1951--1953, and 1959--1961) and provide information on the family of origin, housing and household history, qualifications and educational phases, occupational histories, and further areas of the personal life.

\begin{figure}[ht]
	\centering
	\caption{Cohorts (Start of Occupation by Gender), 1945--1990}
	\label{fig:mayer}
	\fontsize{9pt}{11pt}\selectfont
	\def\svgwidth{.9\textwidth}
	\input{validity.pdf_tex}
	\vspace{2mm}
	\parbox{10cm}{
		\linespread{1}\footnotesize Note: Own calculations based on \cite{Mayer1995}. The mean number of individuals represented by each marker is 891, with a standard deviation of 340. The figure displays individuals who took up a STEM occupation as a share of all individuals who took up a new occupation in the respective decade. The whiskers show the 95\% confidence intervals.}
\end{figure}

The figure displays individuals who took up a STEM occupation as a share of all individuals who took up a new occupation in the respective decade.
As anticipated, the proportion of individuals newly employed in STEM occupations was relatively low prior to the 1960s, as the majority of workers took up occupations that served the purpose of rebuilding the critical infrastructure that had been destroyed during the war.
However, possibly as a result of the ideological shift regarding the sciences in the 1960s, there has been an increase in the share of new male STEM workers, while females stay at a lower share.
Over the last two decades of the existence of the GDR, the male and female shares have converged at approximately 10\%.

% dfd
A number of factors may be identified as contributing to the convergence observed in these shares.
First, the socialist ideology included the concept of removing traditional gender associations from specific occupations.
The law on maternal and child protection and women's rights of 1950 states that ``[...] women should be given more opportunities to work in industry, [...] and public estates, in all organs of state administration, [...] and other institutions of the German Democratic Republic.
The work of women in production should not be limited to the traditional female professions, but should extend to all branches of production, especially the electrical industry, optics, mechanical engineering, precision mechanics, the wood and furniture industry, the shoe industry, and the construction and graphic arts trades.'' \citep[§19.1]{GBl1950}.
This law created the necessity for the state to implement measures to actually facilitate the inclusion of females in technical professions.
It was recognised that there was a disparity in the starting points of females and males in this regard, given that historically, these roles have been predominantly occupied by males.
Second, the school curriculum for basic education was identical for both female and male students, placing a significant emphasis on natural sciences and mathematics \citep{FuchsSchuendeln2016, Campa2019, Davoli2021}.
This was not the case in the FRG, where school curricula for females and males differed until the 1970s.
Third, the state took steps to encourage females to pursue careers in science by establishing the ``Democratic Women's Association of Germany'' (DFD for short).
This association was founded during the short period of the Soviet occupation zone and constituted one of the biggest mass organisations within the GDR.
The organisation's stated objective was to secure equal rights for females in education and employment.
The SED viewed the DFD as an instrument for mobilising females into the labour force, which was in dire need of additional workers throughout the entire existence of the GDR.
In 1962, the Ministerial Council approved a programme designed to encourage females to work in technical occupations, acknowleding the underrepresentation of females in this sector \citep{GBl1962}.
In the area of school education, the SED set itself the goal of improving the quality of science lessons at the extended secondary schools and increasing the proportion of girls in the science stream, thereby fuelling their interest in science.
Also, a special focus on science and technology in grades 5 through 8 of the ten-grade polytechnic secondary school is intended for the same motivation.
When promoting university studies in STEM fields, females in particular should be targeted.
For certain degree programmes, a quota for females must be met for new enrolments.
Evening courses are to be offered for working females and mothers who are unable to study full-time.
In this context, special ``women's classes'' are to be set up.
Furthermore, the legislation sets out a list of occupations that are required to meet a specified quota for female representation.
This list includes, for instance, electromechanics, telecommunication mechanics, skilled chemical workers, and more.

% further steps
The SED implemented additional measures to facilitate female participation in the workforce, including the provision of childcare facilities, the introduction of a monthly ``household day'' to acknowledge the ``double burden'' of working females, and the legalisation of abortion \citep{Budde1999}.
The issue of female participation in the workforce has always been a practical and ideological concern for the party.
Consequently, the working mother has been a consistent focus of state propaganda:
Female tractor drivers and engineers were depicted in East German propaganda films and novels, discussing their enthusiasm for advancing socialism.

Despite all the lip service paid by the SED, hardly any females reached leading positions in the factories and organisations in which they worked \citep{Ross2017, Frauenreport1990}.

\section{Data and Empirical Approach}
\label{dataemp}

Section \ref{gsoep} describes the primary data source utilised in this paper.
Section \ref{epid} provides a detailed account of the theoretical framework that is applied in the later part of the analysis.
Section \ref{epidliterature} gives a non-exhaustive overview of the literature that employs the epidemiological approach.
Lastly, section \ref{specification} presents the specification of the main regression model used in the analysis.

\subsection{SOEP}
\label{gsoep}

% gsoep
In order to examine the developments that have occurred in the STEM sector following the Reunification, this paper employs data from the German Socio-Economic Panel (SOEP for short), a nationally representative panel study initiated in 1984 and maintained by the German Institute for Economic Research (DIW for short) \citep{Siegers2022}.
It frequently serves as a basis for research on the Reunification and East-West differences in Germany across different domains \citep{Petrunyk2016, Bird1994, Hadjar2010}.

% sample for descriptive analysis
In order to provide a descriptive analysis in the first part of this paper, SOEP data from the survey years 1990--1999 is utilised.
This sample comprises individuals of working age (16--65 years old) with information on their employment status, ISCO-88 occupation code, and location in 1989 (the year preceding Reunification) available.
All individuals were either born in the GDR or the FRG.
The International Standard Classification of Occupations (ISCO for short) is an internationally used classification standard for jobs \citep{Elias1997}.
It is developed and maintained by the International Labour Organisation (ILB for short).
The ISCO is designed to provide a structured framework for organising jobs into clearly defined groups based on the tasks and duties involved.
The location in 1989 (either East or West Germany) is used to identify former GDR citizens in the sample.

% sample for epidemiological approach
The estimation sample used for the epidemiological analysis of the transmission of cultural values is derived from the previous sample as follows:
Every individual from the sample used in the descriptive analysis who has one or more children who themselves participate in the SOEP is linked to them.
The estimation sample comprises all adult children who have both parents sucessfully linked to them.
They were born after 1983, i.e. they were 6 years or younger at the time of Reunification, and have valid information on their field of tertiary education.

\subsection{Fernández' Epidemiological Approach}
\label{epid}

In order to analyse the transmission of the cultural emphasis on STEM in the GDR, this paper employs a version of the epidemiological approach formulated by \cite{Fernandez2011}.
Fernández argues that culture can have a significant influence on various economic outcomes, such as savings rates, fertility rates, and FLFP.
She defines the epidemiological approach as an attempt to separate culture from environment and, more precisely, as a method to study different immigrant groups within the same host country. % Special cases Germany where GDR and FRG cititzens are compared
Thus, the outcomes of individuals whose cultures differ but who live in the same economic and institutional setting can be studied.

Fernández brings forward a medical example from the field of epidemiology (hence the name) to illustrate the idea that drives the approach:
Suppose the incidence of heart disease differs significantly between two countries (the source and the host country).
If the incidence of heart disease in immigrants converges to that of natives of the host country, the difference between the two countries is unlikely to be driven by genetics but rather has environmental causes.
Crucially, Fernández points out that failure of convergence does not imply the opposite.

Translating this example into the economic sphere, the objective is to isolate the variation in outcomes that is caused by culture versus the variation that is caused by economic and institutional factors.

Fernández provides a minimal example of how the approach can be executed with a regression model.
For this purpose, suppose that there is data on individuals who live in one given country but whose parents were born in some other country $c$.
Suppose then that we want to estimate the probability that individual $i$ takes some action $y_{ic}$ modelled as

\vspace{-8mm}

\begin{equation}
	\label{eq:fernandezexercise}
	y_{ic} = \beta_{0} + \beta_{1}Y_{c} + \gamma{}X'_{i} + \epsilon_{i}\textnormal{,}
\end{equation}

\noindent where $Y_{c}$ is a proxy for culture in country of ancestry $c$ and $X'_{i}$ is the transpose of a matrix of individual characteristics.
The regressor of interest is the former, which aims to capture variation in $y_{ic}$ caused by cultural influence.
Culture can be proxied by employing a country-of-ancestry dummy for $Y_{c}$.
However, this approach may give rise to several issues:
First, a dummy captures a broad, undifferentiated effect and does not vary for specific cultural traits that might not be homogenous within a country of ancestry.
Second, the economic and cultural influences of the country of ancestry are collapsed into one dummy.
Therefore, additionally controlling for the macroeconomic properties of the country of ancestry might help to obtain a more nuanced picture instead of an average effect.
Third, the country-of-ancestry dummy does not vary over time.
While a country might have significantly changed in several dimensions since the departure of the parents of the individual, this change cannot be captured in a time-constant variable.

% application on former gdr citizens in reunified germany (simple model)
Before specifying a regression model for the purpose of this paper, it is first  beneficial to discuss the epidemiological approach in further detail, following the description of Fernández but modifying her theoretical model of a female's extensive work decision to fit the subject of this paper.
Suppose the decision of a female living in $k \in \{\textnormal{GDR, FRG}\}$ whether to obtain a STEM degree or a non-STEM degree at university can be thought of as a maximisation problem of the two-period model

\vspace{-8mm}

\begin{equation}
	\label{eq:utility}
	U = u(c_{0,k}) - \mathds{1}v_{i} + \beta{}u(c_{1,k})\textnormal{,}
\end{equation}

\noindent where $u$ is a strictly increasing and concave function, $\mathds{1}$ is an indicator function which equals one if the female $i$ obtained a STEM degree and zero if she obtained a non-STEM degree, $v_{i}$ is the disutility of obtaining a STEM degree, $\beta \in (0,1)$ is an exogenously given discount factor of future consumption that does not vary across individuals or countries,

\vspace{-8mm}

\begin{equation*}
	\label{eq:c0k}
	c_{0,k} = w_{0,k}
\end{equation*}

\noindent is consumption while studying at the university, and 

\vspace{-8mm}

\begin{equation*}
	\label{eq:c1k}
	c_{1,k} = w_{1,k} + \mathds{1}(w_{1,k,\textnormal{STEM}} - w_{1,k})
\end{equation*}

\noindent is consumption after receiving the university degree and working.
For simplicity, the incomes $w_{0,k}$ (income as a student in period 0), $w_{1,k}$ (working in non-STEM in period 1), and $w_{1,k,\textnormal{STEM}}$ (working in STEM in period 1) are taken as exogenous and do not vary across individuals.
Further, assume that $w_{0,k} < w_{1,k} < w_{1,k,\textnormal{STEM}}$ holds and all females who study a STEM subject in period 0 will work in STEM in period 1, while all females who study a non-STEM subject in period 0 will work in non-STEM in period 1.

The disutility of obtaining a STEM degree $v_{i}$ varies across females and is assumed to be a random draw from a country-specific distribution with mean $m_{k}$ and variance $\sigma^{2}$.
$\sigma^{2}$ is assumed to be constant across countries.
The cumulative distribution function (cdf for short) is denoted by $G_{k}(m_{k}, \sigma)$.
Therefore, cultural differences between the GDR and the FRG with respect to females obtaining STEM degrees are modelled as an exogenous difference in the mean of the distribution that characterises the disutility of obtaining a STEM degree.
Fernández points out that there are two sources of heterogeneity in the model, namely a within-country heterogeneity given by the fact that females from the same country receive different draws from the same distribution and a cross-country heterogeneity that is reflected in the mean of the distribution from which the females draw their disutility levels.

Given wages in country $k$ and the distribution of disutility $G_{k}$, one can solve for the number of females with a STEM degree as a share of females with a university degree in that country, $S_{k}$. It is given by the cdf evaluated at $v^{*}_{k}$, i. e.

\vspace{-8mm}

\begin{equation}
	\label{eq:share}
	S_{k} = G_{k}(v^{*}_{k})\textnormal{,}
\end{equation}

\noindent where

\vspace{-8mm}

\begin{equation*}
	\label{eq:indiff}
	v^{*}_{k} \equiv v^{*}(w_{1,k}, w_{1,k,\textnormal{STEM}}) = u(w_{1,k,\textnormal{STEM}}) - u(w_{1,k})
\end{equation*}

\noindent is the level of disutility from obtaining a STEM degree in country $k$ which makes a female indifferent between obtaining a STEM degree or obtaining a non-STEM degree.

Suppose now that the disutility of obtaining a STEM degree can be characterised by the standard normal cdf $\Phi{}(x)$ evaluated at $x$, then

\vspace{-8mm}

\begin{equation}
	\label{eq:sharecdf}
	S_{k} = \Phi\left(\frac{v^{*}_{k} - m_{k}}{\sigma}\right)\textnormal{.}
\end{equation}

\noindent Equation (\ref{eq:sharecdf}) depends on $v^{*}_{k}$, which itself depends only on exogenously given wages in country $k$, therefore making it the environmental factor in the model, but also on $m_{k}$, which reflects the cultural stance of country $k$ towards females obtaining a STEM degree.

Making females having a STEM degree less culturally favourable, i. e. marginally increasing $m_{k}$, leads ceteris paribus (c. p. for short) to a decrease of the number of females with a STEM degree as a share of females with a university degree in that country:

\vspace{-8mm}

\begin{equation*}
	\label{eq:derivmk}
	\frac{\delta{}S_{k}}{\delta{}m_{k}} = -\phi\left(\frac{v^{*}_{k} - m_{k}}{\sigma}\right)\frac{1}{\sigma} < 0\textnormal{,}
\end{equation*}

\noindent where $\phi(x)$ is the probability density function (pdf for short) of the standard normal distribution. Similarly, marginally increasing non-STEM wages $w_{1,k}$ leads c. p. to a decrease as well, i. e.

\vspace{-8mm}

\begin{equation*}
\label{eq:derivw1k}
\frac{\delta{}S_{k}}{\delta{}w_{1,k}} = -u'(w_{1,k})\phi\left(\frac{v^{*}_{k} - m_{k}}{\sigma}\right)\frac{1}{\sigma} < 0\textnormal{,}
\end{equation*}

\noindent whereas marginally increasing STEM-wages $w_{1,k,\textnormal{STEM}}$ leads c. p. to an increase, i. e.

\vspace{-8mm}

\begin{equation*}
	\label{eq:derivw1kstem}
	\frac{\delta{}S_{k}}{\delta{}w_{1,k,\textnormal{STEM}}} = u'(w_{1,k,\textnormal{STEM}})\phi\left(\frac{v^{*}_{k} - m_{k}}{\sigma}\right)\frac{1}{\sigma} > 0\textnormal{.}
\end{equation*}

Finally, consider the situation of females in reunified Germany $j$ with parental roots $k \in \{\textnormal{GDR, FRG}\}$ who are confronted with the same decision regarding whether to obtain a STEM degree or a non-STEM degree. Suppose these females are identical except for their cultural beliefs and culture is transmitted perfectly, i. e. they inherit the same $v_{i}$ that their mothers drew in the Cold War era.
Assume again that $G$ is a normal distribution.
Thus, while $v^{*}$ will be identical for all females in reunified Germany, since this threshold parameter only depends on wages in $j$, the individuals inherit their draws from their mothers who drew from two different distributions, namely $\phi(m_{\textnormal{GDR}},\sigma^{2})$ and $\phi(m_{\textnormal{FRG}},\sigma^{2})$.
The females with ancestry $k$ who obtain a STEM degree as a share of all females with a degree with ancestry $k$ is given by

\vspace{-8mm}

\begin{equation}
	\label{eq:sharecdfancestry}
	S_{k,j} = \Phi\left(\frac{v^{*}_{j} - m_{k}}{\sigma}\right)\textnormal{.}
\end{equation}

Fernández points out that culture not mattering would require that $m_{GDR} = m_{FRG}$, i. e. there was the same cultural appreciation for females with a STEM degree both in the GDR and the FRG.

\subsection{Epidemiological Approach in Economic Literature}
\label{epidliterature}

This section provides a non-exhaustive overview of the literature that employs the epidemiological approach.

The first study to utilise the approach is by \cite{Carroll1999}.
They aim to determine whether savings rates of households differ due to cultural rather than economic factors.
By analysing household data from the Canadian Survey of Family Expenditures, they find that savings rates differ between immigrant groups from Canada.
However, they are unable to link this variation back to cultural differences between these groups:
Somewhat counterintuitively, immigrants from countries characterised by a high savings rate, such as South Korea and Japan, exhibit a lower savings rate than immigrants from countries characterised by a low savings rate, such as Italy and Portugal.
This outcome suggests that economic and cultural factors have not been sufficiently disentangled.

In a more recent discussion paper, \cite{Ek2022} investigate the cultural influence on portfolio composition of individual investors.
Specifically, they combine administrative data on the investment outcomes of second-generation immigrants in Sweden with data on their country of ancestry from the Global Preference Survey.
The survey data include information on risk taking and patience, which the authors use to approximate the cultural norms associated with these variables in each country of ancestry.
They find that individuals with ancestral roots in cultures that embrace risk-taking tend to allocate a larger proportion of their wealth to stocks and a smaller portion to mutual funds.
Conversely, individuals with risk-averse countries of ancestry tend to invest a greater share of their wealth in mutual funds.
The authors control for individual and parental socio-economic characteristics in order to filter out the pure cultural influence on the outcomes.

Turning towards the literature on the FLFP, \cite{Antecol2000} investigates cross-country differences in the gender gap in labour force participation rates.
To this end, she employs U.S. Census data from 1990 for first-generation immigrants and data on labour force participation in the countries of ancestry from the ILO Yearbook of Labour Statistics.
Her findings indicate that over half of the overall variation in the gender gap across immigrant groups in the USA can be attributed to the country of ancestry's labour force participation rates.
This implies that cultural perceptions regarding family structure and the employment of females have a significant impact on the size of the gender gap within these immigrant groups.
Further, the influence of country-of-ancestry labour participation rates is smaller for second and higher generation immigrants, indicating cultural assimilation over time.

Finally, \cite{Fernandez2009} investigate the effect that culture has on FLFP and fertility rates.
The study concentrates on second-generation female immigrants in the USA.
The authors use the 1 Percent 1970 Form 2 Metro Sample of the U.S. Census, combining it with past values of FLFP and fertility rates for the country of ancestry sourced from the International Labor Organization and the United Nations Demographic Yearbook, respectively.
These values are used as cultural proxies.
The study reveals that females with countries of ancestry with a low FLFP work fewer hours themselves.
Further, females with countries of ancestry with a high fertility rate tend to have more children.
Furthermore, the authors demonstrate that the influence of culture is amplified for ethnic groups who are more likely to cluster in a neighborhood.
When controlling for the husband's country of ancestry, they find that both countries of ancestry of the wife and husband (if not the same) impact her labour supply, but that the husband's ancestry has a larger impact than the wife's.

\subsection{Econometric Specification}
\label{specification}

This paper applies the epidemiological approach to measure effects of cultural heritage on two dependent binary variables.
The first dependent variable, University$_{itus}$, indicates whether the individual $i$ at time $t$ who was 17 years old at time $u$ and residing in federal state $s$ has obtained a university degree as opposed to a vocational degree.
All individuals in the estimation sample must either have obtained a university degree or a vocational degree.
The second dependent variable, STEM$_{itus}$, indicates whether the individual has obtained a STEM university degree rather than a non-STEM university degree.
The logit regression equation is given by

\vspace{-8mm}

\begin{equation}
	\begin{split}
		\label{eq:specification}
		\textnormal{Education}_{itus} &={} \beta_{0} + \beta_{1}\textnormal{MotherEverSTEM}_{i} + \beta_{2}\textnormal{FatherEverSTEM}_{i} \\
		& + \beta_{3}\textnormal{MotherEast}_{i} + \beta_{4}\textnormal{FatherEast}_{i} \\
		& + \beta_{5}(\textnormal{MotherEverSTEM}_{i} \times \textnormal{MotherEast}_{i}) \\
		& + \beta_{5}(\textnormal{FatherEverSTEM}_{i} \times \textnormal{FatherEast}_{i}) \\
		& + \gamma{}X'_{itus} + \epsilon_{itus}\textnormal{,}
	\end{split}
\end{equation}

\noindent where MotherEverSTEM$_{i}$ and FatherEverSTEM$_{i}$ indicate whether the parents were ever employed in the STEM sector while particpating in the SOEP.
This particular aspect of the specification is inherently problematic because pre-Reunification occupational data is not available for the East German population.
The utilisation of post-Reunification information introduces the disturbances of the 1990's into the model, which may result in a downward bias in the measured effects.
MotherEast$_{i}$ and FatherEast$_{i}$ indicate whether the parents are of East German origin.
This is defined as having resided in East Germany as opposed to West Germany in 1989.
$X'_{itus}$ is the transpose of a control matrix containing the following individual and federal state level controls:
age, age squared, and indirect migration background as individual level controls and unemployment rate, gross domestic product, population density, commuting balance (commuters in - commuters out), and first semester university students as federal state level controls.
The federal state level controls come from the \cite{INKAR2024}.
\cite{Sahoo2021} and \cite{Chevalier2013} both control for mother's and father's years of education when modelling choice of educational stream and post-compulsory schooling participation, respectively.
The specification in this paper, however, abstains from the inclusion of these controls because they might be highly correlated with the regressors of interest.

$t$ equals the latest survey year where the individual has valid information regarding the dependent variable.
Note that all control variables are so-called gross variables that are available even in the absence of a successful person interview with the respondent.
Consequently, these variables are never missing.
Every individual appears only once in the estimation sample which distincts this approach from a pooled panel approach.
The federal state level control variables are fixed at the survey year $u$ when the individual turned 17 years old.
This decision was made in order to control for the economic environment in which the individual was most probably situated at the time of making the decision regarding the two dependent variables.

Note that for all individuals with nonmissing values in STEM$_{itus}$, University$_{itus} = 1$, that is, the second part of the application is done on a subsample of the first part's sample with University$_{itus} = 1$ for all $i$.

In both parts, a restricted version of the model based on equation (\ref{eq:specification}) is initially estimated, which is given by

\vspace{-8mm}

\begin{equation}
	\begin{split}
		\label{eq:specificationrestr}
		\textnormal{Education}_{itus} &={} \beta_{0} + \beta_{1}\textnormal{MotherEverSTEM}_{i} + \beta_{2}\textnormal{FatherEverSTEM}_{i} \\
		& + \gamma{}X'_{itus} + \epsilon_{itus}\textnormal{.}
	\end{split}
\end{equation}

This restricted version is intended to serve as a point of comparison with the unrestriced model, which incorporates the distinctive context of reunified Germany.

In order to investigate the heterogeneity of the effects regarding sex, all regressions are run separately for females and males.
The standard errors are clustered at the household level to correct for within-household correlation.

In order to mitigate the problem of small sample size, two different strategies are explored in sections \ref{robustness} and \ref{extension}, namely model simplification and the use of an alternative dependent variable.

\section{Results}
\label{results}

Section \ref{descriptives} presents the results of the descriptive analysis and section \ref{educational} presents the results from the application of the epidemiological approach.

\subsection{Descriptive Analysis}
\label{descriptives}

The first empirical subsection of this paper aims to provide a descriptive account of the trends in the proportion of professionals working in STEM fields within specific demographic groups in Germany during the initial decade following the Reunification.
Person-level SOEP data from the survey years 1990--1999 is utilised.
The sample drawn from the SOEP contains 424,829 observations from 56,780 distinct individuals.
Restricting the sample to individuals of working age (17--65) who are either full- or part-time employed yields a total of 159,221 observations from 37,225 unique individuals.
Figure \ref{fig:timetrend} shows the development of the share of STEM professionals between 1984--2017.
Four demographic groups that result from combining the two binary characteristics ``Male/Female'' and ``East/West'' in every possible way are formed and their trends depicted.

\begin{figure}[ht]
	\centering
	\caption{Time Trend (by Region and Gender), 1984--2017}
	\label{fig:timetrend}
	\fontsize{9pt}{11pt}\selectfont
	\def\svgwidth{.9\textwidth}
	\input{trend.pdf_tex}
	\vspace{2mm}
	\parbox{10cm}{
		\linespread{1}\footnotesize Note: Own calculations based on \cite{SOEP2023}. The mean number of individuals represented by each marker is 1,264, with a standard deviation of 1,116. Person weights provided by the SOEP are applied when collapsing the data into the respective groups. The vertical dashed line marks the year of the Reunification.}
\end{figure}

In order to account for oversampling, the person weights provided by the SOEP are applied when collapsing the data into the respective groups.
Every marker in the figure contains the information of 1,264 individuals on average.
While data points already exist for West Germans from 1984 onwards, East Germans only enter the panel from 1990 onwards.

Looking at 1990 reveals that both females and males of East German origin have approximately the same number of STEM professionals relative to all full- or part-time employed individuals of the same sex.
Within the first decade following the Reunification, the trend of the East German females converges into the trend of the West German females, which oscillates around 5\%.
This convergence is seen to be robust throughout the entirety of the time span observed.
The West German males show a share of approximately 14\% in 1990.
This share increases in a roughly linear fashion over the entire time span observed.
The East German males appear to be following this trend, albeit with a constant gap of around 3\% below their West German peers.
This finding appears to be somewhat at odds with the findings of \cite{Jessen2023} and \cite{FriedmanSokuler2020}, who both find that immigrants from socialist countries influence their peers native to the host country in terms of female labour supply and educational choices.
However, this figure suggests that East Germans follow the trends of their West German counterparts with regard to STEM occupations.
Nevertheless, this finding is purely descriptive and lacks a causal framework that could validate such a suggestion.
The convergence of trends may be attributable to external factors that are not readily discernible from the internally driven decisions of individuals.

Figure \ref{fig:timetrendzoom} focuses on the trends of the females within the first decade following the Reunification.
From 1992 onwards, the 95\% confidence intervals begin to overlap between the two trends, indicating a quick convergence after a mere three years following the fall of the Berlin Wall.
The confidence intervals remain overlapping after 1992.

\begin{figure}[ht]
	\centering
	\caption{Time Trend (by Region, Females only), 1990--1999}
	\label{fig:timetrendzoom}
	\fontsize{9pt}{11pt}\selectfont
	\def\svgwidth{.9\textwidth}
	\input{trend_zoomed.pdf_tex}
	\vspace{2mm}
	\parbox{10cm}{
		\linespread{1}\footnotesize Note: Own calculations based on \cite{SOEP2023}. The mean number of individuals represented by each marker is 812, with a standard deviation of 535. Person weights provided by the SOEP are applied when collapsing the data into the respective groups. The whiskers show the 95\% confidence intervals.}
\end{figure}

For a better understanding of the individuals in question, tables \ref{tab:descr_summary_east} and \ref{tab:descr_summary_west} provide summary statistics for East and West Germans in the survey year 1990.
The subsample sizes for the four groups are 1,490 East German females and 1,645 males and 343 West German females and 377 males.

\begin{table}[ht]
	\caption{Summary Statistics for East Germans in Survey Year 1990}
	\label{tab:descr_summary_east}
	\begin{center}
		\begin{tabular}{l*{3}{c}}
			\toprule
			& (1) & (2) & (1) - (2) \\
			\midrule
			STEM Profession & 0.11  & 0.11  &   0.00     \\
			&   (0.01)  & (0.01) & (0.01) \\
			\addlinespace
			Age         &   38.63   &  39.32  &  -0.69\sym{*}     \\
			&     (0.28) &        (0.28)         &      (0.40) \\
			\addlinespace
			Spouse/Life Partner &  0.82      &  0.84  &    -0.02    \\
			&      (0.01)&          (0.01)&         (0.01) \\
			\addlinespace
			Household Size      &  3.21   &  3.28   &   -0.07\sym{*}       \\
			&          (0.03)&       (0.03)        &      (0.04)\\
			\addlinespace
			Residence in West Germany&  0.00   &   0.00  &  0.00         \\
			&         (0.00) &       (0.00)&  (0.00)\\
			\midrule
			Observations        &  1,490    &    1,645     &      3,135             \\
			\bottomrule
		\end{tabular}
		
		\vspace{2mm}
		
		\parbox{10cm}{
			\linespread{1}\footnotesize Note: \sym{*} \(p<0.10\), \sym{**} \(p<0.05\), \sym{***} \(p<0.01\). Own calculations based on \cite{SOEP2023}. The sample is restricted to individuals who were born in Germany. The sample comprises individuals of working age (16-65 years old) in the survey year 1990 with information on their employment status, ISCO-88 occupation code, and location in 1989 available. Standard errors are given in parentheses. Individuals are considered to be East German if their household head lived in the GDR prior to Reunification in 1990. Columns (1) and (2) show the mean values of the selected characteristics for females and males. Column (3) presents the differences in means obtained by employing two-sample Student's t-tests.}
		
	\end{center}
\end{table}

\begin{table}[ht]
	\caption{Summary Statistics for West Germans in Survey Year 1990}
	\label{tab:descr_summary_west}
	\begin{center}
		\begin{tabular}{l*{3}{c}}
			\toprule
			& (1) & (2) & (1) - (2) \\
			\midrule
			STEM Profession & 0.06  & 0.14 &  -0.08\sym{***}      \\
			&   (0.01)  & (0.02) & (0.02) \\
			\addlinespace
			Age         & 30.81     &  31.60   &    -0.79     \\
			&     (0.48) &        (0.49)         &      (0.69) \\
			\addlinespace
			Spouse/Life Partner & 0.68       &  0.62   &     0.06\sym{*}      \\
			&      (0.03)&          (0.03)&         (0.04) \\
			\addlinespace
			Household Size      &  2.93   &  2.89   &    0.04       \\
			&          (0.07)&       (0.07)        &      (0.10)\\
			\addlinespace
			Residence in West Germany& 0.96    &  0.97   &    -0.02        \\
			&         (0.01) &       (0.01)&  (0.01)\\
			\midrule
			Observations        &        343 &     377     &       720                \\
			\bottomrule
		\end{tabular}
		
		\vspace{2mm}
		
		\parbox{10cm}{
			\linespread{1}\footnotesize Note: \sym{*} \(p<0.10\), \sym{**} \(p<0.05\), \sym{***} \(p<0.01\). Own calculations based on \cite{SOEP2023}. The sample is restricted to individuals who were born in Germany. The sample comprises individuals of working age (16-65 years old) in the survey year 1990 with information on their employment status, ISCO-88 occupation code, and location in 1989 available. Standard errors are given in parentheses. Individuals are considered to be West German if their household head lived in the FRG prior to Reunification in 1990. Columns (1) and (2) show the mean values of the selected characteristics for females and males. Column (3) presents the differences in means obtained by employing two-sample Student's t-tests.}
		
	\end{center}
\end{table}

One possible reason for the low subsample size for the West Germans can be that approximately 93\% of these individuals stem from the SOEP's inital sample ``A'' which was drawn in 1984.
By 1990, panel attrition shrank the number of successful person interviews in the samples ``A'' and ``B'' down to approximately 78\% of the number in 1984 \citep{Siegers2022}.
In the year 1990, 9,519 sucessful person interviews were conducted for these samples.
However, a large portion of the interviewed individuals are either not of working age, are unemployed, or lack occupational information.
Applying the aforementioned filters results in a sample comprising 720 West German individuals.

In contrast, the then freshly drawn sample ``C'' for East Germany, based on the GDR-Master-Sample designed by Infratest in cooperation with the Department for Social Research of the Radio of GDR, comprises 4,453 successful person interviews in 1990 \citep{Infratest2011, Siegers2022}.
The sample used in the descriptive analysis of this paper comprises 3,135 individuals in the survey year 1990.
Columns (1) and (2) in both tables show the mean values of the selected characteristics for females and males.
Column (3) presents the differences in means, i. e. (1) - (2).
While both East German females and males show statistically identical shares of STEM professionals of 11\%, the difference in shares between West German females and males is -8 percentage points (pp for short).

The East German individuals are approximately 8 years older than their West German peers and therefore also have a higher share of individuals with spouses or life partners.
The proportion of West German females in a relationship is, on average, significantly higher than that of West German males (68\% versus 62\%), despite the former being, on average, slightly younger than the latter.
There is no significant difference in the proportion of East German individuals in a relationship between the sexes.

Figures \ref{fig:survivalfemale} and \ref{fig:survivalmale} show the progression of Eastern females and males who reported being occupied in the STEM sector in 1990, in terms of their occupational status over the subsequent decade.
In 1990, 163 female and 184 male STEM professionals are observed in the East.
The two groups are monitored over the subsequent nine survey years, with occupational data presented as either employment in STEM, employment in non-STEM, lack of regular employment, or missing information.

\begin{figure}[ht]
	\centering
	\caption{Development of 1990's Eastern Female STEM Professionals, 1990--1999}
	\label{fig:survivalfemale}
	\fontsize{9pt}{11pt}\selectfont
	\def\svgwidth{.9\textwidth}
	\input{survival_female.pdf_tex}
	\vspace{2mm}
	\parbox{10cm}{
		\linespread{1}\footnotesize Note: Own calculations based on \cite{SOEP2023}. In 1990, 163 female STEM professionals are observed in the East. The figure shows how this cohort develops in the subsequent decade with regard to occupational status. The share of item- and (partial) unit-nonresponse is particularly high for the years 1994, 1996, and 1999 due to a large share of missing information in the ISCO-88 variable for these survey years.}
\end{figure}

\begin{figure}[ht]
	\centering
	\caption{Development of 1990's Eastern Male STEM Professionals, 1990--1999}
	\label{fig:survivalmale}
	\fontsize{9pt}{11pt}\selectfont
	\def\svgwidth{.9\textwidth}
	\input{survival_male.pdf_tex}
	\vspace{2mm}
	\parbox{10cm}{
		\linespread{1}\footnotesize Note: Own calculations based on \cite{SOEP2023}. In 1990, 184 male STEM professionals are observed in the East. The figure shows how this cohort develops in the subsequent decade with regard to occupational status. The share of item- and (partial) unit-nonresponse is particularly high for the years 1994, 1996, and 1999 due to a large share of missing information in the ISCO-88 variable for these survey years.}
\end{figure}

With regard to the female cohort, the share of STEM professionals decreases to approximately 47\% in 1991, while 26\% of the 163 females work in non-STEM, and 21\% are not engaged in regular employment.
The share of STEM professionals in the male cohort decreases to 50\%, while 30\% of the 184 males work in non-STEM, and only 8\% have no regular employment.
By 1998, the share of STEM professionals within the female cohort had decreased to 19\%, while within the male cohort it had decreased to 23\%.
Note that for the years 1994, 1996, and 1999, the share of item- and (partial) unit-nonresponse is particularly high due to a large share of missing information in the ISCO-88 variable for these survey years.
With the exception of these years, the share of individuals employed in non-STEM is consistently higher than the share of individuals without regular employment for both the female and male cohort.
This appears to be reasonable, given that the majority of STEM roles require a tertiary qualification in the relevant field.
Therefore, the individuals in the observed cohorts are highly educated and, as a result, more in demand in the labour market, which makes them less likely to become unemployed.
In fact, the females have an average of 13 years of education and the males have an average of 14 years of education.
Subtracting the ten years spent in secondary school, this leaves them with 3--4 years spent either in an engineering school, a technical college, or a university.

Figure \ref{fig:netswitches} explores the occupational developments for Eastern females in a different manner.
First, the figure considers all Eastern females, whether they are employed in STEM or non-STEM roles, or are not in regular employment at all.
Second, the figure shows the netted out movements from one occupational status to another.
Thus, for instance, a female who is employed in a STEM role and transitions to a non-STEM position in the subsequent year is classified as having made a switch from STEM to non-STEM.
All switches between two survey years are summed up to a total of switches from one status to another.
Switches in the opposite direction are accounted for by substrating these negative switches from the total of the positive switches.
This yields the net switches between two survey years.

\begin{figure}[ht]
	\centering
	\caption{Net Switches among Eastern Females, 1990--1999}
	\label{fig:netswitches}
	\fontsize{9pt}{11pt}\selectfont
	\def\svgwidth{.9\textwidth}
	\input{net_switches.pdf_tex}
	\vspace{2mm}
	\parbox{10cm}{
		\linespread{1}\footnotesize Note: Own calculations based on \cite{SOEP2023}. All Eastern females in the survey years 1990--1999, whether they are employed in STEM or non-STEM roles, or are not in regular employment at all are considered in this figure. The figure shows the netted out movements from one occupational status to another. The markers along $y = 0$ represent the sample size for each survey year.}
\end{figure}

The following three pathways are considered: employment (full- or part-time) to no regular employment, STEM to non-STEM, and STEM to no regular employment.
The markers along $y = 0$ represent the sample size for each survey year.
Given that the markers remain approximately constant in size over the observed period, this indicates that the attrition of panel members is offset by the inflow of new participants to the panel over time.

From 1990 to 1991, there is the highest net movement along all three pathways.
The net movement for the pathways ``STEM to non-STEM'' and ``STEM to no regular employment'' decreases in the subsequent years until it equals approximately zero between 1993 and 1994.
The net movement from employment to no regular employment between 1994 and 1995 is negative, indicating that a greater number of Eastern females transitioned from no regular employment to employment than the reverse.

In conclusion, the netted out total of female departure from the STEM sector was the highest in the initial year following the Reunification, after which the total declined steadily before reaching zero.
The net movement from STEM to no regular employment is consistently below the net movement from STEM to non-STEM, which lends further support to the findings derived from figure \ref{fig:survivalfemale}, namely, that unemployment was less of an issue with these highly trained individuals.

 \subsection[Legacy Effect on Educational Choices]{Legacy Effect on Educational Choices\footnote{All results presented in this section were additionally re-estimated using OLS instead of logit and no notable differences were found between the resulting estimates. Only the original logit estimates are reported in the paper.}}
\label{educational}

This paper presents a twofold analysis of the effect of parents' upbringing in the GDR on the educational choices of their children in reunified Germany:
First, the effect on the extensive margin, that is to say, whether an adult child has obtained a university degree rather than a vocational degree, is examined.
Second, the effect on the intensive margin, namely whether an adult child has a STEM university degree rather than a non-STEM university degree, is investigated.

Table \ref{tab:descr_summary_epid_ext} provides summary statistics for the sample of the extensive analysis (see table \ref{tab:descr_summary_epid_ext_appendix} in the appendix for extended summary statistics).
The sample comprises 880 females and 898 males both around the age of 27.
Females have a significantly higher share of university graduates (42\%) than males (36\%).
The remaining 58\% of females and 64\% of males have a vocational degree.
Females also have a significantly higher share of individuals with a spouse or a life partner (39\% versus 27\%).
All remaining observed characteristics, including parental origin, show no statistically significant differences.

Table \ref{tab:extdaughters} presents the effects on the extensive educational decision of daughters.
Columns (1) through (3) show estimates of the restricted model based on equation (\ref{eq:specificationrestr}), while columns (4) through (6) show estimates of the unrestricted model based on equation (\ref{eq:specification}).
In both the restricted and unrestricted models, person and federal state controls are incorporated in a stepwise manner, with the baseline model initially estimated without any controls.

The estimates from the restricted model show that both having a mother who works or used to work in STEM and having a father who works or used to work in STEM significantly increase the probability of the daughter of obtaining a university degree instead of a vocational degree.
In the case of a mother with STEM experience, the increase measures 9 pp c. p. and in the case of a father with STEM experience, it measures 12 pp c. p.

Taking advantage of the unique setting in reunified Germany and taking into account the parental origin in the unrestricted model, the effect of the father being in STEM at some point in time is reduced to 8 pp, while the effect of the mother being in STEM at some point in time more than doubles in magnitude.
This is the consequence of addressing the omitted variable biases by adding the variables of parental origin as regressors into the model.
The results indicate a negative bias for STEM mothers and a positive bias for STEM fathers.
Specifically, having an East German mother is associated with a 21 pp reduction in the probability of a daughter obtaining a university degree c. p., while having an East German father is associated with a 12 pp increase in this probability c. p.
However, the latter estimate is not statistically significant.

The additional effects of having an East German mother who has also worked in STEM at some point $(\textnormal{MotherEverSTEM}_{i} \times \textnormal{MotherEast}_{i})$  and having an East German father who has also worked in STEM at some point $(\textnormal{FatherEverSTEM}_{i} \times \textnormal{FatherEast}_{i})$ show in opposite directions:
It is a negative additional effect of 31 pp for the mothers, whereas for the fathers, it is a positive additional effect of 17 pp.
Both effects are statistically significant at the 5\% level.
Therefore, while East German female STEM professionals appear to be inclined towards encouraging their daughters to pursue a non-academic career, East German male STEM professionals seem to show the opposite tendency.

Table \ref{tab:extsons} presents the effects on the extensive educational decision of sons.
Starting with the restrictive model again, columns (1) through (3) show that, for sons, having a mother who worked in STEM at some point in time and having a father who worked in STEM at some point in time both significantly increase the probability of the son obtaining a university degree rather than a vocational degree.
Interestingly, the effect is higher for mothers than for fathers (12 pp versus 10 pp).
Looking back to the daughters, the opposite is true.

\begin{table}[ht]
	\caption{Summary Statistics for Adult Children (Extensive Analysis)}
	\label{tab:descr_summary_epid_ext}
	\begin{center}
		\begin{tabular}{l*{3}{c}}
			\toprule
			& (1) & (2) & (1) - (2) \\
			\midrule
			University Degree instead of Vocational Degree & 0.42  & 0.36 & 0.06\sym{**}    \\
			&   (0.02)  & (0.02) & (0.02) \\
			\addlinespace
			Age         &   26.79   &  26.88   &  -0.09    \\
			&     (0.15) &        (0.14)         &      (0.20) \\
			\addlinespace
			Spouse/Life Partner &   0.39    &  0.27 &   0.12\sym{***}    \\
			&      (0.02)&          (0.01)&         (0.02) \\
			\addlinespace
			Household Size      &  2.78   & 2.78    &   0.00        \\
			&          (0.05)&       (0.04)   &   (0.06) \\
			\addlinespace
			Residence in West Germany  &  0.75   &  0.73   &  0.02    \\
			&         (0.01) &       (0.01)&  (0.02)\\
			\addlinespace
			Mother: East German Origin &  0.29   & 0.29    &  -0.01       \\
			&         (0.02) &   (0.02)&  (0.02)\\
			\addlinespace
			Father: East German Origin &  0.27   &   0.27  &    -0.01     \\
			&         (0.02) &       (0.02)&  (0.02)\\
			\addlinespace
			Mother: Ever STEM Profession & 0.08   &  0.06   &  0.02   \\
			&         (0.01) &       (0.01) &  (0.01) \\
			\addlinespace
			Father: Ever STEM Profession &   0.27  &  0.27   &  0.00   \\
			&         (0.01) &       (0.01)&  (0.02)\\
			\addlinespace
			Indirect Migration Background &  0.01   & 0.01    &  0.00    \\
			&         (0.00) &       (0.00)&  (0.01) \\
			\midrule
			Observations        &  880     &  898     &        1,778            \\
			\bottomrule
		\end{tabular}
		
		\vspace{2mm}
		
		\parbox{10cm}{
			\linespread{1}\footnotesize Note: \sym{*} \(p<0.10\), \sym{**} \(p<0.05\), \sym{***} \(p<0.01\). Own calculations based on \cite{SOEP2023}. The sample is restricted to individuals who were born in Germany. Every individual from the sample used in the descriptive analysis who has one or more children who themselves participate in the SOEP is linked to them. The presented estimation sample comprises all adult children who have both parents sucessfully linked to them. They were born after 1983, i.e. they were 6 years or younger at the time of Reunification, and have valid information on whether they received a university or vocational degree. Standard errors are given in parentheses. Columns (1) and (2) show the mean values of the selected characteristics for females and males. Column (3) presents the differences in means obtained by employing two-sample Student's t-tests. Table \ref{tab:descr_summary_epid_ext_appendix} in the appendix presents an extended version of this table.}
		
	\end{center}
\end{table}

\begin{landscape}
	\begin{table}[ht]
		\caption[Extensive Margins for Daughters]{Extensive Margins (University Degree or Vocational Degree) for Daughters}
		\label{tab:extdaughters}
		\begin{center}
			\resizebox{1.24\textwidth}{!}{\begin{tabular}{l*{6}{c}}
					\toprule
					&\multicolumn{1}{c}{(1)}         &\multicolumn{1}{c}{(2)}         &\multicolumn{1}{c}{(3)}         &\multicolumn{1}{c}{(4)}         &\multicolumn{1}{c}{(5)}         &\multicolumn{1}{c}{(6)}         \\
					\midrule
					Mother: Ever STEM Profession&      0.1159\sym{*}  &      0.0789         &      0.0943\sym{*}  &      0.2025\sym{***}&      0.2017\sym{***}&      0.2081\sym{***}\\
					&    (0.0602)         &    (0.0573)         &    (0.0561)         &    (0.0768)         &    (0.0730)         &    (0.0734)         \\
					\addlinespace
					Father: Ever STEM Profession&      0.1487\sym{***}&      0.1382\sym{***}&      0.1238\sym{***}&      0.1028\sym{**} &      0.0919\sym{**} &      0.0842\sym{**} \\
					&    (0.0352)         &    (0.0335)         &    (0.0332)         &    (0.0403)         &    (0.0380)         &    (0.0380)         \\
					\addlinespace
					Mother: Eastern Origin&                     &                     &                     &     -0.2449\sym{**} &     -0.2194\sym{**} &     -0.2145\sym{**} \\
					&                     &                     &                     &    (0.1213)         &    (0.1100)         &    (0.1092)         \\
					\addlinespace
					Father: Eastern Origin&                     &                     &                     &      0.0709         &      0.0511         &      0.1228         \\
					&                     &                     &                     &    (0.1258)         &    (0.1144)         &    (0.1193)         \\
					\addlinespace
					Mother: Ever STEM Profession $\times$ Mother: Eastern Origin&                     &                     &                     &     -0.2121         &     -0.2979\sym{**} &     -0.3129\sym{**} \\
					&                     &                     &                     &    (0.1309)         &    (0.1222)         &    (0.1222)         \\
					\addlinespace
					Father: Ever STEM Profession $\times$ Father: Eastern Origin&                     &                     &                     &      0.1513\sym{*}  &      0.1535\sym{*}  &      0.1736\sym{**} \\
					&                     &                     &                     &    (0.0852)         &    (0.0810)         &    (0.0806)         \\
					\midrule
					Person Controls & & X  & X & & X & X \\
					Federal State Controls & & & X & & & X \\
					Rank                &      2         &      5         &     10         &      6         &      9         &     14         \\
					Observations                   &    880         &    880         &    880         &    880         &    880         &    880         \\
					\bottomrule
			\end{tabular}}
			
			\vspace{2mm}
			
			\parbox{15cm}{
				\linespread{1}\footnotesize Note: \sym{*} \(p<0.10\), \sym{**} \(p<0.05\), \sym{***} \(p<0.01\). Margins (dy/dx) of a Logistic Regression Model. Own calculations based on \cite{SOEP2023}. For more information on the estimation sample, see table \ref{tab:descr_summary_epid_ext}. Standard errors clustered at the household level are given in parentheses. Columns (1) through (3) show estimates of the restricted model based on equation (\ref{eq:specificationrestr}), while columns (4) through (6) show estimates of the unrestricted model based on equation (\ref{eq:specification}). In both the restricted and unrestricted models, person and federal state control are incorporated in a stepwise manner, with the baseline model initially estimated without any controls.}
			
		\end{center}
	\end{table}
\end{landscape}

\begin{landscape}
	\begin{table}[ht]
		\caption[Extensive Margins for Sons]{Extensive Margins (University Degree or Vocational Degree) for Sons}
		\label{tab:extsons}
		\begin{center}
			\resizebox{1.23\textwidth}{!}{\begin{tabular}{l*{6}{c}}
					\toprule
					&\multicolumn{1}{c}{(1)}         &\multicolumn{1}{c}{(2)}         &\multicolumn{1}{c}{(3)}         &\multicolumn{1}{c}{(4)}         &\multicolumn{1}{c}{(5)}         &\multicolumn{1}{c}{(6)}         \\
					\midrule
					Mother: Ever STEM Profession&      0.0835         &      0.0810         &      0.1157\sym{*}  &      0.0844         &      0.1122         &      0.1086         \\
					&    (0.0653)         &    (0.0622)         &    (0.0617)         &    (0.0878)         &    (0.0827)         &    (0.0812)         \\
					\addlinespace
					Father: Ever STEM Profession&      0.1166\sym{***}&      0.1078\sym{***}&      0.0978\sym{***}&      0.0903\sym{**} &      0.0848\sym{**} &      0.0790\sym{**} \\
					&    (0.0343)         &    (0.0326)         &    (0.0321)         &    (0.0395)         &    (0.0373)         &    (0.0368)         \\
					\addlinespace
					Mother: Eastern Origin&                     &                     &                     &     -0.0922         &     -0.1082         &     -0.0802         \\
					&                     &                     &                     &    (0.0996)         &    (0.0940)         &    (0.0914)         \\
					\addlinespace
					Father: Eastern Origin&                     &                     &                     &     -0.0394         &     -0.0420         &      0.1257         \\
					&                     &                     &                     &    (0.1038)         &    (0.0981)         &    (0.1038)         \\
					\addlinespace
					Mother: Ever STEM Profession $\times$ Mother: Eastern Origin&                     &                     &                     &      0.0371         &     -0.0191         &      0.0132         \\
					&                     &                     &                     &    (0.1328)         &    (0.1257)         &    (0.1275)         \\
					\addlinespace
					Father: Ever STEM Profession $\times$ Father: Eastern Origin&                     &                     &                     &      0.0853         &      0.0674         &      0.0890         \\
					&                     &                     &                     &    (0.0822)         &    (0.0771)         &    (0.0771)         \\
					\midrule
					Person Controls & & X  & X & & X & X \\
					Federal State Controls & & & X & & & X \\
					Rank                &      2         &      5         &     10         &      6         &      9         &     14         \\
					Observations   &    898         &    898         &    898         &    898         &    898         &    898         \\
					\bottomrule
			\end{tabular}}
			
			\vspace{2mm}
			
			\parbox{15cm}{
				\linespread{1}\footnotesize Note: \sym{*} \(p<0.10\), \sym{**} \(p<0.05\), \sym{***} \(p<0.01\). Margins (dy/dx) of a Logistic Regression Model. Own calculations based on \cite{SOEP2023}. For more information on the estimation sample, see table \ref{tab:descr_summary_epid_ext}. Standard errors clustered at the household level are given in parentheses. Columns (1) through (3) show estimates of the restricted model based on equation (\ref{eq:specificationrestr}), while columns (4) through (6) show estimates of the unrestricted model based on equation (\ref{eq:specification}). In both the restricted and unrestricted models, person and federal state control are incorporated in a stepwise manner, with the baseline model initially estimated without any controls.}
			
		\end{center}
	\end{table}
\end{landscape}

Comparing the results from the restricted model in column (3) to the results from the unrestricted model in column (6) reveals that, for sons, the omitted variable bias from the parental origin is not nearly as high nor as significant as for daughters.
Further, the omitted variable bias is now positive not only for the fathers, but also for the mothers:
The effect for mothers decreases to 11 pp and for fathers to 8 pp.
This indicates that the parental origin is more influential for daughters in their educational choices than for sons.
Furthermore, the findings indicate that Eastern mothers do not exert a significant negative effect on their sons' decision to pursue an academic education, but they do exert it on their daughters'.

The following part investigates the effect on the intensive margin, specifically if daughters and sons choose to obtain a STEM university degree in preference to a non-STEM university degree.
For this part of the analysis, the sample size is notably reduced to 298 females and 287 males, as the sample employed in this part constitutes a subsample of the sample utilised before.
Table \ref{tab:descr_summary_epid_int} provides summary statistics for the sample of the intensive analysis (see table \ref{tab:descr_summary_epid_int_appendix} in the appendix for extended summary statistics).
Comparing these statistics to the ones from table \ref{tab:descr_summary_epid_ext} shows that the analysis is done on a non-random subsample of the sample employed in the extensive analysis.
First, the individuals are older on average with approximately 29 years of age as opposed to 27 years.
In fact, using age as the single regressor for estimating whether an individual in the larger sample has a missing value in the dependent variable of the intensive analysis (STEM degree versus non-STEM degree) reveals that age is a highly significant predictor for having non-valid information in this variable.\footnote{The logit regression $\textnormal{UniFieldMissing}_{it} = \textnormal{Age}_{it} + u_{it}$ was run on the sample described in table \ref{tab:descr_summary_epid_ext}. Standard errors were clustered at the household level.}
However, males and females within the subsample are statistically of the same age.
Females still have a higher share of individuals with a spouse or life partner in comparison to males (13 pp difference).
All the remaining characteristics  show no statistically significant differences between the sexes and also remain comparable in magnitude with the values in the larger sample.
The shares of individuals with a mother or a father of Eastern origin are now slightly lower than in the larger sample, while the shares of individuals with parents who worked in STEM at some point in time are now slightly higher.

Table \ref{tab:intdaughters} presents the effects on the intensive educational decision of daughters.
The first thing to note is that none of the regressors in both the restricted and unrestricted models have significant coefficients, except for the interaction $(\textnormal{FatherEverSTEM}_{i} \times \textnormal{FatherEast}_{i})$  in the baseline stage of the unrestricted model in column (4).
This may be due to the small sample size.
However, when this result is later compared with the effects on sons, where the sample size is even smaller, the lack of significant predictors may not only be due to high imprecision.
There is little change in the coefficients between the restricted and the unrestricted model, suggesting that parental origin is not as influential in the intensive decision as it is in the extensive decision for daughters.
The effect of having a mother from the East still reads -21 pp, however, the effect of having a mother who worked in STEM at some point in time is reduced to 8 pp for the intensive decision.
This suggests that STEM mothers are supportive of their daughters pursuing a university degree, but are less concerned about whether it should be a STEM degree.
Additionally, the interactive effect of having an East German STEM mother reads 1 pp.
Keeping in mind that all of these effects are not statistically different from zero on all common confidence levels, this means that there is no evidence to support the theory that females are influenced by their East German STEM mothers to obtain a STEM degreee instead of a non-STEM degree at university.
Interestingly, the influence from the opposite-sex parent again encourages the child to make a positive decision, as it was the case with the extensive study decision:
As previously stated, the interaction $(\textnormal{FatherEverSTEM}_{i} \times \textnormal{FatherEast}_{i})$ produces the only significant coefficient in the baseline stage of the unrestricted model.
In the full unrestricted model presented in column (6), the interaction effect reads 18 pp, representing the largest positive effect observed in this model.
It appears that it is the East German STEM fathers rather than the mothers who are more likely to encourage their daughters to not only pursue a university degree, but to study a STEM subject at university.

Lastly, table \ref{tab:intsons} presents the effects on the intensive educational decision of sons.
The first thing that catches the eye is the highly significant effect that STEM fathers exert on their sons' decision on whether to follow their career path.
This effect is present throughout all stages of both the restricted and the unrestricted model and reads 19 pp in the full unrestricted model.
However, this effect cancels out if the father is from the East, as the effect of that reads 0 pp and the effect from the interaction term $(\textnormal{FatherEverSTEM}_{i} \times \textnormal{FatherEast}_{i})$  reads -21 pp.
It is again important to note that the sample size is relatively small, which limits the scope of the findings.
However, they indicate that the influence of a father's involvement in STEM has a significant positive impact only on the sons of Western fathers.
Having a STEM mother initially decreases the probability of a positive outcome, however, if the mother is from the East, the negative effect cancels out.

\begin{table}[ht]
	\caption{Summary Statistics for Adult Children (Intensive Analysis)}
	\label{tab:descr_summary_epid_int}
	\begin{center}
		\begin{tabular}{l*{3}{c}}
			\toprule
			& (1) & (2) & (1) - (2) \\
			\midrule
			STEM University Degree instead of non-STEM Degree & 0.26  & 0.47  &  -0.21\sym{***}   \\
			&   (0.03)  & (0.03) & (0.04) \\
			\addlinespace
			Age         &  28.56    &  28.80   &   -0.24   \\
			&     (0.23) &        (0.23)         &      (0.32) \\
			\addlinespace
			Spouse/Life Partner &   0.50    & 0.37  &  0.13\sym{***}     \\
			&      (0.03)&          (0.03)&         (0.04) \\
			\addlinespace
			Household Size      &  2.50   &  2.39   &      0.11     \\
			&          (0.07)&       (0.08)   &   (0.11) \\
			\addlinespace
			Residence in West Germany  &  0.80   &  0.75   &   0.04   \\
			&         (0.02) &       (0.03)&  (0.03)\\
			\addlinespace
			Mother: East German Origin &  0.21   & 0.25    &   -0.04      \\
			&         (0.02) &   (0.03)&  (0.03)\\
			\addlinespace
			Father: East German Origin &  0.20   &  0.23   &   -0.03      \\
			&         (0.02) &       (0.03)&  (0.03)\\
			\addlinespace
			Mother: Ever STEM Profession & 0.12   &  0.08   & 0.04    \\
			&         (0.02) &       (0.02) &  (0.03) \\
			\addlinespace
			Father: Ever STEM Profession &  0.34   & 0.35    & -0.01    \\
			&         (0.03) &       (0.03)&  (0.04)\\
			\addlinespace
			Indirect Migration Background &   0.02  &  0.01   &  0.01    \\
			&         (0.01) &       (0.01)&  (0.01) \\
			\midrule
			Observations        &  298     &  287     &        585            \\
			\bottomrule
		\end{tabular}
		
		\vspace{2mm}
		
		\parbox{10cm}{
			\linespread{1}\footnotesize Note: \sym{*} \(p<0.10\), \sym{**} \(p<0.05\), \sym{***} \(p<0.01\). Own calculations based on \cite{SOEP2023}. The sample is restricted to individuals who were born in Germany. Every individual from the sample used in the descriptive analysis who has one or more children who themselves participate in the SOEP is linked to them. The presented estimation sample comprises all adult children who have both parents sucessfully linked to them. They were born after 1983, i.e. they were 6 years or younger at the time of Reunification, and have valid information on their field of tertiary education. Standard errors are given in parentheses. Columns (1) and (2) show the mean values of the selected characteristics for females and males. Column (3) presents the differences in means obtained by employing two-sample Student's t-tests. Table \ref{tab:descr_summary_epid_int_appendix} in the appendix presents an extended version of this table.}
		
	\end{center}
\end{table}

\begin{landscape}
	\begin{table}[ht]
		\caption[Intensive Margins for Daughters]{Intensive Margins (STEM University Degree or non-STEM University Degree) for Daughters}
		\label{tab:intdaughters}
		\begin{center}
			\resizebox{1.2\textwidth}{!}{\begin{tabular}{l*{6}{c}}
					\toprule
					&\multicolumn{1}{c}{(1)}         &\multicolumn{1}{c}{(2)}         &\multicolumn{1}{c}{(3)}         &\multicolumn{1}{c}{(4)}         &\multicolumn{1}{c}{(5)}         &\multicolumn{1}{c}{(6)}         \\
					\midrule
					Mother: Ever STEM Profession&      0.1046         &      0.1097         &      0.0883         &      0.0864         &      0.0945         &      0.0846         \\
					&    (0.0709)         &    (0.0711)         &    (0.0717)         &    (0.0819)         &    (0.0822)         &    (0.0827)         \\
					\addlinespace
					Father: Ever STEM Profession&      0.0215         &      0.0240         &      0.0292         &     -0.0270         &     -0.0204         &     -0.0131         \\
					&    (0.0531)         &    (0.0531)         &    (0.0532)         &    (0.0601)         &    (0.0604)         &    (0.0606)         \\
					\addlinespace
					Mother: Eastern Origin&                     &                     &                     &     -0.1761         &     -0.1684         &     -0.2148         \\
					&                     &                     &                     &    (0.2575)         &    (0.2648)         &    (0.2817)         \\
					\addlinespace
					Father: Eastern Origin&                     &                     &                     &      0.0846         &      0.0755         &      0.0569         \\
					&                     &                     &                     &    (0.2594)         &    (0.2663)         &    (0.2807)         \\
					\addlinespace
					Mother: Ever STEM Profession $\times$ Mother: Eastern Origin&                     &                     &                     &      0.0348         &      0.0244         &      0.0066         \\
					&                     &                     &                     &    (0.1682)         &    (0.1695)         &    (0.1708)         \\
					\addlinespace
					Father: Ever STEM Profession $\times$ Father: Eastern Origin&                     &                     &                     &      0.2186\sym{*}  &      0.2003         &      0.1827         \\
					&                     &                     &                     &    (0.1295)         &    (0.1305)         &    (0.1333)         \\
					\midrule
					Person Controls & & X  & X & & X & X \\
					Federal State Controls & & & X & & & X \\
					Rank                &      2         &      5         &     10         &      6         &      9         &     14         \\
					Observations                   &    298         &    298         &    298         &    298         &    298         &    298         \\
					\bottomrule
			\end{tabular}}
			
			\vspace{2mm}
			
			\parbox{15cm}{
				\linespread{1}\footnotesize Note: \sym{*} \(p<0.10\), \sym{**} \(p<0.05\), \sym{***} \(p<0.01\). Margins (dy/dx) of a Logistic Regression Model. Own calculations based on \cite{SOEP2023}. For more information on the estimation sample, see table \ref{tab:descr_summary_epid_int}. Standard errors clustered at the household level are given in parentheses. Columns (1) through (3) show estimates of the restricted model based on equation (\ref{eq:specificationrestr}), while columns (4) through (6) show estimates of the unrestricted model based on equation (\ref{eq:specification}). In both the restricted and unrestricted models, person and federal state control are incorporated in a stepwise manner, with the baseline model initially estimated without any controls.}
			
		\end{center}
	\end{table}
\end{landscape}

\begin{landscape}
	\begin{table}[ht]
		\caption[Intensive Margins for Sons]{Intensive Margins (STEM University Degree or non-STEM University Degree) for Sons}
		\label{tab:intsons}
		\begin{center}
			\resizebox{1.2\textwidth}{!}{\begin{tabular}{l*{6}{c}}
					\toprule
					&\multicolumn{1}{c}{(1)}         &\multicolumn{1}{c}{(2)}         &\multicolumn{1}{c}{(3)}         &\multicolumn{1}{c}{(4)}         &\multicolumn{1}{c}{(5)}         &\multicolumn{1}{c}{(6)}         \\
					\midrule
					Mother: Ever STEM Profession&     -0.1601         &     -0.1593         &     -0.1595         &     -0.1724         &     -0.1636         &     -0.1542         \\
					&    (0.1087)         &    (0.1086)         &    (0.1094)         &    (0.1416)         &    (0.1418)         &    (0.1415)         \\
					\addlinespace
					Father: Ever STEM Profession&      0.1661\sym{***}&      0.1674\sym{***}&      0.1465\sym{**} &      0.2076\sym{***}&      0.2106\sym{***}&      0.1931\sym{***}\\
					&    (0.0592)         &    (0.0594)         &    (0.0601)         &    (0.0666)         &    (0.0672)         &    (0.0679)         \\
					\addlinespace
					Mother: Eastern Origin&                     &                     &                     &      0.0848         &      0.1178         &      0.1236         \\
					&                     &                     &                     &    (0.2266)         &    (0.2398)         &    (0.2452)         \\
					\addlinespace
					Father: Eastern Origin&                     &                     &                     &      0.0437         &      0.0065         &      0.0014         \\
					&                     &                     &                     &    (0.2388)         &    (0.2530)         &    (0.2751)         \\
					\addlinespace
					Mother: Ever STEM Profession $\times$ Mother: Eastern Origin&                     &                     &                     &      0.0430         &      0.0310         &      0.0284         \\
					&                     &                     &                     &    (0.2260)         &    (0.2261)         &    (0.2256)         \\
					\addlinespace
					Father: Ever STEM Profession $\times$ Father: Eastern Origin&                     &                     &                     &     -0.1902         &     -0.1999         &     -0.2110         \\
					&                     &                     &                     &    (0.1451)         &    (0.1459)         &    (0.1468)         \\
					\midrule
					Person Controls & & X  & X & & X & X \\
					Federal State Controls & & & X & & & X \\
					Rank                &      2         &      5         &     10         &      6         &      9         &     14         \\
					Observations &    287         &    287         &    287         &    287         &    287         &    287         \\
					\bottomrule
			\end{tabular}}
			
			\vspace{2mm}
			
			\parbox{15cm}{
				\linespread{1}\footnotesize Note: \sym{*} \(p<0.10\), \sym{**} \(p<0.05\), \sym{***} \(p<0.01\). Margins (dy/dx) of a Logistic Regression Model. Own calculations based on \cite{SOEP2023}. For more information on the estimation sample, see table \ref{tab:descr_summary_epid_int}. Standard errors clustered at the household level are given in parentheses. Columns (1) through (3) show estimates of the restricted model based on equation (\ref{eq:specificationrestr}), while columns (4) through (6) show estimates of the unrestricted model based on equation (\ref{eq:specification}). In both the restricted and unrestricted models, person and federal state control are incorporated in a stepwise manner, with the baseline model initially estimated without any controls.}
			
		\end{center}
	\end{table}
\end{landscape}

\section{Discussion}
\label{discussion}

The fact that only the occupational history of the parents after 1989 can be taken into account may lead to biased estimates in the intensive analysis. This is due to the fact presented in the descriptive analysis that a large portion of Eastern females left the STEM sector after 1989. The influence of the GDR's occupational ideology cannot therefore be neatly disentangled from the disturbances of the 1990s. With this in mind, this paper cautiously presents an interpretation of the results at hand.

The results of the extensive and the intensive analysis regarding educational choices show a pattern of the opposite-sex parent having a positive impact on their children's decisions.
The fact that Eastern STEM fathers have a positive effect on the choice their daughters make on whether to follow their career path in STEM needs to be pointed out.
This effect is negative when investigating Eastern STEM mothers and their daughters.

The results of the analysis, when considered alongside the findings of the descriptive analysis, allow for the following working theory:
that, following rejection from the STEM sector immediately after the Reunification, mothers who had worked in that sector in the GDR advised their daughters against pursuing the career path they had taken.
However, there was no such sharp decline in the share of males from the former GDR working in STEM.
Given the socialist upbringing of these males and their possible lack of disillusionment when compared to their female counterparts after the Reunification, it seems reasonable to suggest that they transmitted the socialist emphasis on STEM to their daughters.

A further perspective on the results can be provided by the fact that the political and cultural values in the GDR were communicated strictly top to bottom.
The loss of the motor that facilitated this transmission, namely the socialist system of the SED, led to a complete failure of spreading the ideas of socialism into the wider society.
This top-to-bottom design was necessary in the GDR because socialism did not ``organically'' grew, but was imposed from above by the Soviet Union in order to maintain a firm grip on the former enemy.
Further, the Soviet Union was also sceptical about the technological efforts of East Germany:
\cite{Hoegselius2009} points out that the Soviet Union largely isolated these efforts and was not inclined to support them because the citizens of the GDR were viewed as Germans rather than socialist brethren.

Added to this is the obvious fact that the people in the GDR were oppressed by their political system and that the economic system led to constant shortages in supplies, which led to general dissatisfaction.
It is therefore hardly surprising that a large share of individuals from the former GDR rejected the ideology of their failed state and did not allow it to influence their decisions in their own lives.
So, alongside the state of the GDR, its ideology fell as well.

The findings contrast the conclusions of \cite{FriedmanSokuler2020}, who suggest that socialist ideas are retained by post-Cold War generations in Israel and disseminated to peers with no parental background in a socialist state.
The findings of this study do not directly address the question of whether Eastern cultural values regarding STEM were dispersed to West Germans following the Reunification.
However, before values can be disseminated, they must first be preserved among peers within the cultural circle.
The findings indicate that this preservation can be shown to be achieved only through one channel, namely East German fathers with a background in STEM who encourage their daughters to pursue similar career paths.

\section{Robustness}
\label{robustness}

In an attempt to alleviate the problem of small sample sizes in the estimates presented above, I modify the regressors presented in equation (\ref{eq:specification}) as follows:

\vspace{-8mm}

\begin{equation}
	\begin{split}
		\label{eq:specificationrobust}
		\textnormal{Education}_{itus} &={} \beta_{0} + \beta_{1}\textnormal{AnyParentEverSTEM}_{i} + \beta_{2}\textnormal{AnyParentEast}_{i} \\
		& + \beta_{4}(\textnormal{AnyParentEverSTEM}_{i} \times \textnormal{AnyParentEast}_{i}) \\
		& + \gamma{}X'_{itus} + \epsilon_{itus}\textnormal{.}
	\end{split}
\end{equation}

\noindent The use of this alternative specification allows individuals to be included in the sample who have only one parent with valid information on origin and occupation instead of both parents.
Now it is no longer possible to distinguish between maternal and paternal influence, as they are merged into one parental influence stream.
Therefore, the coefficients of the regressor defined in equation (\ref{eq:specificationrobust}) are expected to lie between the coefficients of the maternal and paternal regressors in the original specification.

Table \ref{tab:intdaughtersrobust} shows the results of the intensive analysis on daughters using this alternative specification.
The sample size increases by 29\%.
As expected, the coefficents always fall between the coefficients of the regressors from the original analysis:
For example, in the full unrestricted model of the original analysis, the effect of having a mother from the East reads -21 pp and the effect of having a father from the East reads 6 pp (see table \ref{tab:intdaughters}).
Now, the effect of having any parent from the East reads -12 pp in the full unrestricted model.
This result confirms that these more generally defined coefficients are the means of the more nuanced effects of the original model.

The same result can be observed for the sons in table \ref{tab:intsonsrobust}:
For example, having a mother with a STEM background has an effect of -15 pp, while having a father with a STEM background has an effect of 19 pp (see table \ref{tab:intsons}).
Now, the effect of having any parent with a STEM background reads 9 pp.
The effect (which was highly significant on the paternal side) lost its significance due to the balancing of the opposing effects.

\clearpage

\begin{landscape}
	\begin{table}[ht]
		\caption[Intensive Margins for Daughters (Robustness)]{Intensive Margins (STEM University Degree or non-STEM University Degree) for Daughters (Robustness)}
		\label{tab:intdaughtersrobust}
		\begin{center}
			\resizebox{1.2\textwidth}{!}{\begin{tabular}{l*{6}{c}}
					\toprule
					&\multicolumn{1}{c}{(1)}         &\multicolumn{1}{c}{(2)}         &\multicolumn{1}{c}{(3)}         &\multicolumn{1}{c}{(4)}         &\multicolumn{1}{c}{(5)}         &\multicolumn{1}{c}{(6)}         \\
					\midrule
					Any Parent: Ever STEM Profession&      0.0852\sym{*}  &      0.0838\sym{*}  &      0.0838\sym{*}  &      0.0676         &      0.0697         &      0.0768         \\
					&    (0.0446)         &    (0.0445)         &    (0.0443)         &    (0.0503)         &    (0.0502)         &    (0.0503)         \\
					\addlinespace
					Any Parent: Eastern Origin&                     &                     &                     &     -0.0364         &     -0.0484         &     -0.1195         \\
					&                     &                     &                     &    (0.0760)         &    (0.0759)         &    (0.1041)         \\
					\addlinespace
					Any Parent: Ever STEM Profession $\times$ Any Parent: Eastern Origin&                     &                     &                     &      0.0866         &      0.0726         &      0.0457         \\
					&                     &                     &                     &    (0.1129)         &    (0.1128)         &    (0.1131)         \\
					\midrule
					Person Controls & & X  & X & & X & X \\
					Federal State Controls & & & X & & & X \\
					Rank                &      1         &      4         &      9         &      3         &      6         &     11         \\
					N                   &    384         &    384         &    384         &    384         &    384         &    384         \\
					\bottomrule
			\end{tabular}}
			
			\vspace{2mm}
			
			\parbox{15cm}{
				\linespread{1}\footnotesize Note: \sym{*} \(p<0.10\), \sym{**} \(p<0.05\), \sym{***} \(p<0.01\). Margins (dy/dx) of a Logistic Regression Model. Own calculations based on \cite{SOEP2023}. Standard errors clustered at the household level are given in parentheses. Columns (1) through (3) show estimates of the restricted model based on equation (\ref{eq:specificationrestr}), while columns (4) through (6) show estimates of the unrestricted model based on equation (\ref{eq:specification}). In both the restricted and unrestricted models, person and federal state control are incorporated in a stepwise manner, with the baseline model initially estimated without any controls. In contrast to the previous results, the data do not distinguish between mothers and fathers being from the East and/or having ever worked in STEM. If either of these characteristics is reported to be true for at least one parent, the corresponding binary variable takes the value ``1''.}
			
		\end{center}
	\end{table}
\end{landscape}

\begin{landscape}
	\begin{table}[ht]
		\caption[Intensive Margins for Sons (Robustness)]{Intensive Margins (STEM University Degree or non-STEM University Degree) for Sons (Robustness)}
		\label{tab:intsonsrobust}
		\begin{center}
			\resizebox{1.2\textwidth}{!}{\begin{tabular}{l*{6}{c}}
					\toprule
					&\multicolumn{1}{c}{(1)}         &\multicolumn{1}{c}{(2)}         &\multicolumn{1}{c}{(3)}         &\multicolumn{1}{c}{(4)}         &\multicolumn{1}{c}{(5)}         &\multicolumn{1}{c}{(6)}         \\
					\midrule
					Any Parent: Ever STEM Profession&      0.0794         &      0.0795         &      0.0636         &      0.0957         &      0.0980         &      0.0887         \\
					&    (0.0556)         &    (0.0559)         &    (0.0556)         &    (0.0639)         &    (0.0644)         &    (0.0637)         \\
					\addlinespace
					Any Parent: Eastern Origin&                     &                     &                     &      0.0842         &      0.0830         &      0.1101         \\
					&                     &                     &                     &    (0.0796)         &    (0.0797)         &    (0.1007)         \\
					\addlinespace
					Any Parent: Ever STEM Profession $\times$ Any Parent: Eastern Origin&                     &                     &                     &     -0.0798         &     -0.0871         &     -0.1034         \\
					&                     &                     &                     &    (0.1303)         &    (0.1312)         &    (0.1297)         \\
					\midrule
					Person Controls & & X  & X & & X & X \\
					Federal State Controls & & & X & & & X \\
					Rank                &      1         &      4         &      9         &      3         &      6         &     11         \\
					N                   &    348         &    348         &    348         &    348         &    348         &    348         \\
					\bottomrule
			\end{tabular}}
			
			\vspace{2mm}
			
			\parbox{15cm}{
				\linespread{1}\footnotesize Note: \sym{*} \(p<0.10\), \sym{**} \(p<0.05\), \sym{***} \(p<0.01\). Margins (dy/dx) of a Logistic Regression Model. Own calculations based on \cite{SOEP2023}. Standard errors clustered at the household level are given in parentheses. Columns (1) through (3) show estimates of the restricted model based on equation (\ref{eq:specificationrestr}), while columns (4) through (6) show estimates of the unrestricted model based on equation (\ref{eq:specification}). In both the restricted and unrestricted models, person and federal state control are incorporated in a stepwise manner, with the baseline model initially estimated without any controls. In contrast to the previous results, the data do not distinguish between mothers and fathers being from the East and/or having ever worked in STEM. If either of these characteristics is reported to be true for at least one parent, the corresponding binary variable takes the value ``1''.}
			
		\end{center}
	\end{table}
\end{landscape}

\clearpage

\section{Extension: A White- and Blue-Collar Definition of STEM}
\label{extension}

One thing that might be problematic about the definition of STEM in the context of the GDR is that while STEM refers to a highly academic category of occupations and curricula, the GDR understood itself as ``Worker and Peasant State'', which led to an intense propagation of blue-collar work, albeit highly technical, that does not fit the umbrella term with its U.S. origin.

Therefore, the regressand of the intensive analysis is now redefined to also include vocational degrees within the technical field instead of just university degrees.
These vocational degrees include degrees for occupations such as electromechanic, industrial mechanic, energy electronics technician, chemical laboratory technician, optician, and pharmaceutical-technical assistant.

Table \ref{tab:intdaughtersextension} presents the results of the intensive analysis on daughters using the alternative regressand.
The sample size increases by 272\%.
Comparing the full unrestricted model with the alternative regressand with the main results presented in the intensive analysis on daughters shows that the coefficients do not notably change, except that the coefficient for having a father from the East increases from 6 pp to 17 pp.
The standard error also halves.
This again supports the opposite-sex parent effect from the main results.
Despite drastically increasing the sample size and allowing for a more ``GDR-compliant'' definition of STEM, there are still no significant effects regarding the mothers.
This confirms the results from the main analysis and also shows that the opposite-sex parent channel of influence is strengthened when the definition of STEM is conformed.

Table \ref{tab:intsonsextension} presents the results for the sons using the alternative regressand.
The sample size increases by 301\%.
The positive influence of having a father in STEM weakens for the sons, falling from 19 pp to 10 pp.
The negative influence of having a mother in STEM also becomes weaker.
The interaction effect of having a mother from the East with a STEM background increases from 2 pp to 19 pp.

Therefore, conforming the regressand inclines Eastern STEM mothers to support their sons to pursue a technical occupation.
This again shows the positive opposite-sex parent effect.

\clearpage

\begin{landscape}
	\begin{table}[ht]
		\caption[Intensive Margins for Daughters (Extension)]{Intensive Margins (STEM University/Vocational Degree or non-STEM University/Vocational Degree) for Daughters}
		\label{tab:intdaughtersextension}
		\begin{center}
			\resizebox{1.1\textwidth}{!}{\begin{tabular}{l*{6}{c}}
					\toprule
					&\multicolumn{1}{c}{(1)}         &\multicolumn{1}{c}{(2)}         &\multicolumn{1}{c}{(3)}         &\multicolumn{1}{c}{(4)}         &\multicolumn{1}{c}{(5)}         &\multicolumn{1}{c}{(6)}         \\
					\midrule
					Mother: Ever STEM Profession&      0.0799\sym{**} &      0.0705\sym{**} &      0.0684\sym{*}  &      0.0749\sym{*}  &      0.0745\sym{*}  &      0.0717         \\
					&    (0.0358)         &    (0.0357)         &    (0.0359)         &    (0.0443)         &    (0.0441)         &    (0.0444)         \\
					\addlinespace
					Father: Ever STEM Profession&      0.0585\sym{**} &      0.0542\sym{**} &      0.0541\sym{**} &      0.0359         &      0.0328         &      0.0349         \\
					&    (0.0253)         &    (0.0251)         &    (0.0252)         &    (0.0289)         &    (0.0288)         &    (0.0289)         \\
					\addlinespace
					Mother: Eastern Origin&                     &                     &                     &     -0.2267         &     -0.2192         &     -0.2356         \\
					&                     &                     &                     &    (0.1481)         &    (0.1434)         &    (0.1490)         \\
					\addlinespace
					Father: Eastern Origin&                     &                     &                     &      0.1643         &      0.1563         &      0.1733         \\
					&                     &                     &                     &    (0.1486)         &    (0.1441)         &    (0.1490)         \\
					\addlinespace
					Mother: Ever STEM Profession $\times$ Mother: Eastern Origin&                     &                     &                     &      0.0135         &     -0.0091         &     -0.0183         \\
					&                     &                     &                     &    (0.0766)         &    (0.0766)         &    (0.0781)         \\
					\addlinespace
					Father: Ever STEM Profession $\times$ Father: Eastern Origin&                     &                     &                     &      0.0842         &      0.0821         &      0.0827         \\
					&                     &                     &                     &    (0.0592)         &    (0.0590)         &    (0.0597)         \\
					\midrule
					Person Controls & & X  & X & & X & X \\
					Federal State Controls & & & X & & & X \\
					Rank                &      2         &      5         &     10         &      6         &      9         &     14         \\
					Observations &    812         &    812         &    812         &    812         &    812         &   812         \\
					\bottomrule
			\end{tabular}}
			
			\vspace{2mm}
			
			\parbox{15cm}{
				\linespread{1}\footnotesize Note: \sym{*} \(p<0.10\), \sym{**} \(p<0.05\), \sym{***} \(p<0.01\). Margins (dy/dx) of a Logistic Regression Model. Own calculations based on \cite{SOEP2023}. Standard errors clustered at the household level are given in parentheses. Columns (1) through (3) show estimates of the restricted model based on equation (\ref{eq:specificationrestr}), while columns (4) through (6) show estimates of the unrestricted model based on equation (\ref{eq:specification}). In both the restricted and unrestricted models, person and federal state control are incorporated in a stepwise manner, with the baseline model initially estimated without any controls. In contrast to the previous results, the data include both university degrees and vocational degrees classified as STEM degrees, provided that they align with the established definition of STEM. Both non-STEM university and non-STEM vocational degrees are designated as ``0'' in the regressand.}
			
		\end{center}
	\end{table}
\end{landscape}

\clearpage

\begin{landscape}
	\begin{table}[ht]
		\caption[Intensive Margins for Sons (Extension)]{Intensive Margins (STEM University/Vocational Degree or non-STEM University/Vocational Degree) for Sons}
		\label{tab:intsonsextension}
		\begin{center}
			\resizebox{1.1\textwidth}{!}{\begin{tabular}{l*{6}{c}}
					\toprule
					&\multicolumn{1}{c}{(1)}         &\multicolumn{1}{c}{(2)}         &\multicolumn{1}{c}{(3)}         &\multicolumn{1}{c}{(4)}         &\multicolumn{1}{c}{(5)}         &\multicolumn{1}{c}{(6)}         \\
					\midrule
					Mother: Ever STEM Profession&      0.0048         &      0.0025         &     -0.0061         &     -0.0780         &     -0.0818         &     -0.0872         \\
					&    (0.0721)         &    (0.0719)         &    (0.0720)         &    (0.0972)         &    (0.0970)         &    (0.0965)         \\
					\addlinespace
					Father: Ever STEM Profession&      0.0551         &      0.0566         &      0.0567         &      0.1031\sym{**} &      0.1038\sym{**} &      0.1008\sym{**} \\
					&    (0.0384)         &    (0.0384)         &    (0.0383)         &    (0.0442)         &    (0.0441)         &    (0.0441)         \\
					\addlinespace
					Mother: Eastern Origin&                     &                     &                     &      0.0370         &      0.0506         &      0.0599         \\
					&                     &                     &                     &    (0.1030)         &    (0.1030)         &    (0.1035)         \\
					\addlinespace
					Father: Eastern Origin&                     &                     &                     &      0.0289         &      0.0160         &      0.0059         \\
					&                     &                     &                     &    (0.1065)         &    (0.1065)         &    (0.1166)         \\
					\addlinespace
					Mother: Ever STEM Profession $\times$ Mother: Eastern Origin&                     &                     &                     &      0.1854         &      0.1869         &      0.1927         \\
					&                     &                     &                     &    (0.1474)         &    (0.1470)         &    (0.1471)         \\
					\addlinespace
					Father: Ever STEM Profession $\times$ Father: Eastern Origin&                     &                     &                     &     -0.1817\sym{**} &     -0.1775\sym{**} &     -0.1700\sym{*}  \\
					&                     &                     &                     &    (0.0878)         &    (0.0874)         &    (0.0873)         \\
					\midrule
					Person Controls & & X  & X & & X & X \\
					Federal State Controls & & & X & & & X \\
					Rank                &      2         &      5         &     10         &      6         &      9         &     14         \\
					Observations &    863         &    863         &    863         &    863         &    863         &    863         \\
					\bottomrule
			\end{tabular}}
			
			\vspace{2mm}
			
			\parbox{15cm}{
				\linespread{1}\footnotesize Note: \sym{*} \(p<0.10\), \sym{**} \(p<0.05\), \sym{***} \(p<0.01\). Margins (dy/dx) of a Logistic Regression Model. Own calculations based on \cite{SOEP2023}. Standard errors clustered at the household level are given in parentheses. Columns (1) through (3) show estimates of the restricted model based on equation (\ref{eq:specificationrestr}), while columns (4) through (6) show estimates of the unrestricted model based on equation (\ref{eq:specification}). In both the restricted and unrestricted models, person and federal state control are incorporated in a stepwise manner, with the baseline model initially estimated without any controls. In contrast to the previous results, the data include both university degrees and vocational degrees classified as STEM degrees, provided that they align with the established definition of STEM. Both non-STEM university and non-STEM vocational degrees are designated as ``0'' in the regressand.}
			
		\end{center}
	\end{table}
\end{landscape}

\clearpage

\section{Conclusion}
\label{conclusion}

This paper investigates whether the emphasis on technological progress as a key to the development of socialism was retained as a cultural value by East Germans after the fall of the Iron Curtain and the subsequent Reunification in 1990.

For this purpose, first the development of the German STEM sector following 1990 is analysed in a descriptive manner.
The findings show that Eastern female STEM professionals quickly leave the sector in the first three years following the reunification.
However, it cannot be said whether leaving the sector is primarily a result of the females' desire to leave or whether they were forced out.

In the second part of the analysis, the so-called epidemiological approach coined by \cite{Fernandez2011} is utilised.
It examines whether the decision of adult children to obtain a university degree rather than a vocational degree and, if they do, whether it is a STEM or non-STEM degree, is influenced by the socialisation and occupational history of their parents.
Specifically, the regressors of interest are whether the parents have a STEM background and whether they are from the East.
The findings show that parental origin is more influential for daughters than for sons in the external decision (university vs vocational degree).
Further, the findings indicate that Eastern mothers do not exert a significant negative effect on their sons' decision, but they do on their daughters'.
However, these results may be biased by the lack of information on the parents' occupation before Reunification.

This paper cannot provide any evidence to support the theory that daughters are influenced by their East German STEM mothers to obtain a STEM university degree rather than a non-STEM university degree.
However, the results show that for both daughters and sons, there is a positive effect of the opposite-sex parent on both the extensive and intensive decision (STEM vs non-STEM university degree).
The results do not change significantly when the definition of STEM is adapted to the SED's understanding of what such a term would mean, namely a blue-collar dominated field driven by technological progress.

Thus, the findings uncover a positive influence of opposite-sex parents on the children making a decision whether to enter STEM.
However, further research is needed to determine why this effect is observed.
This paper offers a basic interpretation based on the historical circumstances of reunified Germany in the 1990s, namely that Eastern STEM fathers, who did not experience job loss to the same extent as their female counterparts, encouraged their daughters to pursue STEM as well.
This may be due to a lack of disillusionment on the part of the fathers, who did not experience the changes in the sector in the same way as the females, and so passed on to their daughters the socialist emphasis on STEM associated with the pseudo-gender-egalitarian ideology of the SED.

German policymakers need to understand how East Germans grapple with their socialist past and socialisation.
In order to achieve a truly unified nation state, East German biographies and problems resulting from the in some respects heavy-handed Reunification must be taken into account.
Until this happens, East Germans will continue to feel unemancipated and alienated in Germany, leading to further division in the country.
Nevertheless, nostalgia for the GDR is not helpful in moving forward.
While some results of SED policies may sound progressive for their time, the reasons for these policies and the sheer brutality of the party to keep its citizens in line are often ignored in the literature relevant to this paper.

% either or
%\newpage
\vspace{4cm}

\phantomsection
\addcontentsline{toc}{section}{References}

\makeatletter % prevent newpage within bib-item
\interlinepenalty=10000

\bibliography{../references}
\label{references}

\makeatother

\vspace{-.3cm}

\clearpage

\phantomsection
\addcontentsline{toc}{section}{Appendix}
\mbox{} \vfill \centering \bfseries{{\Large Appendix}} \vfill \mbox{}
\label{appendix}

}% close \fontsize{12pt}{18pt}\selectfont
{\fontsize{11pt}{11pt}\selectfont

\begin{table}[ht]
	\caption{Extented Summary Statistics for Adult Children (Extensive Analysis)}
	\label{tab:descr_summary_epid_ext_appendix}
	\begin{center}
	\resizebox{0.9\textwidth}{!}{\begin{tabular}{llllll}
			\toprule
			\multicolumn{1}{c}{} &
			\multicolumn{1}{r}{Mean} &
			\multicolumn{1}{r}{Std. Dev.} &
			\multicolumn{1}{r}{Min} &
			\multicolumn{1}{r}{Max} &
			\multicolumn{1}{r}{Obs.} \\
			\midrule
			\multicolumn{1}{l}{University Degree instead of Vocational Degree} &
			\multicolumn{1}{r}{0.39} &
			\multicolumn{1}{r}{0.49} &
			\multicolumn{1}{r}{0.00} &
			\multicolumn{1}{r}{1.00} &
			\multicolumn{1}{r}{1,778} \\
			\multicolumn{1}{l}{\hspace{1em}Male} &
			\multicolumn{1}{r}{0.36} &
			\multicolumn{1}{r}{0.48} &
			\multicolumn{1}{r}{0.00} &
			\multicolumn{1}{r}{1.00} &
			\multicolumn{1}{r}{898} \\
			\multicolumn{1}{l}{\hspace{1em}Female} &
			\multicolumn{1}{r}{0.42} &
			\multicolumn{1}{r}{0.49} &
			\multicolumn{1}{r}{0.00} &
			\multicolumn{1}{r}{1.00} &
			\multicolumn{1}{r}{880} \\
			\multicolumn{1}{l}{Age} &
			\multicolumn{1}{r}{26.83} &
			\multicolumn{1}{r}{4.30} &
			\multicolumn{1}{r}{18.00} &
			\multicolumn{1}{r}{37.00} &
			\multicolumn{1}{r}{1,778} \\
			\multicolumn{1}{l}{\hspace{1em}Male} &
			\multicolumn{1}{r}{26.88} &
			\multicolumn{1}{r}{4.22} &
			\multicolumn{1}{r}{18.00} &
			\multicolumn{1}{r}{37.00} &
			\multicolumn{1}{r}{898} \\
			\multicolumn{1}{l}{\hspace{1em}Female} &
			\multicolumn{1}{r}{26.79} &
			\multicolumn{1}{r}{4.37} &
			\multicolumn{1}{r}{19.00} &
			\multicolumn{1}{r}{37.00} &
			\multicolumn{1}{r}{880} \\
			\multicolumn{1}{l}{Spouse/Life Partner} &
			\multicolumn{1}{r}{0.33} &
			\multicolumn{1}{r}{0.47} &
			\multicolumn{1}{r}{0.00} &
			\multicolumn{1}{r}{1.00} &
			\multicolumn{1}{r}{1,778} \\
			\multicolumn{1}{l}{\hspace{1em}Male} &
			\multicolumn{1}{r}{0.27} &
			\multicolumn{1}{r}{0.44} &
			\multicolumn{1}{r}{0.00} &
			\multicolumn{1}{r}{1.00} &
			\multicolumn{1}{r}{898} \\
			\multicolumn{1}{l}{\hspace{1em}Female} &
			\multicolumn{1}{r}{0.39} &
			\multicolumn{1}{r}{0.49} &
			\multicolumn{1}{r}{0.00} &
			\multicolumn{1}{r}{1.00} &
			\multicolumn{1}{r}{880} \\
			\multicolumn{1}{l}{Household Size} &
			\multicolumn{1}{r}{2.78} &
			\multicolumn{1}{r}{1.34} &
			\multicolumn{1}{r}{1.00} &
			\multicolumn{1}{r}{9.00} &
			\multicolumn{1}{r}{1,778} \\
			\multicolumn{1}{l}{\hspace{1em}Male} &
			\multicolumn{1}{r}{2.78} &
			\multicolumn{1}{r}{1.35} &
			\multicolumn{1}{r}{1.00} &
			\multicolumn{1}{r}{7.00} &
			\multicolumn{1}{r}{898} \\
			\multicolumn{1}{l}{\hspace{1em}Female} &
			\multicolumn{1}{r}{2.78} &
			\multicolumn{1}{r}{1.34} &
			\multicolumn{1}{r}{1.00} &
			\multicolumn{1}{r}{9.00} &
			\multicolumn{1}{r}{880} \\
			\multicolumn{1}{l}{Residence in West Germany} &
			\multicolumn{1}{r}{0.74} &
			\multicolumn{1}{r}{0.44} &
			\multicolumn{1}{r}{0.00} &
			\multicolumn{1}{r}{1.00} &
			\multicolumn{1}{r}{1,778} \\
			\multicolumn{1}{l}{\hspace{1em}Male} &
			\multicolumn{1}{r}{0.73} &
			\multicolumn{1}{r}{0.44} &
			\multicolumn{1}{r}{0.00} &
			\multicolumn{1}{r}{1.00} &
			\multicolumn{1}{r}{898} \\
			\multicolumn{1}{l}{\hspace{1em}Female} &
			\multicolumn{1}{r}{0.75} &
			\multicolumn{1}{r}{0.43} &
			\multicolumn{1}{r}{0.00} &
			\multicolumn{1}{r}{1.00} &
			\multicolumn{1}{r}{880} \\
			\multicolumn{1}{l}{Mother: Eastern Origin} &
			\multicolumn{1}{r}{0.29} &
			\multicolumn{1}{r}{0.45} &
			\multicolumn{1}{r}{0.00} &
			\multicolumn{1}{r}{1.00} &
			\multicolumn{1}{r}{1,778} \\
			\multicolumn{1}{l}{\hspace{1em}Male} &
			\multicolumn{1}{r}{0.29} &
			\multicolumn{1}{r}{0.46} &
			\multicolumn{1}{r}{0.00} &
			\multicolumn{1}{r}{1.00} &
			\multicolumn{1}{r}{898} \\
			\multicolumn{1}{l}{\hspace{1em}Female} &
			\multicolumn{1}{r}{0.29} &
			\multicolumn{1}{r}{0.45} &
			\multicolumn{1}{r}{0.00} &
			\multicolumn{1}{r}{1.00} &
			\multicolumn{1}{r}{880} \\
			\multicolumn{1}{l}{Father: Eastern Origin} &
			\multicolumn{1}{r}{0.27} &
			\multicolumn{1}{r}{0.44} &
			\multicolumn{1}{r}{0.00} &
			\multicolumn{1}{r}{1.00} &
			\multicolumn{1}{r}{1,778} \\
			\multicolumn{1}{l}{\hspace{1em}Male} &
			\multicolumn{1}{r}{0.27} &
			\multicolumn{1}{r}{0.45} &
			\multicolumn{1}{r}{0.00} &
			\multicolumn{1}{r}{1.00} &
			\multicolumn{1}{r}{898} \\
			\multicolumn{1}{l}{\hspace{1em}Female} &
			\multicolumn{1}{r}{0.27} &
			\multicolumn{1}{r}{0.44} &
			\multicolumn{1}{r}{0.00} &
			\multicolumn{1}{r}{1.00} &
			\multicolumn{1}{r}{880} \\
			\multicolumn{1}{l}{Mother: Ever STEM Profession} &
			\multicolumn{1}{r}{0.07} &
			\multicolumn{1}{r}{0.25} &
			\multicolumn{1}{r}{0.00} &
			\multicolumn{1}{r}{1.00} &
			\multicolumn{1}{r}{1,778} \\
			\multicolumn{1}{l}{\hspace{1em}Male} &
			\multicolumn{1}{r}{0.06} &
			\multicolumn{1}{r}{0.24} &
			\multicolumn{1}{r}{0.00} &
			\multicolumn{1}{r}{1.00} &
			\multicolumn{1}{r}{898} \\
			\multicolumn{1}{l}{\hspace{1em}Female} &
			\multicolumn{1}{r}{0.08} &
			\multicolumn{1}{r}{0.27} &
			\multicolumn{1}{r}{0.00} &
			\multicolumn{1}{r}{1.00} &
			\multicolumn{1}{r}{880} \\
			\multicolumn{1}{l}{Father: Ever STEM Profession} &
			\multicolumn{1}{r}{0.27} &
			\multicolumn{1}{r}{0.44} &
			\multicolumn{1}{r}{0.00} &
			\multicolumn{1}{r}{1.00} &
			\multicolumn{1}{r}{1,778} \\
			\multicolumn{1}{l}{\hspace{1em}Male} &
			\multicolumn{1}{r}{0.27} &
			\multicolumn{1}{r}{0.44} &
			\multicolumn{1}{r}{0.00} &
			\multicolumn{1}{r}{1.00} &
			\multicolumn{1}{r}{898} \\
			\multicolumn{1}{l}{\hspace{1em}Female} &
			\multicolumn{1}{r}{0.27} &
			\multicolumn{1}{r}{0.44} &
			\multicolumn{1}{r}{0.00} &
			\multicolumn{1}{r}{1.00} &
			\multicolumn{1}{r}{880} \\
			\multicolumn{1}{l}{Indirect Migration Background} &
			\multicolumn{1}{r}{0.01} &
			\multicolumn{1}{r}{0.12} &
			\multicolumn{1}{r}{0.00} &
			\multicolumn{1}{r}{1.00} &
			\multicolumn{1}{r}{1,778} \\
			\multicolumn{1}{l}{\hspace{1em}Male} &
			\multicolumn{1}{r}{0.01} &
			\multicolumn{1}{r}{0.11} &
			\multicolumn{1}{r}{0.00} &
			\multicolumn{1}{r}{1.00} &
			\multicolumn{1}{r}{898} \\
			\multicolumn{1}{l}{\hspace{1em}Female} &
			\multicolumn{1}{r}{0.01} &
			\multicolumn{1}{r}{0.12} &
			\multicolumn{1}{r}{0.00} &
			\multicolumn{1}{r}{1.00} &
			\multicolumn{1}{r}{880} \\
			\multicolumn{1}{l}{Unemployment Rate (Fed. State)} &
			\multicolumn{1}{r}{9.58} &
			\multicolumn{1}{r}{4.67} &
			\multicolumn{1}{r}{2.90} &
			\multicolumn{1}{r}{20.49} &
			\multicolumn{1}{r}{1,778} \\
			\multicolumn{1}{l}{\hspace{1em}Male} &
			\multicolumn{1}{r}{9.65} &
			\multicolumn{1}{r}{4.65} &
			\multicolumn{1}{r}{3.62} &
			\multicolumn{1}{r}{20.49} &
			\multicolumn{1}{r}{898} \\
			\multicolumn{1}{l}{\hspace{1em}Female} &
			\multicolumn{1}{r}{9.52} &
			\multicolumn{1}{r}{4.70} &
			\multicolumn{1}{r}{2.90} &
			\multicolumn{1}{r}{20.49} &
			\multicolumn{1}{r}{880} \\
			\multicolumn{1}{l}{GDP in Thousand of Euros (Fed. State)} &
			\multicolumn{1}{r}{2.84e+08} &
			\multicolumn{1}{r}{1.89e+08} &
			\multicolumn{1}{r}{2.31e+07} &
			\multicolumn{1}{r}{7.03e+08} &
			\multicolumn{1}{r}{1,778} \\
			\multicolumn{1}{l}{\hspace{1em}Male} &
			\multicolumn{1}{r}{2.75e+08} &
			\multicolumn{1}{r}{1.85e+08} &
			\multicolumn{1}{r}{2.31e+07} &
			\multicolumn{1}{r}{6.53e+08 } &
			\multicolumn{1}{r}{898} \\
			\multicolumn{1}{l}{\hspace{1em}Female} &
			\multicolumn{1}{r}{2.93e+08} &
			\multicolumn{1}{r}{1.92e+08} &
			\multicolumn{1}{r}{2.37e+07} &
			\multicolumn{1}{r}{7.03e+08} &
			\multicolumn{1}{r}{880} \\
			\multicolumn{1}{l}{Population Density (Fed. State)} &
			\multicolumn{1}{r}{417.77} &
			\multicolumn{1}{r}{713.43} &
			\multicolumn{1}{r}{68.78} &
			\multicolumn{1}{r}{3,891.32} &
			\multicolumn{1}{r}{1,778} \\
			\multicolumn{1}{l}{\hspace{1em}Male} &
			\multicolumn{1}{r}{414.60} &
			\multicolumn{1}{r}{707.50} &
			\multicolumn{1}{r}{68.89} &
			\multicolumn{1}{r}{3,891.32} &
			\multicolumn{1}{r}{898} \\
			\multicolumn{1}{l}{\hspace{1em}Female} &
			\multicolumn{1}{r}{421.00} &
			\multicolumn{1}{r}{719.82} &
			\multicolumn{1}{r}{68.78} &
			\multicolumn{1}{r}{3,891.32} &
			\multicolumn{1}{r}{880} \\
			\multicolumn{1}{l}{Net Commuter Traffic (Fed. State)} &
			\multicolumn{1}{r}{-1.23} &
			\multicolumn{1}{r}{7.07} &
			\multicolumn{1}{r}{-18.87} &
			\multicolumn{1}{r}{30.04} &
			\multicolumn{1}{r}{1,778} \\
			\multicolumn{1}{l}{\hspace{1em}Male} &
			\multicolumn{1}{r}{-1.28} &
			\multicolumn{1}{r}{7.17} &
			\multicolumn{1}{r}{-18.87} &
			\multicolumn{1}{r}{29.54} &
			\multicolumn{1}{r}{898} \\
			\multicolumn{1}{l}{\hspace{1em}Female} &
			\multicolumn{1}{r}{-1.19} &
			\multicolumn{1}{r}{6.96} &
			\multicolumn{1}{r}{-18.87} &
			\multicolumn{1}{r}{30.04} &
			\multicolumn{1}{r}{880} \\
			\multicolumn{1}{l}{Students in First Semester as Share of all Students (Fed. State)} &
			\multicolumn{1}{r}{16.95} &
			\multicolumn{1}{r}{2.40} &
			\multicolumn{1}{r}{11.06} &
			\multicolumn{1}{r}{22.70} &
			\multicolumn{1}{r}{1,778} \\
			\multicolumn{1}{l}{\hspace{1em}Male} &
			\multicolumn{1}{r}{16.94} &
			\multicolumn{1}{r}{2.37} &
			\multicolumn{1}{r}{11.06} &
			\multicolumn{1}{r}{22.70} &
			\multicolumn{1}{r}{898} \\
			\multicolumn{1}{l}{\hspace{1em}Female} &
			\multicolumn{1}{r}{16.96} &
			\multicolumn{1}{r}{2.43} &
			\multicolumn{1}{r}{11.06} &
			\multicolumn{1}{r}{22.70} &
			\multicolumn{1}{r}{880} \\
			\bottomrule
	\end{tabular}}

\vspace{2mm}

\parbox{15cm}{
	\linespread{1}\footnotesize Note: Own calculations based on \cite{SOEP2023}. For more information on the estimation sample, see the footnotes to table \ref{tab:descr_summary_epid_ext}.}

\end{center}
\end{table}

\begin{table}[ht]
	\caption{Extented Summary Statistics for Adult Children (Intensive Analysis)}
	\label{tab:descr_summary_epid_int_appendix}
	\begin{center}
	\resizebox{0.9\textwidth}{!}{\begin{tabular}{llllll}
			\toprule
			\multicolumn{1}{c}{} &
			\multicolumn{1}{r}{Mean} &
			\multicolumn{1}{r}{Std. Dev.} &
			\multicolumn{1}{r}{Min} &
			\multicolumn{1}{r}{Max} &
			\multicolumn{1}{r}{Obs.} \\
			\midrule
			\multicolumn{1}{l}{STEM University Degree instead of non-STEM Degree} &
			\multicolumn{1}{r}{0.36} &
			\multicolumn{1}{r}{0.48} &
			\multicolumn{1}{r}{0.00} &
			\multicolumn{1}{r}{1.00} &
			\multicolumn{1}{r}{585} \\
			\multicolumn{1}{l}{\hspace{1em}Male} &
			\multicolumn{1}{r}{0.47} &
			\multicolumn{1}{r}{0.50} &
			\multicolumn{1}{r}{0.00} &
			\multicolumn{1}{r}{1.00} &
			\multicolumn{1}{r}{287} \\
			\multicolumn{1}{l}{\hspace{1em}Female} &
			\multicolumn{1}{r}{0.26} &
			\multicolumn{1}{r}{0.44} &
			\multicolumn{1}{r}{0.00} &
			\multicolumn{1}{r}{1.00} &
			\multicolumn{1}{r}{298} \\
			\multicolumn{1}{l}{Age} &
			\multicolumn{1}{r}{28.68} &
			\multicolumn{1}{r}{3.94} &
			\multicolumn{1}{r}{20.00} &
			\multicolumn{1}{r}{37.00} &
			\multicolumn{1}{r}{585} \\
			\multicolumn{1}{l}{\hspace{1em}Male} &
			\multicolumn{1}{r}{28.80} &
			\multicolumn{1}{r}{3.92} &
			\multicolumn{1}{r}{21.00} &
			\multicolumn{1}{r}{37.00} &
			\multicolumn{1}{r}{287} \\
			\multicolumn{1}{l}{\hspace{1em}Female} &
			\multicolumn{1}{r}{28.56} &
			\multicolumn{1}{r}{3.97} &
			\multicolumn{1}{r}{20.00} &
			\multicolumn{1}{r}{37.00} &
			\multicolumn{1}{r}{298} \\
			\multicolumn{1}{l}{Spouse/Life Partner} &
			\multicolumn{1}{r}{0.43} &
			\multicolumn{1}{r}{0.50} &
			\multicolumn{1}{r}{0.00} &
			\multicolumn{1}{r}{1.00} &
			\multicolumn{1}{r}{585} \\
			\multicolumn{1}{l}{\hspace{1em}Male} &
			\multicolumn{1}{r}{0.37} &
			\multicolumn{1}{r}{0.48} &
			\multicolumn{1}{r}{0.00} &
			\multicolumn{1}{r}{1.00} &
			\multicolumn{1}{r}{287} \\
			\multicolumn{1}{l}{\hspace{1em}Female} &
			\multicolumn{1}{r}{0.50} &
			\multicolumn{1}{r}{0.50} &
			\multicolumn{1}{r}{0.00} &
			\multicolumn{1}{r}{1.00} &
			\multicolumn{1}{r}{298} \\
			\multicolumn{1}{l}{Household Size} &
			\multicolumn{1}{r}{2.45} &
			\multicolumn{1}{r}{1.30} &
			\multicolumn{1}{r}{1.00} &
			\multicolumn{1}{r}{6.00} &
			\multicolumn{1}{r}{585} \\
			\multicolumn{1}{l}{\hspace{1em}Male} &
			\multicolumn{1}{r}{2.39} &
			\multicolumn{1}{r}{1.33} &
			\multicolumn{1}{r}{1.00} &
			\multicolumn{1}{r}{6.00} &
			\multicolumn{1}{r}{287} \\
			\multicolumn{1}{l}{\hspace{1em}Female} &
			\multicolumn{1}{r}{2.50} &
			\multicolumn{1}{r}{1.28} &
			\multicolumn{1}{r}{1.00} &
			\multicolumn{1}{r}{6.00} &
			\multicolumn{1}{r}{298} \\
			\multicolumn{1}{l}{Residence in West Germany} &
			\multicolumn{1}{r}{0.77} &
			\multicolumn{1}{r}{0.42} &
			\multicolumn{1}{r}{0.00} &
			\multicolumn{1}{r}{1.00} &
			\multicolumn{1}{r}{585} \\
			\multicolumn{1}{l}{\hspace{1em}Male} &
			\multicolumn{1}{r}{0.75} &
			\multicolumn{1}{r}{0.43} &
			\multicolumn{1}{r}{0.00} &
			\multicolumn{1}{r}{1.00} &
			\multicolumn{1}{r}{287} \\
			\multicolumn{1}{l}{\hspace{1em}Female} &
			\multicolumn{1}{r}{0.80} &
			\multicolumn{1}{r}{0.40} &
			\multicolumn{1}{r}{0.00} &
			\multicolumn{1}{r}{1.00} &
			\multicolumn{1}{r}{298} \\
			\multicolumn{1}{l}{Mother: Eastern Origin} &
			\multicolumn{1}{r}{0.23} &
			\multicolumn{1}{r}{0.42} &
			\multicolumn{1}{r}{0.00} &
			\multicolumn{1}{r}{1.00} &
			\multicolumn{1}{r}{585} \\
			\multicolumn{1}{l}{\hspace{1em}Male} &
			\multicolumn{1}{r}{0.25} &
			\multicolumn{1}{r}{0.43} &
			\multicolumn{1}{r}{0.00} &
			\multicolumn{1}{r}{1.00} &
			\multicolumn{1}{r}{287} \\
			\multicolumn{1}{l}{\hspace{1em}Female} &
			\multicolumn{1}{r}{0.21} &
			\multicolumn{1}{r}{0.41} &
			\multicolumn{1}{r}{0.00} &
			\multicolumn{1}{r}{1.00} &
			\multicolumn{1}{r}{298} \\
			\multicolumn{1}{l}{Father: Eastern Origin} &
			\multicolumn{1}{r}{0.22} &
			\multicolumn{1}{r}{0.41} &
			\multicolumn{1}{r}{0.00} &
			\multicolumn{1}{r}{1.00} &
			\multicolumn{1}{r}{585} \\
			\multicolumn{1}{l}{\hspace{1em}Male} &
			\multicolumn{1}{r}{0.23} &
			\multicolumn{1}{r}{0.42} &
			\multicolumn{1}{r}{0.00} &
			\multicolumn{1}{r}{1.00} &
			\multicolumn{1}{r}{287} \\
			\multicolumn{1}{l}{\hspace{1em}Female} &
			\multicolumn{1}{r}{0.20} &
			\multicolumn{1}{r}{0.40} &
			\multicolumn{1}{r}{0.00} &
			\multicolumn{1}{r}{1.00} &
			\multicolumn{1}{r}{298} \\
			\multicolumn{1}{l}{Mother: Ever STEM Profession} &
			\multicolumn{1}{r}{0.10} &
			\multicolumn{1}{r}{0.30} &
			\multicolumn{1}{r}{0.00} &
			\multicolumn{1}{r}{1.00} &
			\multicolumn{1}{r}{585} \\
			\multicolumn{1}{l}{\hspace{1em}Male} &
			\multicolumn{1}{r}{0.08} &
			\multicolumn{1}{r}{0.28} &
			\multicolumn{1}{r}{0.00} &
			\multicolumn{1}{r}{1.00} &
			\multicolumn{1}{r}{287} \\
			\multicolumn{1}{l}{\hspace{1em}Female} &
			\multicolumn{1}{r}{0.12} &
			\multicolumn{1}{r}{0.33} &
			\multicolumn{1}{r}{0.00} &
			\multicolumn{1}{r}{1.00} &
			\multicolumn{1}{r}{298} \\
			\multicolumn{1}{l}{Father: Ever STEM Profession} &
			\multicolumn{1}{r}{0.35} &
			\multicolumn{1}{r}{0.48} &
			\multicolumn{1}{r}{0.00} &
			\multicolumn{1}{r}{1.00} &
			\multicolumn{1}{r}{585} \\
			\multicolumn{1}{l}{\hspace{1em}Male} &
			\multicolumn{1}{r}{0.35} &
			\multicolumn{1}{r}{0.48} &
			\multicolumn{1}{r}{0.00} &
			\multicolumn{1}{r}{1.00} &
			\multicolumn{1}{r}{287} \\
			\multicolumn{1}{l}{\hspace{1em}Female} &
			\multicolumn{1}{r}{0.34} &
			\multicolumn{1}{r}{0.47} &
			\multicolumn{1}{r}{0.00} &
			\multicolumn{1}{r}{1.00} &
			\multicolumn{1}{r}{298} \\
			\multicolumn{1}{l}{Indirect Migration Background} &
			\multicolumn{1}{r}{0.02} &
			\multicolumn{1}{r}{0.12} &
			\multicolumn{1}{r}{0.00} &
			\multicolumn{1}{r}{1.00} &
			\multicolumn{1}{r}{585} \\
			\multicolumn{1}{l}{\hspace{1em}Male} &
			\multicolumn{1}{r}{0.01} &
			\multicolumn{1}{r}{0.10} &
			\multicolumn{1}{r}{0.00} &
			\multicolumn{1}{r}{1.00} &
			\multicolumn{1}{r}{287} \\
			\multicolumn{1}{l}{\hspace{1em}Female} &
			\multicolumn{1}{r}{0.02} &
			\multicolumn{1}{r}{0.14} &
			\multicolumn{1}{r}{0.00} &
			\multicolumn{1}{r}{1.00} &
			\multicolumn{1}{r}{298} \\
			\multicolumn{1}{l}{Unemployment Rate (Fed. State)} &
			\multicolumn{1}{r}{9.22} &
			\multicolumn{1}{r}{4.38} &
			\multicolumn{1}{r}{2.90} &
			\multicolumn{1}{r}{20.49} &
			\multicolumn{1}{r}{585} \\
			\multicolumn{1}{l}{\hspace{1em}Male} &
			\multicolumn{1}{r}{9.29} &
			\multicolumn{1}{r}{4.37} &
			\multicolumn{1}{r}{3.62} &
			\multicolumn{1}{r}{20.47} &
			\multicolumn{1}{r}{287} \\
			\multicolumn{1}{l}{\hspace{1em}Female} &
			\multicolumn{1}{r}{9.16} &
			\multicolumn{1}{r}{4.39} &
			\multicolumn{1}{r}{2.90} &
			\multicolumn{1}{r}{20.49} &
			\multicolumn{1}{r}{298} \\
			\multicolumn{1}{l}{GDP in Thousand of Euros (Fed. State)} &
			\multicolumn{1}{r}{3.02e+08} &
			\multicolumn{1}{r}{1.92e+08  } &
			\multicolumn{1}{r}{2.31e+07} &
			\multicolumn{1}{r}{7.03e+08} &
			\multicolumn{1}{r}{585} \\
			\multicolumn{1}{l}{\hspace{1em}Male} &
			\multicolumn{1}{r}{2.89e+08} &
			\multicolumn{1}{r}{1.85e+08} &
			\multicolumn{1}{r}{2.31e+07} &
			\multicolumn{1}{r}{6.37e+08} &
			\multicolumn{1}{r}{287} \\
			\multicolumn{1}{l}{\hspace{1em}Female} &
			\multicolumn{1}{r}{3.14e+08} &
			\multicolumn{1}{r}{1.99e+08} &
			\multicolumn{1}{r}{2.37e+07} &
			\multicolumn{1}{r}{7.03e+08  } &
			\multicolumn{1}{r}{298} \\
			\multicolumn{1}{l}{Population Density (Fed. State)} &
			\multicolumn{1}{r}{521.20} &
			\multicolumn{1}{r}{853.30} &
			\multicolumn{1}{r}{68.89} &
			\multicolumn{1}{r}{3,891.32} &
			\multicolumn{1}{r}{585} \\
			\multicolumn{1}{l}{\hspace{1em}Male} &
			\multicolumn{1}{r}{546.91} &
			\multicolumn{1}{r}{905.91} &
			\multicolumn{1}{r}{68.89} &
			\multicolumn{1}{r}{3,891.32} &
			\multicolumn{1}{r}{287} \\
			\multicolumn{1}{l}{\hspace{1em}Female} &
			\multicolumn{1}{r}{496.44} &
			\multicolumn{1}{r}{800.13} &
			\multicolumn{1}{r}{68.95} &
			\multicolumn{1}{r}{3,880.87} &
			\multicolumn{1}{r}{298} \\
			\multicolumn{1}{l}{Net Commuter Traffic (Fed. State)} &
			\multicolumn{1}{r}{0.17} &
			\multicolumn{1}{r}{7.26} &
			\multicolumn{1}{r}{-18.45} &
			\multicolumn{1}{r}{30.04} &
			\multicolumn{1}{r}{585} \\
			\multicolumn{1}{l}{\hspace{1em}Male} &
			\multicolumn{1}{r}{0.27} &
			\multicolumn{1}{r}{7.25} &
			\multicolumn{1}{r}{-17.86} &
			\multicolumn{1}{r}{29.54} &
			\multicolumn{1}{r}{287} \\
			\multicolumn{1}{l}{\hspace{1em}Female} &
			\multicolumn{1}{r}{0.08} &
			\multicolumn{1}{r}{7.29} &
			\multicolumn{1}{r}{-18.45} &
			\multicolumn{1}{r}{30.04} &
			\multicolumn{1}{r}{298} \\
			\multicolumn{1}{l}{Students in First Semester as Share of all Students (Fed. State)} &
			\multicolumn{1}{r}{16.77} &
			\multicolumn{1}{r}{2.40} &
			\multicolumn{1}{r}{11.06} &
			\multicolumn{1}{r}{22.70} &
			\multicolumn{1}{r}{585} \\
			\multicolumn{1}{l}{\hspace{1em}Male} &
			\multicolumn{1}{r}{16.73} &
			\multicolumn{1}{r}{2.42} &
			\multicolumn{1}{r}{11.59} &
			\multicolumn{1}{r}{22.70} &
			\multicolumn{1}{r}{287} \\
			\multicolumn{1}{l}{\hspace{1em}Female} &
			\multicolumn{1}{r}{16.81} &
			\multicolumn{1}{r}{2.39} &
			\multicolumn{1}{r}{11.06} &
			\multicolumn{1}{r}{22.70} &
			\multicolumn{1}{r}{298} \\
			\bottomrule
	\end{tabular}}

	\vspace{2mm}
	
	\parbox{15cm}{
		\linespread{1}\footnotesize Note: Own calculations based on \cite{SOEP2023}. For more information on the estimation sample, see the footnotes to table \ref{tab:descr_summary_epid_int}.}
	
\end{center}
\end{table}

\clearpage

\phantomsection
\addcontentsline{toc}{section}{Statutory Declaration}
\section*{Statutory Declaration}
\label{declarations}

I hereby declare that I have written this thesis independently and without the use of sources and aids other than those specified. I have not used the services of any agency providing specimen, model, or ghostwritten work in the preparation of this submitted work. This also includes the use of AI-generated texts or services such as ChatGPT. Sentences or parts of sentences quoted literally are marked as such; other references with regard to the statement and scope are indicated by full details of the publications concerned. The thesis in the same or similar form has not been submitted to any examination body and has not been published. This thesis was not yet, even in part, used in another examination or as a course performance.

\vspace{1.5cm}

\includegraphics[height = 25mm]{unterschrift_schuett.pdf}

\noindent Potsdam, \today{}

} % close \fontsize{11pt}{11pt}

\end{document}
