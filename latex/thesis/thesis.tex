\documentclass[a4paper, oneside, hyperfootnotes = false]{article}

\usepackage[utf8]{inputenc}
\usepackage[T1]{fontenc}
\usepackage[UKenglish]{babel}
\usepackage{amsmath}
\usepackage{anyfontsize}
\usepackage{bookmark}
\usepackage{booktabs}
\usepackage{caption}
\usepackage{fancyvrb}
\usepackage[left = 3cm, right = 2.5cm, top = 2.5cm, bottom = 2.5cm]{geometry}
\usepackage{graphicx}
\usepackage{hyperref}
\usepackage{multirow}
\usepackage[round]{natbib}
\usepackage{pdflscape}
\usepackage{svg}
\usepackage{tocloft}

\bibliographystyle{apalike}

% line spread
\linespread{1.5}

% remove ugly boxes around hyperlinks
\hypersetup{
	pdfborder = {0 0 0}
}

% vertical space between rule and main text
\addtolength{\skip\footins}{8pt}
% vertical space between footnotes
\addtolength{\footnotesep}{5pt}

% path to figures
\graphicspath{{../../stata/figures/pdf_tex/}}

\makeatletter
\def\input@path{{../../stata/figures/pdf_tex/}}
\makeatother

% for tables
\def\sym#1{\ifmmode^{#1}\else\(^{#1}\)\fi}

% title page ------
\title{Putting the GDR's Legacy Effect under the Microscope: \linebreak Eastern Female STEM Professionals in Reunified Germany.}

\author{\href{mailto:johannes.schuett@fu-berlin.de}{Schütt, Johannes} \\
5574549}

\date{\today{}}
% title page end ---

\begin{document}

{\fontsize{12pt}{18pt}\selectfont

\maketitle

\thispagestyle{empty}
\pagenumbering{Alph}

\vspace{1.5cm}

\noindent\begin{tabular}{l}
	A Master Thesis presented to \\
	The Faculty of Economics at the Freie Universität Berlin \\
    [\normalbaselineskip]
    In Partial Fulfillment of Requirements for \\
    The Degree of Master of Science \\
    [\normalbaselineskip]
    Module Examination according to 0353cE1.1P at \\
    The Chair of Empirical Economics and Gender under \\
    The Supervision of Prof. Natalia Danzer, Ph.D. \\
    [\normalbaselineskip]
    Summer Term 2024
\end{tabular}

\vspace{3cm}

\begin{center}
    \includegraphics[width=0.4\textwidth, angle=0]{fu_logo.pdf}
\end{center}

\newpage

\pagestyle{plain}
\pagenumbering{roman}

{\fontsize{12pt}{15pt}\selectfont

\tableofcontents
\newpage

\phantomsection
\addcontentsline{toc}{section}{List of Figures}
\listoffigures

\vspace{2cm}

\phantomsection
\addcontentsline{toc}{section}{List of Tables}
\listoftables

%\newpage

\vspace{2cm}

\phantomsection
\addcontentsline{toc}{section}{List of Abbreviations}
\section*{List of Abbreviations}
\noindent\begin{tabular}{@{}ll}
	DFD & \emph{Demokratischer Frauenbund Deutschlands} \\
    DIW & German Institute for Economic Research \\
    FLFP & Female Labour Force Participation \\
    FRG & Federal Republic of Germany \\
    GDR & German Democratic Republic \\
    GSOEP & German Socio-Economic Panel \\
    ISCO & International Standard Classification of Occupations \\
    PISA & Programme for International Student Assessment \\
    pp & percentage points \\
    SED & \emph{Sozialistische Einheitspartei Deutschlands} \\
    STEM & Science, Technology, Engineering, and Mathematics
\end{tabular}

\newpage

\begin{center}
{\large\bfseries Abstract}

\vspace{5mm}

\parbox{400pt}{
    \emph{\noindent Lorem ipsum dolor sit amet, consetetur sadipscing elitr, sed diam nonumy eirmod tempor invidunt ut labore et dolore magna aliquyam erat, sed diam voluptua. At vero eos et accusam et justo duo dolores et ea rebum. Stet clita kasd gubergren, no sea takimata sanctus est Lorem ipsum dolor sit amet. Lorem ipsum dolor sit amet, consetetur sadipscing elitr, sed diam nonumy eirmod tempor invidunt ut labore et dolore magna aliquyam erat, sed diam voluptua. At vero eos et accusam et justo duo dolores et ea rebum. Stet clita kasd gubergren, no sea takimata sanctus est Lorem ipsum dolor sit amet.
    }
}

\end{center}

}% end \fontsize{12pt}{15pt}\selectfont

\vspace{5mm}

\pagenumbering{arabic}
\renewcommand\thesection{\arabic{section}}

\section{Introduction}
\label{intro}

\section{Historical Background}
\label{background}

% hour zero
During the Cold War, Germany was divided into two nation states: the socialist German Democratic Republic, which together with the other Warsaw Pact states was allied with the Soviet Union, and the Federal Republic of Germany, which together with the other NATO states was allied with the United States of America.
Since their foundation in 1949, the two countries had grown considerably apart in political and economic terms:
While the GDR operated according to a command economy based on the principles of Marxism-Leninism, the FRG employed a social market economy.
In the first year of the existence of these two states, female labour force participation (FLFP for short) was 40\% \citep[Chapter~2]{Menschik1974} for both.
However, this high share of females in the labour force can be explained by the large number of males of working age who were absent due to the war and its aftermath, and even when the husband returned to his family, one income was not enough to support them.
Following the \emph{Wirtschaftswunder} of the 1950s, FLFP declined in the FRG, while females in the GDR never returned in significant numbers to the conservative role of a housewife \citep{Ostner1991, Rosenfeld2004}.

% ulbricht's new economic system
In 1963, Walter Ulbricht, the First Secretary of the Socialist Unity Party (SED for short), sought to reform the struggling East German economic regime, which was essentially the Stalinist model that he instated in the first place. He attempted to introduce limited free-market elements, such as a more flexible price system and a certain level of autonomy granted to factory managers \citep[Chapter~4]{Grieder1999}.
This policy shift included the ideological goal of technocratising the general labour force:
At the third party conference in 1956, Ulbricht posited that workers must ``[...] acquire the most progressive insights offered by technology and science in order to increase their labour productivity [...]'' \citep{Ulbricht1956, Sanderson1981}.

% technocratisation
The GDR -- as a ``Worker and Peasant State'' -- had a labour force that largely consisted of blue-collar workers \citep{DDRJahrbuch1957}.
Therefore, every labourer should understand the purpose of every single cog in the machine they operate.
Beyond that, the SED stimulated the scientific and technological sectors in hopes of innovation that could directly translate into higher production power \citep{Hoegselius2009}.
The overarching ambition was to nurture the most pioneering scientific community in the world, which stood in direct service of socialism.
This ambition led to significant movement in these work sectors in the following decade.
Figure \ref{fig:mayer} uses data from the East German Life History Study to approximate how the share of STEM professionals developed over the duration of the GDR's existence.
For the purpose of the East German Life History Study, 2,331 East German individuals were interviewed retrospectively between September 1991 and October 1992 about their lives in the GDR.
The data cover four birth cohorts (1929--1931, 1939--1941, 1951--1953, and 1959--1961) and provide information on the family of origin, housing and household history, qualifications and educational phases, occupational histories, and further areas of the personal life.
The figure displays individuals who took up a STEM occupation as a share of all individuals who took up a new occupation in the respective decade.
As anticipated, the proportion of individuals newly employed in STEM occupations was relatively low prior to the 1960s, as the majority of workers took up occupations that served the purpose of rebuilding the critical infrastructure that had been destroyed during the war.
However, possibly as a result of the ideological shift regarding the sciences in the 1960s, there has been an increase in the share of new male STEM workers, while females stay at a lower share.
Over the last two decades of the existence of the GDR, the male and female shares have converged at approximately 10\%.

% dfd
A number of factors may be identified as contributing to the convergence observed in these shares.
First, the socialist ideology included the concept of removing traditional gender associations from specific occupations.
The law on maternal and child protection and women's rights of 1950 states that ``[...] women should be given more opportunities to work in industry, [...] and public estates, in all organs of state administration, [...] and other institutions of the German Democratic Republic.
The work of women in production should not be limited to the traditional female professions, but should extend to all branches of production, especially the electrical industry, optics, mechanical engineering, precision mechanics, the wood and furniture industry, the shoe industry, and the construction and graphic arts trades.'' \citep[§19.1]{GBl1950}.
This law created the necessity for the state to implement measures to actually facilitate the inclusion of females in technical professions.
It was recognised that there was a disparity in the starting points of females and males in this regard, given that historically, these roles have been predominantly occupied by males.
Second, the school curriculum for basic education was identical for both female and male students, placing a significant emphasis on natural sciences and mathematics \citep{FuchsSchuendeln2016, Campa2019, Davoli2021}.
This was not the case in the FRG, where school curricula for females and males differed until the 1970s.
Third, the state took steps to encourage females to pursue careers in science by establishing the ``Democratic Women's Association of Germany'' (DFD for short).
This association was founded during the short period of the Soviet occupation zone and constituted one of the biggest mass organisations within the GDR.
The organisation's stated objective was to secure equal rights for females in education and employment.
The SED viewed the DFD as an instrument for mobilising females into the labour force, which was in dire need of additional workers throughout the entire existence of the GDR.
In 1962, the Ministerial Council approved a programme designed to encourage females to work in technical occupations, acknowleding the underrepresentation of females in this sector \citep{GBl1962}.
In the area of school education, the SED set itself the goal of improving the quality of science lessons at the extended secondary schools and increasing the proportion of girls in the science stream, thereby fuelling their interest in science.
Also, a special focus on science and technology in grades 5 to 8 of the ten-grade polytechnic secondary school is intended for the same motivation.
When promoting university studies in STEM fields, females in particular should be targeted.
For certain degree programmes, a quota for females must be met for new enrolments.
Evening courses are to be offered for working females and mothers who are unable to study full-time.
In this context, special ``women's classes'' are to be set up.
Furthermore, the legislation sets out a list of occupations that are required to meet a specified quota for female representation.
This list includes, for instance, electromechanics, telecommunication mechanics, skilled chemical workers, and more.

% further steps
The SED implemented additional measures to facilitate female participation in the workforce, including the provision of childcare facilities, the introduction of a monthly ``household day'' to acknowledge the ``double burden'' of working females, and the legalisation of abortion \citep{Budde1999}.
The issue of female participation in the workforce has always been a practical and ideological concern for the party.
Consequently, the working mother has been a consistent focus of state propaganda:
Female tractor drivers and engineers were depicted in East German propaganda films and novels, discussing their enthusiasm for advancing socialism.

Despite all the lip service paid by the SED, hardly any females reached leading positions in the factories and organisations in which they worked \citep{Ross2017, Frauenreport1990}.

\section{Data and Empirical Approach}
\label{dataemp}

\subsection{GSOEP}
\label{gsoep}

% gsoep
In order to examine the developments that have occurred in the STEM sector following the Reunification, this paper employs data from the German Socio-Economic Panel (GSOEP for short), a nationally representative panel study initiated in 1984 and maintained by the German Institute for Economic Research (DIW for short) \citep{Siegers2022}.
It frequently serves as a basis for research on the Reunification and East-West differences in Germany across different domains \citep{Petrunyk2016, Bird1994, Hadjar2010}.

% sample for descriptive analysis
In order to provide a descriptive analysis in the first part of this paper, GSOEP data from the survey years 1990–1999 has been utilised.
This sample comprises individuals of working age (16–65 years old) with information on their employment status, ISCO-88 occupation code, and location in 1989 (the year preceding Reunification) available.
All individuals were either born in the GDR or the FRG.
The International Standard Classification of Occupations (ISCO for short) is an internationally used classification standard for jobs \citep{Elias1997}.
It is developed and maintained by the International Labour Organisation (ILB for short).
The ISCO is designed to provide a structured framework for organising jobs into clearly defined groups based on the tasks and duties involved.
The location in 1989 (either East or West Germany) is used to identify former GDR citizens in the sample.

% sample for epidemiological approach
The estimation sample used for the epidemiological analysis of the transmission of cultural values is derived from the previous sample as follows:
Every individual from the sample used in the descriptive analysis who has one or more children who themselves participate in the GSOEP is linked to them.
The estimation sample comprises all adult children who have both parents sucessfully linked to them.
They were born after 1983, i.e. they were 6 years or younger at the time of Reunification, and have valid information on their field of tertiary education.

\subsection{Epidemiological Approach}
\label{epid}

In order to analyse the transmission of the cultural emphasis on STEM in the GDR, this paper employs the epidemiological approach formulated by \cite{Fernandez2011}.
Fernández argues that culture can have a significant influence on various economic outcomes, such as savings rates, fertility rates, and FLFP.
She defines the epidemiological approach as an attempt to separate culture from environment and, more precisely, as a method to study different immigrant groups within the same host country.
Thus, the outcomes of individuals whose cultures differ but who live in the same economic and institutional setting can be studied.

Fernández brings forward a medical example from the field of epidemiology (hence the name) to illustrate the idea that drives the approach:
Suppose the incidence of heart disease differs significantly between two countries (the source and the host country).
If the incidence of heart disease in immigrants converges to that of natives of the host country, the difference between the two countries is unlikely to be driven by genetics but rather has environmental causes.
Crucially, Fernández points out that failure of convergence does not imply the opposite.

Translating this example into the economic sphere, the objective is to isolate the variation in outcomes that is caused by culture versus the variation that is caused by economic and institutional factors.

Fernández provides a minimal example of how the approach can be executed with a regression model.
For this purpose, suppose that there is data on individuals who live in one given country but whose parents were born in some other country $c$.
Suppose then that we want to estimate the probability that individual $i$ takes some action $y_{ic}$ modelled as

\vspace{-8mm}

\begin{equation}
	\label{eq:fernandezexercise}
	y_{ic} = \beta_{0} + \beta_{1}X_{i} + \beta_{2}Y_{c} + \epsilon_{i}
\end{equation}

\noindent with $X_{i}$ being a vector of individual characteristics and $Y_{c}$ being a proxy for culture in country of ancestry $c$.
The regressor of interest is the latter, which aims to capture variation in $y_{ic}$ caused by cultural influence.
Culture can be proxied by employing a country-of-ancestry dummy fo $Y_{c}$.
However, this approach may give rise to several issues:
First, a dummy captures a broad, undifferentiated effect and does not vary for specific cultural traits that might vary within a country of ancestry.
Second, the economic and cultural influences of the country of ancestry are collapsed into one dummy.
Therefore, additionally controlling for the macroeconomic properties of the country of ancestry might help to obtain a more nuanced picture instead of an average effect.
Third, the country-of-ancestry dummy does not vary over time.
While a country might have significantly changed in several dimensions since the departure of the parents of the individual, this change cannot be captured in a time-constant variable.

% application on former gdr citizens in reunified germany (simple model)
Before discussing on how to specify a regression model for the purpose of this paper, it is first  beneficial to discuss the epidemiological approach in further detail, following the description of Fernández, but modifying her theoretical model of a female's extensive work decision to fit the subject of this paper. 


% studies that use the approach

\subsection{Econometric Specification}
\label{specification}

\section{Results}
\label{results}

\subsection{Descriptive Analysis}
\label{descriptives}

\subsection{Legacy Effect on Educational Choices}
\label{educational}

\section{Robustness: Different Model Specifications}
\label{robustness}

\section{Extension: Comparing the Blocs}
\label{Extension}

\section{Conclusion}
\label{conclusion}

% either or
%\newpage
%\vspace{4cm}

\phantomsection
\addcontentsline{toc}{section}{References}

\makeatletter % prevent newpage within bib-item
\interlinepenalty=10000

\bibliography{../references}
\label{references}

\makeatother

\vspace{-.3cm}

\clearpage

}

{\fontsize{11pt}{16.5pt}\selectfont

\phantomsection
\addcontentsline{toc}{section}{Appendix}
\section*{Appendix}
\label{appendix}

\phantomsection
\addcontentsline{toc}{subsection}{Figures}
\subsection*{Figures}
\label{figures}

\begin{figure}[ht]
    \centering
    \caption{Cohorts (Start of Occupation by Gender), 1945–1990}
    \label{fig:mayer}
    \fontsize{9pt}{11pt}\selectfont
	\def\svgwidth{\textwidth}
	\input{validity.pdf_tex}
    \vspace{2mm}
    \parbox{10cm}{
    \linespread{1}\footnotesize Note: The data come from \cite{Mayer1995}.}
\end{figure}

\clearpage

\phantomsection
\addcontentsline{toc}{subsection}{Tables}
\subsection*{Tables}
\label{tables}

\iffalse
\begin{table}[ht]
    \caption[Self-Declared Maths Grades by Adults -- GSOEP]{Self-Declared Maths Grades by German Adults -- GSOEP}
    \label{tab:soepadults}
    \begin{center}
        \begin{tabular}{l*{3}{c}}
            \toprule
            \multicolumn{4}{c}{\emph{Dependent Variable: Self-Declared Last Maths Grade (0-1 scale)}} \\
            \midrule
                                &\multicolumn{1}{c}{(1)}         &\multicolumn{1}{c}{(2)}         &\multicolumn{1}{c}{(3)}         \\
            \midrule
            Female              &       -0.09\sym{***}&       -0.08\sym{***}&       -0.08\sym{***}\\
                                &      (0.01)         &      (0.01)         &      (0.01)         \\
            [1em]
            East                &        0.06\sym{***}&        0.08\sym{***}&        0.07\sym{***}\\
                                &      (0.01)         &      (0.01)         &      (0.01)         \\
            [1em]
            East*Female         &        0.08\sym{***}&        0.08\sym{***}&        0.08\sym{***}\\
                                &      (0.02)         &      (0.02)         &      (0.02)         \\
            \midrule
            Person \& Household Controls & & X & X \\
            Länder Controls & & & X \\
            \midrule
            Observations        &       18,607         &       18,607         &       18,607         \\
            \midrule
            \bottomrule
        \end{tabular}
        
        \vspace{2mm}
        
        \parbox{10cm}{
        \linespread{1}\footnotesize Note: \sym{*} \(p<0.10\), \sym{**} \(p<0.05\), \sym{***} \(p<0.01\). Linear Probability Model. The data come from the German Socio-Economic Panel. The sample is restricted to individuals who were born in Germany before 1971. Standard errors clustered at the household level are given in parentheses. East=1 if the household head lived in the GDR prior to Reunification in 1990. Question: ``Can you remember your last report card from school? What grade did you have in mathematics (1-6 scale with 1 being the highest grade)?'' Answer originally on a 1-6 scale recoded on a 0-1 scale, 1 corresponding to a grade of 1 or 2.}
        
    \end{center}
\end{table}
\fi

\clearpage

}
{\fontsize{11pt}{11pt}\selectfont

\phantomsection
\addcontentsline{toc}{section}{Statutory Declaration}
\section*{Statutory Declaration}
\label{declarations}

I hereby declare that I have written this thesis independently and without the use of sources and aids other than those specified. I have not used the services of any agency providing specimen, model, or ghostwritten work in the preparation of this submitted work. This also includes the use of AI-generated texts or services such as ChatGPT. Sentences or parts of sentences quoted literally are marked as such; other references with regard to the statement and scope are indicated by full details of the publications concerned. The thesis in the same or similar form has not been submitted to any examination body and has not been published. This thesis was not yet, even in part, used in another examination or as a course performance.

\vspace{1.5cm}

\includegraphics[height = 25mm]{unterschrift_schuett.pdf}

\noindent Potsdam, \today{}

} % close \fontsize{11}{16.5}

\end{document}
